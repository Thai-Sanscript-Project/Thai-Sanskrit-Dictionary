\section{Heslové slovo}

\subsection{Řazení slov }
Heslová  slova jsou zobrazena tučně a jsou seřazena podle islandské abecedy :

\textbf{a, á, b, d, ð, e, é, f, g, h, i, í, j, k, l, m, n, o, ó, p, r, s, t, u, ú, v, x, y, ý, þ, æ, ö}

\subsection{Členění heslových slov}
Uvnitř heslových slov  se vyskytují značky  \textbf{“·”} a \textbf{“|”}. Složená slova jsou rozdělena značkami \textbf{“··”}, pokud se jedná o hlavní dělení slova,  


\entry{ráð··hús}{}
Pokud se jedná o vedlejší dělení slova, je použita jen jedna značka \textbf{“·”} 


\entry{við·bótar··líf·eyris·sparnað|ur}{}
Značka \textbf{“|”} označuje místo ve slově, kde se deklinační koncovka připojuje ke slovu


\entry{heils|a} {\small{f (-u) }}
Pokud se ohýbá celá poslední část složeného slova je to označeno znaky \textbf{“··”} (hlavní dělení složeného slova) a \textbf{“|”}.


\entry{við··|bragð} {\small{n (-bragðs, -brögð)}}

\subsection{Varianty}
Některá heslová slova mají varianty. Varianty jsou uvedeny v záhlaví hesla a lze je vyhledat ve slovníku jako svébytná hesla. V online verzi lze variantu navštívit kliknutím na hypertextový odkaz. 

	
\entry{albatros|i albatros} {\small{m (-a, -ar)}}

\subsection{Homonyma }
Homonyma (heslová slova stejného reprezentativního tvaru) se označují povýšenými arabskými číslicemi uváděnými před klíčovým slovem. 


\entry{\textsuperscript{1}vor} {\small{n (-s, -)}} 
\entry{\textsuperscript{2}vor} {\small{pron poss}}
Homonyma jsou řazena podle slovního druhu. V případě dvou  podstatných jmen je pořadí mužský rod, ženský rod, střední rod. V případě dvou podstatných jmen stejného rodu je pořadí podle druhu skloňování – nejdříve je slovo se slabým skloňováním, potom se silným. V případě dvou sloves, opět má přednost sloveso se slabým časováním před slovesem se silným časováním.


\section{Slovní druhy}
Za každým klíčovým slovem se nachází zkratka, která popisuje slovní druh heslového slova. Pokud se heslové slovo řadí do více slovních druhů, jsou jednotlivé zkratky pro slovní druhy odděleny znakem \textbf{“/”}


\entry{af} {\small{prep/ adv}}
a seřazeny podle českého ustáleného řazení slovních druhů (podst. jméno, příd. jméno, zájmeno, číslovka, sloveso, příslovce, předložka, spojka, částice, citoslovce)

\subsection{Podstatná jména}
Podstatná jména jsou označená následovně: \textbf{m} – podst. jméno rodu mužského, \textbf{f} – podst. jméno rodu ženského, \textbf{n} – podst. jméno rodu středního

	
\entry{hestur} {\small{ m}}
\entry{kona } {\small{f}}
\entry{hús } {\small{n}}

\subsection{Přídavná jména}
Přídavná jména se ve slovníku vyskytují v rodě mužském. Přídavná jména jsou označena zkratkou \textbf{adj}. Druhý stupeň stupňování je označen zkratkou \textbf{comp} a třetí stupeň zkratkou \textbf{sup}


\entry{fallegur} {\small{ adj}}
\entry{fremri} {\small{ adj comp}}
\entry{einasti} {\small{ adj sup}}
Pokud se přídavné jméno pojí s určitý pádem, jsou použity zkratky \textbf{acc} (4. pád), \textbf{dat} (3.pád), \textbf{gen} (2.pád).


\entry{líkur} {\small{ adj dat}}

\subsection{Zájmena}
Zájmena jsou označena zkratkou \textbf{pron}. Dále jsou zájmena tříděna na zájmena ukazovací (\textbf{dem}), přivlastňovací (\textbf{poss}), osobní (\textbf{pers}) a neurčitá (\textbf{indef}).


\entry{\textsuperscript{2}hver} {\small{ pron}}
\entry{þessi} {\small{ pron dem}}
\entry{minn} {\small{ pron poss}}
\entry{ég} {\small{ pron pers }}
\entry{nokkur} {\small{ pron indef}}
U zájmena osobních je dále popsán pád a číslo


\entry{mig} {\small{ pron pers acc sg}}

\subsection{Číslovky}
Číslovky jsou označeny zkratkou \textbf{num}. V případě prvních čtyř číslovek (1-4) jsou ve slovníku uvedeny jako heslová slova také tvary ženského i středního rodu. V takovém případě jsou označeny zkratkou \textbf{m} pro mužský rod, zkratkou \textbf{f} pro ženský rod a zkratkou \textbf{n} pro střední rod. Zkratkou \textbf{ord} jsou označeny číslovky řadové.


\entry{tuttugu} {\small{ num}}
\entry{tveir} {\small{ num m}}
\entry{tvær} {\small{ num f}}
\entry{tvö} {\small{ num n}}
\entry{fyrsti} {\small{ num ord}}

\subsection{Slovesa}
Slovesa jsou jako heslová slova uvedena v infinitivu. Slovesa jsou označena zkratkou \textbf{v}. Mediopasiva jsou označena jako \textbf{refl} a slovesa, která jsou vždy neosobní, jsou označena \textbf{impers}


\entry{fara} {\small{ v}}
\entry{nálgast} {\small{ v refl}}
\entry{svima} {\small{ v impers}}
U sloves je dále uvedeno s jakým pádem se pojí. Zkratka \textbf{acc} označuje 4. pád, zkratka \textbf{dat} 3. pád, zkratka \textbf{gen} 2. pád, zkratka \textbf{nom} 1. pád. V případě, že se sloveso pojí s různými pády je použit znak \textbf{“/”}


\entry{klór|a} {\small{ v acc/dat}}
V případě, že se sloveso pojí s více pády je použit znak \textbf{“+”}

	
\entry{gefa} {\small{ v dat + acc}}
 
\subsection{Příslovce}
Příslovce jsou označena zkratkou \textbf{adv}. Druhý stupeň stupňování je označen zkratkou \textbf{comp} a třetí stupeň zkratkou \textbf{sup}


\entry{nýlega} {\small{ adv}}
\entry{ofar} {\small{ adv comp }}
\entry{síðast} {\small{ adv sup}}

\subsection{Předložky}
Předložky jsou označeny zkratkou \textbf{prep}. Pokud se předložka pojí výhradně s jedním pádem, jsou použity zkratky pro 4. pád (\textbf{acc}), 3. pád (\textbf{dat}), 2. pád (\textbf{gen}). V případě, že se předložky pojí s různými pády je použit znak \textbf{“/”}


\entry{af} {\small{ prep/ adv dat}}
\entry{fyrir} {\small{ prep/ adv acc/ dat}}
\entry{milli} {\small{ prep gen}}

\subsection{Spojky}
Spojky jsou označeny zkratkou \textbf{conj}

\subsection{Částice}
Částice jsou označeny zkratkou \textbf{part}

\subsection{Citoslovce}
Citoslovce jsou označeny zkratkou \textbf{inter}


\section{Skloňování a časování}
V online verzi jsou po pravé straně uvedeny kompletní deklinační tabulky pro podstatná jména, přídavná jména, zájmena, číslovky (1-4), slovesa a stupňování příslovcí. Následující příklad ukazuje skloňovací tabulku pro heslové slovo hvalur.

Deklinační tvary nebo koncovky podst. jmen, příd. jmen, zájmen, sloves jsou uvedeny v závorkách za slovním druhem. Pokud se ve slově vyskytuje změna samohlásky, je uvedena celá poslední část složeného slova. Pokud se nejedná o slovo složené a vyskytuje se změna samohlásky, pak jsou uvedeny tvary celého slova.


\entry{mynd} {\small{ f (-ar, -ir)}}
\entry{á··|lag} {\small{ n (-lags, -lög) }}
\entry{maður} {\small{ m (manns, menn)}}
V případě, že heslové slovo má více deklinačních tvarů např. pro 2. pád jednotného čísla, jsou varianty koncovky odděleny znakem \textbf{“/”}


\entry{beð|ur} {\small{ m (-s/-jar, -ir)}}

\subsection{Podstatná jména}
U podstatných jmen je ukázán tvar heslového slova pro 2. pád jednotného čísla a 1. pád množného čísla.

	
\entry{hval|ur} {\small{ m (-s, -ir)}}
kde hvals je 2. pád jednotného čísla a hvalir 1. pád množného čísla
 Pokud je uveden pouze jeden tvar, jedná se o 2. pád jednotného čísla a znamená to rovněž, že heslové slovo se nevyskytuje v množném čísle.


\entry{heisl|a} {\small{ (-u)}}
kde heislu je 2. pád jednotného čísla (množné číslo se nevyskytuje)
 Pokud není uveden žádný tvar v závorce a místo toho je použita zkratka \textbf{pl}, znamená to, že heslové slovo se vyskytuje pouze v množném čísle. 


\entry{afar··kostir} {\small{ m pl}}
Pokud je použita zkratka \textbf{indecl}, znamená to, že podst. jméno je nesklonné.


\entry{fræði} {\small{ f indecl}}

\subsection{Přídavná jména}
U přídavných jmen jsou uvedeny v závorkách pouze tvary, které nejsou tvořeny pravidelně nebo se v nich objevuje změna samohlásky. Pokud je přídavené jméno nesklonné, je uvedena zkratka \textbf{indecl}


\entry{reglu··|samur} {\small{ adj (f -söm)}}
\entry{hýr } {\small{adj (f hýr)}}
\entry{tví··mála} {\small{ adj indecl}}

\subsection{Slovesa}
Islandská slovesa se dělí na slabá a silná. Do jaké skupiny sloveso patří, lze poznat podle počtu koncovek nebo slovních tvarů uvedených v závorce.
3.3.1 Slabá slovesa, která tvoří 1. osobu jednotného čísla minulého času koncovkou -aði, mají v závorce uveden jen jednu koncovku a to jmenovitě (-aði). Příčestí minulé není ukázáno, neboť se tvoří pravidelně přidáním -ð k infinitivu slovesa .


\entry{ætl|a} {\small{ v (-aði)}}

3.3.2 U zbývajících skupin slabých sloves se v závorce nachází koncovka  1. osoby jednotného čísla minulého času a příčestí minulé ve středním rodě.


\entry{kenn|a} {\small{ v (-di, -t) }}
kde tvar kenndi je 1. osoba jednotného čísla minulého času a tvar kennt je příčestí minulé ve středním rodě. 

3.3.3 Silná a nepravidelná slovesa mají v závorce vždy čtyři tvary – jmenovitě 1. osobu jednotného čísla přítomného času, 1. osobu jednotného čísla minulého času, 1. osobu množného čísla minulého času a příčestí minulé ve středním rodě


\entry{grípa} {\small{ v (gríp, greip, gripum, gripið) acc}}

\subsection{Gramatické informace ve významech heslových slov}
U některých významů v heslovém slově se vyskytují gramatické informace, které popisují chování danného významu


\entry{batn/a} {\small{ v (-aði)}

...
\textbf{2.}\textbf{ e-m batnar }{\small{impers }}{(kdo) se uzdravuje, (komu) je lépe } 
	\textit{Mér batnaði fljótt.} \textit{Rychle jsem se uzdravil. }}
	
Zkratka \textbf{impers} označuje, že druhý význam slovesa batna se vyskytuje jako neosobní sloveso. V tomto případě je ukázáno na příkladu, v jakém pádu je podnět v mrtvé příkladě (e-m batnar) i ilustračním příkladě (mér batnaði fljót). 
Nějčastějšími gramatickými informacemi u významů v heslovém slově jsou \textbf{pl} pro množné číslo podstatného jména    


\entry{bót} {\small{ f (bótar, bætur)}

...
\textbf{bætur} \textit{(tryggingafé)} {\small{pl} {dávky, příspěvky }}}
dále \textbf{refl} pro mediopasivní tvary slovesa a \textbf{impers} pro neosobní slovesa. 


\section{Definice}

\subsection{Základní tvar}
Základním tvarem definice je, že jednomu islandskému slovu odpovídá jeden nebo více českých významů.


\entry{landa··fræði} {\small{ f indecl}

{zeměpis, geografie}}
V některých případech je v české části použita závorka. Slovo v závorce a) zužuje význam českého slova 


\entry{að··stoð} {\small{ f (-ar)} 
{(malá) pomoc }}
a b) brání dvojznačnosti


\entry{fíkj|a} {\small{ f (-u, -ur)}
\footnotesize{bot.} {fík (plod) }}

V mnoha případech je závorka v české části použita pro příklad užití českého slova a tím vymezení jeho významu

\entry{að·gengi··legur} {\small{ adj}
{přístupný (vchod ap.) }}
Značka \textbf{“/”} slouží v české části významu k oddělení více významů pojících se s jedním nebo více slovy a je použit v české části následujícím způsobem

\entry{andlits··fall} {\small{ n (-s)}
{rysy/podoba tváře }}
čte se rysy tváře, podoba tváře

\subsection{Opis heslového slova}
V případech, kdy islandské slovo nemá ekvivalent v českém jazyku, jsou významy islandských slov opsány  delším opisem, a menším písmem. Příkladem jsou slova z lidových pověstí, kulinářské speciality nebo botanické pojmy. 

\entry{til··ber|i} {\small{ m (-a, -ar)} 
\footnotesize{pov.} \textit{stvoření, které dojí krávy a ovce jiných hospodářů}}

\entry{klein|a} {\small{ f (-u, -ur)} 
\footnotesize{kulin. }\textit{druh islandské koblihy }}

\subsection{Slovní spojení}
V definici heslového slovo jsou uváděny slovní spojení. Ustálená slovní spojení jsou psány tučně a červenou barvou. Po slovním spojení následuje překlad slovního spojení. 

\entry{lær|a} {\small{v (-ði, -t) acc}
...
\textbf{læra utanbókar} {učit se zpaměti, memorovat}}
Mezi slovní spojení v tomto slovníku řadíme a) tvary heslového slova, které se chovají jako samostatné slovo b) příkladová spojení c) ustálená slovní spojení (fráze) a d) rčení, přísloví. 

4.3.1 	Tvary heslového slova, které se chovají jako samostatné slovo 
Zde řadíme např. množné číslo heslového slova, které má odlišný význam. (zkratka \textbf{pl})

\entry{bót} {\small{ f (bótar, bætur)}
...
\textbf{bætur} \textit{(tryggingafé)} {\small{ pl} {dávky, příspěvky} }}
Dále sem patří mediopasivní slovní tvar (zkratka \textbf{refl})

\entry{finna} {\small{ v (finn, fann, fundum, fundið) acc}
...
\textbf{finnast} {\small{ refl} {potkat se, setkat se, shledat se} }}
Rovněž do této skupiny patří tvar příčestí minulého v mužském rodě. Tento tvar je označen zkratkou \textbf{pp}. 

\entry{drekka} {
...
\textbf{drukkinn} {\small{ pp }} {$\rightarrow$} \textbf{drukkinn}}
V tomto případě se slovní tvar drukkinn vyskytuje ve slovníku jako heslové slovo a je použit odkaz na toto slovo. V online verzi lze navštívit slovní tvar kliknutím na hypertextový odkaz.

4.3.2	 Příkladová spojení
Příkladová spojení  jednotlivých významů jsou uváděna za významem, ke kterému patří. Příkladová spojení ilustrují komunikační chování těchto slov v příslušných významech a v různých syntagmatech, např. sloveso s určitou předložkou ap. 

\entry{leit/a} {\small{ v (-aði) gen}
...
\textbf{leita að e-u} \small{dat} {hledat (co)}
	\textit{leita að lyklunum} \textit{hledat klíče}  }

4.3.3	 Ustálená slovní spojení 
Tento druh slovních spojení je charakterizován většinou přeneseným významem ve slovním spojení a tato spojení jsou uváděna za posledním významem, viz. 6. Řazení významů v definici. 

\textbf{leita e-s dyrum og dyngjum} {hledat (co/ koho) úplně všude/ po všech čertech}

4.3.4	 Rčení a přísloví 

\entry{eik} {\small{ f (-ar/-ur, -ur)}
...
\textbf{Eplið fellur sjaldan langt frá eikinni.} \footnotesize{přís.} {Jablko nepadá daleko od stromu.}}

v plném znění je slovní spojení  halda á einhverju, přičemž einhverju zastupuje neživotné podstatné jméno v 3. pádě.
Znak \textbf{"/"} vyjadřuje různé možnosti vyjádření slovního spojení

\entry{blín|a} {\small{ v (-di, -t)}
...
\textbf{blína á e-ð/e-n} {koukat se na (co/koho)} }
se může přečíst jako blína á e-ð nebo blína á e-n
Závorky jsou použity na tu část slovního spojení, které je možné vypustit.
	
\entry{\textsuperscript{1}ár} {\small{ f (-ar, -ar) }
...
\textbf{taka (of) djúpt í árinni} \footnotesize{přen.} {přehnat (co v tvrzení)}}
se může přečíst jako taka djúpt í árinni nebo taka of djúpt í árinni

\section{Vícevýznamovost}

\subsection{Jednotlivá slova}
Velké množství islandských slov má více než jeden význam. Odlišné významy jsou uvedeny každý zvlášť a rozlišeny arabskými číslicemi a islandskými synonymy. 

\entry{land} {\small{ n (lands, lönd) }
\textbf{1.} \textit{(þurrlendi)} {souš, pevnina, země} 
\textbf{2.} \textit{(árbakki)} {břeh }
\textbf{3.} \textit{(ríki)} {země, stát }
\textbf{4.} \textit{(landareign)} {pozemek} }

\subsection{Slovní spojení}
V případě, že slovní spojení uvnitř definice má více významů, jsou jednotlivé významy odlišeny písmeny a., b., c.

\entry{drag} {\small{ n (drags, drög) }
{mokřina, podmoklý terén} 
\textbf{drög}\textbf{ a.} \textit{(uppsprettur)} {\small{ pl}} {prameny (řeky ap.)} 
\textbf{drög} \textbf{b.} \textit{(undirbúningur)} {\small{ pl }}{náčtr, návrh}}

\section{Řazení významů v definici }
V definici jsou nejříve seřazeny významy slova a odlišeny arabskými číslicemi.

\subsection{V rámci významu}
Ustálená slovní spojení jsou uváděna v rámci jednotlivého významu, pokud se s ním významově váží.   

\entry{tím|i} {\small{ m (-a, -ar) }
\textbf{1.} \textit{(tíð)} {čas} 
\textbf{í þann tíma} \textit{(\textsuperscript{2}þá)} {pak} 
\textbf{2.} \textit{(klukkustund)} {hodina (šedesát minut)} }

\subsection{Za posledním očíslovaným významem}
Pokud se slovní spojení nijak silně neváže k žádnému významu, je řazeno po posledním očíslovaném významu. Ustálená slovní spojení jsou uvedena značkou {$\diamondsuit$}

\entry{finna} {\small{ v (finn, fann, fundum, fundið) acc }
\textbf{1.} \textit{(uppgötva)} {najít, nalézt} 
...
\textbf{3.} \textit{(skynja)} {cítit, vnímat} 
...
{$\diamondsuit$}
\textbf{finna að} \textit{(gagnrýna)} {kritizovat, nacházet chyby }
...}

\subsection{Způsob řazení slovních spojení:}
Slovní spojení, které se nachází po posledním očíslovaném významu, jsou řazeny podle předložky, se kterou se pojí. Pro orientaci v rozsáhlých definicích je uvedeno heslové slovo \textbf{"+"} předložka před slovním spojením. 

\entry{\textsuperscript{2}koma} {\small{ v (kem, kom, komum, komið) dat}
...
{\small{koma + úr}}
\textbf{e-ð er komið úr móð} {(co) vychází z módy}
{\small{koma + út}}
\textbf{koma út} {vyjít, objevit se }
\textbf{koma upp úr kafinu}{ objasnit se, vyjasnit se}
...}
Potom následují slovní spojení mediopasiva řazená opět podle předložky, se kterou se pojí. 

\entry{\textsuperscript{2}koma} {\small{ v (kem, kom, komum, komið) dat}
... 	
{\small{komast}}
\textbf{komast} {\small{ refl}} {přijet, dostat se} 
...}

Poté následuje příčestí minulé v mužském rodě. V případě, že se příčestí minulé vyskytuje jako heslové slovo ve slovníku, je uveden odkaz na toto heslové slovo.

\entry{\textsuperscript{2}koma} {\small{ v (kem, kom, komum, komið) dat}
...
{\small{kominn}} 
\textbf{kominn af góðu fólki} {být z dobré rodiny}
...}

Poté následují slovní spojení, která neobsahují předložky nebo které nebylo možné zařadit.

\entry{\textsuperscript{2}koma} {\small{ v (kem, kom, komum, komið) dat}
... 
{\small{Slovní spojení}}

\textbf{koma e-u heim og saman} \textit{(samrýma)} {uvést v soulad} 
...}

Nakonec přicházejí přísloví nebo rčení. 

\entry{\textsuperscript{2}koma} {\small{ v (kem, kom, komum, komið) dat}
...
{\small{Přísloví}}

\textbf{Ekki (eigi) fellur eik við fyrsta högg.} \footnotesize{přís.} {Jedna vlaštovka jaro nedělá.} }

\section{Užití synonym a antonym}

\subsection{Synonyma}
Před opisem významu heslového slova jsou uváděny v závorkách synonyma nebo synonymní slovní spojení. Islandská synonyma slouží a) islandskému uživateli k rozlišení českých významů b) českému uživateli  jako dodatečná informace o významu slova. V online verzi se přechod na význam synonyma uskutečňuje kliknutím na hypertextový odkaz. 

\entry{staða} {\small{ f (stöðu, stöður) }
\textbf{1.} \textit{(ástand)} {situace} 
\textbf{2.} \textit{(starf)} {pozice (ve firmě ap.), profese}}
V případě, že je uveden synonymní slovní spojení je v závorce se znakem \textbf{"*"} za tímto spojením uveden odkaz na heslové slovo ve slovníku. V online verzi je tento odkaz realizován hypertextovým odkazem a není pro to nutné odkaz uvádět.
V tištěné verzi

\entry{bygging} {\small{ f (-ar, -ar) }
\textbf{1.}\textit{(það að byggja)(* byggja)} {(vý)stavba, stavení} 
\textbf{2.} \textit{(hús)} {budova }}

\subsection{Antonyma}
Antonyma jsou uváděna za opisem významu heslového slova se značkou \textbf{x}. Antonyma pomáhají vymezovat význam slova a také slouží k větší provázanosti slov. V online verzi je přenesení na antonymum realizován kliknutím na hypertextový odkaz.

\entry{kaldur} {\small{ adj (f köld) }
\textbf{1.} {studený, chladný} \textit{x (heitur)}}

\section{Oborové a stylové charakteristiky}
K upřesnění významu heslového slova slouží oborové a stylové charakteristiky.

\subsection{Oborové charakteristiky}
Oborové charakteristiky plní více funkcí
a) Oborové charakteristiky zařazují islandské slovo do určitého oboru a naznačují použití danného slova v jazyce.

\entry{berg } {\small{n (-s, -)}
\textbf{1.} \footnotesize{ geol. } {hornina, kámen} 
\textbf{2. } \textit{(klöpp)} { skalní stěna } }
b) Oborové charakteristiky vymezují význam českého slova,

\entry{mús} {\small{ f (músar, mýs)}
\textbf{1.} \footnotesize{ zool.} { myš } (Mus)
\textbf{2.} \footnotesize{ poč.} { myš } }

\subsection{Stylové charakteristiky}
Stylové charakteristiky jsou uváděny v případu, kdy význam heslového slova je možné použít pouze při určité příležitosti nebo je význam jazykově zabarven.

\entry{dís} {\small{ f (-ar, -ir) }
...
\textbf{2.} \footnotesize{básn.} {sestra} }

\section{Syntax}
Informace o syntaxi se slovníku vyskytuje ve třech různých zápisech. 1) Informace uvedené zkratkou 2) Mrtvé příklady 3) Ilustrační příklady. 
Jaká informace o syntaxi je ve slovníku obsažena a co je možno vyčíst z informace?

Informace o syntaxi popisují
a) zda-li je sloveso tranzitivní nebo intranzitivní a s jakým(i) pádem (pády) se pojí, pokud je tranzitivní
b) zda-li je podmět a předmět slovesa živý či neživý
c) zda-li je sloveso osobní či neosobní a v jakém pádu je podmět, pokud je sloveso neosobní
d) zda-li je podmět gramatický? (gerfifrumlag)
e) zda-li je sloveso použito v mediopasivním tvaru
f) informace o významu - nejčastější příklady, které se pojí se slovesem
g) s kterými příslovci nebo předložkami se sloveso pojí
h) samostatná definice pro příčestí minulé nebo přítomné
\subsection{Informace o syntaxi uvedené zkratkou (viz. oddíl 2. Slovní druhy)}
Zde máme na mysli informace o pádě (nom, acc, dat, gen), slovesu v mediopasivním tvaru (refl), neosobním slovesu (impers), slovesu počasí (met), čísle (sg, pl) ap.
\subsection{Mrtvý příklad}
Mrtvým příkladem nazýváme příklad, kdy předmět je zastoupen zájmenem neurčitým v pádě, s kterým se slovo pojí a slovesa je zpravidla v infinitivu. V případě, že podmět není v 1. pádě (např. u sloves neosobních), je sloveso uvedeno v 3. osobě jednotného čísla přítomného času.
Ve slovních spojeních používáme zkratky pro tvary islandského slova einhver (někdo) a eitthvað (něco). Tyto zkratky vyjadřují a) pád a b) životnost nebo neživotnost předmětu
Vysvětlení zkratek: 
\textbf{e-ð}  - eitthvað (4. pád, neživotný předmět), \textbf{e-n}  - einhvern (4. pád, životný předmět), \textbf{e-u} -einhverju (3. pád, neživotný předmět), \textbf{e-m} – einhverjum (3. pád, životný předmět),  \textbf{e-s} einhvers (2. pád, neživotný předmět), \textbf{e-rs} einhvers (2. pád, životný předmět)

\entry{
leit|a } {\small{v (-aði) gen}
...
{\small{leita + að}}
\textbf{leita að e-u/ e-m} {hledat (co/ koho)}
	\textit{
leita að lyklunum} \textit{ hledat klíče  }}
Příklad tedy přečteme jako leita að einhverju (hledat co) a leita að einhverjum (hledat koho). V tomto případě se tedy sloveso pojí s 3. pádem a předmět může být životný i neživotný.

\subsection{Ilustrační příklad}
Ve slovníku se vyskytuje velké množství ilustračních příkladů a jejich překladů. Ilustrační příklad se liší od příkladu mrtvého tím, že místo zájmen neurčitých se vyskytuje frekventované slovo v bězném projevu. Ilustrační příklady slouží k ilustraci použití
heslového slova. Snahou bylo uvést příklady, které jsou časté v běžné řeči.
Ilustračním příkladem může být a)(složené) slovo, b) slovní spojení, c) věta 

9.3.1	 Slovo jako ilustrační příklad 

\entry{
hett/a } {\small{f (-u, -ur)}
\textbf{1. } {kapuce}
\textit{hettuúlpa} \textit{ bunda s kapucí  }}

9.3.2	 Slovní spojení jako ilustrační příklad 

\entry{
tím|i  } {\small{m (-a, -ar)}
...
\textbf{2. } \textit{ (klukkustund) }{hodina (šedesát minut)}
\textit{tveggja tíma gangur} \textit{ dvouhodinový pochod  }}

9.3.3	 Věta jako ilustrační příklad 

\entry{
mæt|a   } {\small{v (-ti, -t) dat}
...
\textbf{mætast } \textit{ (ná saman) }
{\small{ refl }}{setkat se}
\textit{Þau mættust á miðri leið.} \textit{  Potkali se na půli cesty. }} 

\section{Fotografie, obrázky a ilustrace}
V online verzi je publikováno značné množství fotografií zejména rostlin a živočichů ale i fotografií, které dokumentují místopisně heslová slova (např. knihovnu, poštu, řeku nebo islandské kulinářské speciality (různé druhy koláčů, dortů, koblih ap.)).

Fotografie z oblasti biologie a botaniky jsou dvojího druhu. Buď a) odkazují hypertextově na www.biolib.cz (Databázi rostlin a živočichů), kde je možné se dozvědět více o daném druhu rostliny nebo živočicha nebo b) po kliknutí na miniaturní fotografii se otevře detailní fotografie ve vyskakovacím okně (kliknutím poza fotografii se okno s fotografií automaticky zavře).
Všechny fotografie, které se vyskytují ve slovníku, jsou publikovány pod veřejnými licencemi a jsou uvedeny s názvem autora a licence.

\par\begin{center}\setlength\fboxsep{0pt}\setlength\fboxrule{0.5pt}\fbox{\includegraphics[width=6cm]{/home/chejnik/Dokumenty/web/HVALUR-JOINED/www/images/uploaded_files/ds_image_posthus_0_2.jpg}}\end{center}
\par\begin{center}\footnotesize {Autor:hvalur.org Licence: CC Unported Licence}\end{center}

\section{Výslovnost slov }
V hranatých závorkách je uveden IPA zápis výslovnosti heslového slova (např. heslové slovo úlfur - [ulv{\textscy}r]). Dvojtečka označuje dlouhou samohlásku nebo souhlásku (např. renna - [r{\textepsilon}n{\textlengthmark}a]). V IPA zápise nejsou zaznačeny hlavní a vedlejší přízvuky.


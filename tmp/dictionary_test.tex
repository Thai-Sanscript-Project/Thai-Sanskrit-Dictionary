\documentclass[twocolumn]{book}
\usepackage[top=2cm, bottom=2cm, left=2cm, right=2cm]{geometry}
\usepackage{fancyhdr}
\usepackage[icelandic, czech, english]{babel}

\usepackage[utf8x, utf8]{inputenc}
\usepackage[T1]{fontenc}
\usepackage{enumitem}
\usepackage{hanging}
\newcommand{\entry}[2]{\hangpara{2em}{1}\textsf{\textbf{#1}}\ #2\markboth{#1}{#1}\par}\nopagebreak[4]
\newcommand*{\dictchar}[1]{\centerline{\LARGE\textbf{#1}}\par}
\pagestyle{fancy}
\fancypagestyle{basicstyle}{%
 \fancyhf{}
  \renewcommand{\headrulewidth}{0.4pt}
  \renewcommand{\footrulewidth}{0pt}
  \fancyhead[LE,RO]{\textsf{\textbf{\chaptitle}}}
  \fancyhead[LO,RE]{\textsf{\textbf{\thepage}}}}
\fancypagestyle{dictstyle}{%

  \fancyhf{}
  \fancyhead[LE,RO]{\textsf{\textbf{\rightmark\ -- \leftmark}}}
  \fancyhead[LO,RE]{\textsf{\textbf{\thepage}}}}
\usepackage{tipa}
\usepackage{fix2col}
\usepackage{marvosym}
\usepackage{dingbat}
\usepackage{manfnt}
\usepackage{latexsym}
\usepackage{graphicx}
\usepackage{color}

\usepackage{titlesec}
\titleformat{\chapter}[block]
  {\normalfont\huge\bfseries}{\thechapter.}{1em}{\Huge}
\titlespacing*{\chapter}{20pt}{20pt}{20pt}
\definecolor{darkgreen}{rgb}{0.80,0.40,0.11}
\graphicspath{{/home/chejnik/Dokumenty/web/HVALUR-JOINED/www/images/biolib/full/}{/home/chejnik/Dokumenty/web/HVALUR-JOINED/www/images/uploaded_files/}}

\title{\textbf{Islandsko-český studijní slovník}
       \thanks{Tato kniha byla vytvořena v \LaTeX{}u pod Ubuntu 10.04. Poděkování patří všem autorům, kteří publikují pod svobodnými licencemi.}}
\author{Aleš Chejn, Jón Gíslason}

\date{říjen 2011}

\begin{document}
 \makeatletter\@openrightfalse
\maketitle
\@openrighttrue\makeatother
\renewcommand {\contentsname} {Obsah}
\renewcommand {\chaptername} {Kapitola}
 \makeatletter\@openrightfalse
\tableofcontents
\@openrighttrue\makeatother
\newpage\pagestyle{basicstyle}

\newcommand*{\sectitle}{}

\renewcommand*{\sectionmark}[1]{%
    \renewcommand*{\sectitle}{#1}}
\newcommand*{\chaptitle}{}
\renewcommand*{\chaptermark}[1]{%
    \renewcommand*{\chaptitle}{#1}}
    \makeatletter\@openrightfalse
\chapter{Průvodce po slovníku}
\@openrighttrue\makeatother

\section{Heslové slovo}

\subsection{Řazení slov }
Heslová  slova jsou zobrazena tučně a jsou seřazena podle islandské abecedy :

\textbf{a, á, b, d, ð, e, é, f, g, h, i, í, j, k, l, m, n, o, ó, p, r, s, t, u, ú, v, x, y, ý, þ, æ, ö}

\subsection{Členění heslových slov}
Uvnitř heslových slov  se vyskytují značky  \textbf{“·”} a \textbf{“|”}. Složená slova jsou rozdělena značkami \textbf{“··”}, pokud se jedná o hlavní dělení slova,  


\entry{ráð··hús}{}
Pokud se jedná o vedlejší dělení slova, je použita jen jedna značka \textbf{“·”} 


\entry{við·bótar··líf·eyris·sparnað|ur}{}
Značka \textbf{“|”} označuje místo ve slově, kde se deklinační koncovka připojuje ke slovu


\entry{heils|a} {\small{f (-u) }}
Pokud se ohýbá celá poslední část složeného slova je to označeno znaky \textbf{“··”} (hlavní dělení složeného slova) a \textbf{“|”}.


\entry{við··|bragð} {\small{n (-bragðs, -brögð)}}

\subsection{Varianty}
Některá heslová slova mají varianty. Varianty jsou uvedeny v záhlaví hesla a lze je vyhledat ve slovníku jako svébytná hesla. V online verzi lze variantu navštívit kliknutím na hypertextový odkaz. 

	
\entry{albatros|i albatros} {\small{m (-a, -ar)}}

\subsection{Homonyma }
Homonyma (heslová slova stejného reprezentativního tvaru) se označují povýšenými arabskými číslicemi uváděnými před klíčovým slovem. 


\entry{\textsuperscript{1}vor} {\small{n (-s, -)}} 
\entry{\textsuperscript{2}vor} {\small{pron poss}}
Homonyma jsou řazena podle slovního druhu. V případě dvou  podstatných jmen je pořadí mužský rod, ženský rod, střední rod. V případě dvou podstatných jmen stejného rodu je pořadí podle druhu skloňování – nejdříve je slovo se slabým skloňováním, potom se silným. V případě dvou sloves, opět má přednost sloveso se slabým časováním před slovesem se silným časováním.


\section{Slovní druhy}
Za každým klíčovým slovem se nachází zkratka, která popisuje slovní druh heslového slova. Pokud se heslové slovo řadí do více slovních druhů, jsou jednotlivé zkratky pro slovní druhy odděleny znakem \textbf{“/”}


\entry{af} {\small{prep/ adv}}
a seřazeny podle českého ustáleného řazení slovních druhů (podst. jméno, příd. jméno, zájmeno, číslovka, sloveso, příslovce, předložka, spojka, částice, citoslovce)

\subsection{Podstatná jména}
Podstatná jména jsou označená následovně: \textbf{m} – podst. jméno rodu mužského, \textbf{f} – podst. jméno rodu ženského, \textbf{n} – podst. jméno rodu středního

	
\entry{hestur} {\small{ m}}
\entry{kona } {\small{f}}
\entry{hús } {\small{n}}

\subsection{Přídavná jména}
Přídavná jména se ve slovníku vyskytují v rodě mužském. Přídavná jména jsou označena zkratkou \textbf{adj}. Druhý stupeň stupňování je označen zkratkou \textbf{comp} a třetí stupeň zkratkou \textbf{sup}


\entry{fallegur} {\small{ adj}}
\entry{fremri} {\small{ adj comp}}
\entry{einasti} {\small{ adj sup}}
Pokud se přídavné jméno pojí s určitý pádem, jsou použity zkratky \textbf{acc} (4. pád), \textbf{dat} (3.pád), \textbf{gen} (2.pád).


\entry{líkur} {\small{ adj dat}}

\subsection{Zájmena}
Zájmena jsou označena zkratkou \textbf{pron}. Dále jsou zájmena tříděna na zájmena ukazovací (\textbf{dem}), přivlastňovací (\textbf{poss}), osobní (\textbf{pers}) a neurčitá (\textbf{indef}).


\entry{\textsuperscript{2}hver} {\small{ pron}}
\entry{þessi} {\small{ pron dem}}
\entry{minn} {\small{ pron poss}}
\entry{ég} {\small{ pron pers }}
\entry{nokkur} {\small{ pron indef}}
U zájmena osobních je dále popsán pád a číslo


\entry{mig} {\small{ pron pers acc sg}}

\subsection{Číslovky}
Číslovky jsou označeny zkratkou \textbf{num}. V případě prvních čtyř číslovek (1-4) jsou ve slovníku uvedeny jako heslová slova také tvary ženského i středního rodu. V takovém případě jsou označeny zkratkou \textbf{m} pro mužský rod, zkratkou \textbf{f} pro ženský rod a zkratkou \textbf{n} pro střední rod. Zkratkou \textbf{ord} jsou označeny číslovky řadové.


\entry{tuttugu} {\small{ num}}
\entry{tveir} {\small{ num m}}
\entry{tvær} {\small{ num f}}
\entry{tvö} {\small{ num n}}
\entry{fyrsti} {\small{ num ord}}

\subsection{Slovesa}
Slovesa jsou jako heslová slova uvedena v infinitivu. Slovesa jsou označena zkratkou \textbf{v}. Mediopasiva jsou označena jako \textbf{refl} a slovesa, která jsou vždy neosobní, jsou označena \textbf{impers}


\entry{fara} {\small{ v}}
\entry{nálgast} {\small{ v refl}}
\entry{svima} {\small{ v impers}}
U sloves je dále uvedeno s jakým pádem se pojí. Zkratka \textbf{acc} označuje 4. pád, zkratka \textbf{dat} 3. pád, zkratka \textbf{gen} 2. pád, zkratka \textbf{nom} 1. pád. V případě, že se sloveso pojí s různými pády je použit znak \textbf{“/”}


\entry{klór|a} {\small{ v acc/dat}}
V případě, že se sloveso pojí s více pády je použit znak \textbf{“+”}

	
\entry{gefa} {\small{ v dat + acc}}
 
\subsection{Příslovce}
Příslovce jsou označena zkratkou \textbf{adv}. Druhý stupeň stupňování je označen zkratkou \textbf{comp} a třetí stupeň zkratkou \textbf{sup}


\entry{nýlega} {\small{ adv}}
\entry{ofar} {\small{ adv comp }}
\entry{síðast} {\small{ adv sup}}

\subsection{Předložky}
Předložky jsou označeny zkratkou \textbf{prep}. Pokud se předložka pojí výhradně s jedním pádem, jsou použity zkratky pro 4. pád (\textbf{acc}), 3. pád (\textbf{dat}), 2. pád (\textbf{gen}). V případě, že se předložky pojí s různými pády je použit znak \textbf{“/”}


\entry{af} {\small{ prep/ adv dat}}
\entry{fyrir} {\small{ prep/ adv acc/ dat}}
\entry{milli} {\small{ prep gen}}

\subsection{Spojky}
Spojky jsou označeny zkratkou \textbf{conj}

\subsection{Částice}
Částice jsou označeny zkratkou \textbf{part}

\subsection{Citoslovce}
Citoslovce jsou označeny zkratkou \textbf{inter}


\section{Skloňování a časování}
V online verzi jsou po pravé straně uvedeny kompletní deklinační tabulky pro podstatná jména, přídavná jména, zájmena, číslovky (1-4), slovesa a stupňování příslovcí. Následující příklad ukazuje skloňovací tabulku pro heslové slovo hvalur.

Deklinační tvary nebo koncovky podst. jmen, příd. jmen, zájmen, sloves jsou uvedeny v závorkách za slovním druhem. Pokud se ve slově vyskytuje změna samohlásky, je uvedena celá poslední část složeného slova. Pokud se nejedná o slovo složené a vyskytuje se změna samohlásky, pak jsou uvedeny tvary celého slova.


\entry{mynd} {\small{ f (-ar, -ir)}}
\entry{á··|lag} {\small{ n (-lags, -lög) }}
\entry{maður} {\small{ m (manns, menn)}}
V případě, že heslové slovo má více deklinačních tvarů např. pro 2. pád jednotného čísla, jsou varianty koncovky odděleny znakem \textbf{“/”}


\entry{beð|ur} {\small{ m (-s/-jar, -ir)}}

\subsection{Podstatná jména}
U podstatných jmen je ukázán tvar heslového slova pro 2. pád jednotného čísla a 1. pád množného čísla.

	
\entry{hval|ur} {\small{ m (-s, -ir)}}
kde hvals je 2. pád jednotného čísla a hvalir 1. pád množného čísla
 Pokud je uveden pouze jeden tvar, jedná se o 2. pád jednotného čísla a znamená to rovněž, že heslové slovo se nevyskytuje v množném čísle.


\entry{heisl|a} {\small{ (-u)}}
kde heislu je 2. pád jednotného čísla (množné číslo se nevyskytuje)
 Pokud není uveden žádný tvar v závorce a místo toho je použita zkratka \textbf{pl}, znamená to, že heslové slovo se vyskytuje pouze v množném čísle. 


\entry{afar··kostir} {\small{ m pl}}
Pokud je použita zkratka \textbf{indecl}, znamená to, že podst. jméno je nesklonné.


\entry{fræði} {\small{ f indecl}}

\subsection{Přídavná jména}
U přídavných jmen jsou uvedeny v závorkách pouze tvary, které nejsou tvořeny pravidelně nebo se v nich objevuje změna samohlásky. Pokud je přídavené jméno nesklonné, je uvedena zkratka \textbf{indecl}


\entry{reglu··|samur} {\small{ adj (f -söm)}}
\entry{hýr } {\small{adj (f hýr)}}
\entry{tví··mála} {\small{ adj indecl}}

\subsection{Slovesa}
Islandská slovesa se dělí na slabá a silná. Do jaké skupiny sloveso patří, lze poznat podle počtu koncovek nebo slovních tvarů uvedených v závorce.
3.3.1 Slabá slovesa, která tvoří 1. osobu jednotného čísla minulého času koncovkou -aði, mají v závorce uveden jen jednu koncovku a to jmenovitě (-aði). Příčestí minulé není ukázáno, neboť se tvoří pravidelně přidáním -ð k infinitivu slovesa .


\entry{ætl|a} {\small{ v (-aði)}}

3.3.2 U zbývajících skupin slabých sloves se v závorce nachází koncovka  1. osoby jednotného čísla minulého času a příčestí minulé ve středním rodě.


\entry{kenn|a} {\small{ v (-di, -t) }}
kde tvar kenndi je 1. osoba jednotného čísla minulého času a tvar kennt je příčestí minulé ve středním rodě. 

3.3.3 Silná a nepravidelná slovesa mají v závorce vždy čtyři tvary – jmenovitě 1. osobu jednotného čísla přítomného času, 1. osobu jednotného čísla minulého času, 1. osobu množného čísla minulého času a příčestí minulé ve středním rodě


\entry{grípa} {\small{ v (gríp, greip, gripum, gripið) acc}}

\subsection{Gramatické informace ve významech heslových slov}
U některých významů v heslovém slově se vyskytují gramatické informace, které popisují chování danného významu


\entry{batn/a} {\small{ v (-aði)}

...
\textbf{2.}\textbf{ e-m batnar }{\small{impers }}{(kdo) se uzdravuje, (komu) je lépe } 
	\textit{Mér batnaði fljótt.} \textit{Rychle jsem se uzdravil. }}
	
Zkratka \textbf{impers} označuje, že druhý význam slovesa batna se vyskytuje jako neosobní sloveso. V tomto případě je ukázáno na příkladu, v jakém pádu je podnět v mrtvé příkladě (e-m batnar) i ilustračním příkladě (mér batnaði fljót). 
Nějčastějšími gramatickými informacemi u významů v heslovém slově jsou \textbf{pl} pro množné číslo podstatného jména    


\entry{bót} {\small{ f (bótar, bætur)}

...
\textbf{bætur} \textit{(tryggingafé)} {\small{pl} {dávky, příspěvky }}}
dále \textbf{refl} pro mediopasivní tvary slovesa a \textbf{impers} pro neosobní slovesa. 


\section{Definice}

\subsection{Základní tvar}
Základním tvarem definice je, že jednomu islandskému slovu odpovídá jeden nebo více českých významů.


\entry{landa··fræði} {\small{ f indecl}

{zeměpis, geografie}}
V některých případech je v české části použita závorka. Slovo v závorce a) zužuje význam českého slova 


\entry{að··stoð} {\small{ f (-ar)} 
{(malá) pomoc }}
a b) brání dvojznačnosti


\entry{fíkj|a} {\small{ f (-u, -ur)}
\footnotesize{bot.} {fík (plod) }}

V mnoha případech je závorka v české části použita pro příklad užití českého slova a tím vymezení jeho významu

\entry{að·gengi··legur} {\small{ adj}
{přístupný (vchod ap.) }}
Značka \textbf{“/”} slouží v české části významu k oddělení více významů pojících se s jedním nebo více slovy a je použit v české části následujícím způsobem

\entry{andlits··fall} {\small{ n (-s)}
{rysy/podoba tváře }}
čte se rysy tváře, podoba tváře

\subsection{Opis heslového slova}
V případech, kdy islandské slovo nemá ekvivalent v českém jazyku, jsou významy islandských slov opsány  delším opisem, a menším písmem. Příkladem jsou slova z lidových pověstí, kulinářské speciality nebo botanické pojmy. 

\entry{til··ber|i} {\small{ m (-a, -ar)} 
\footnotesize{pov.} \textit{stvoření, které dojí krávy a ovce jiných hospodářů}}

\entry{klein|a} {\small{ f (-u, -ur)} 
\footnotesize{kulin. }\textit{druh islandské koblihy }}

\subsection{Slovní spojení}
V definici heslového slovo jsou uváděny slovní spojení. Ustálená slovní spojení jsou psány tučně a červenou barvou. Po slovním spojení následuje překlad slovního spojení. 

\entry{lær|a} {\small{v (-ði, -t) acc}
...
\textbf{læra utanbókar} {učit se zpaměti, memorovat}}
Mezi slovní spojení v tomto slovníku řadíme a) tvary heslového slova, které se chovají jako samostatné slovo b) příkladová spojení c) ustálená slovní spojení (fráze) a d) rčení, přísloví. 

4.3.1 	Tvary heslového slova, které se chovají jako samostatné slovo 
Zde řadíme např. množné číslo heslového slova, které má odlišný význam. (zkratka \textbf{pl})

\entry{bót} {\small{ f (bótar, bætur)}
...
\textbf{bætur} \textit{(tryggingafé)} {\small{ pl} {dávky, příspěvky} }}
Dále sem patří mediopasivní slovní tvar (zkratka \textbf{refl})

\entry{finna} {\small{ v (finn, fann, fundum, fundið) acc}
...
\textbf{finnast} {\small{ refl} {potkat se, setkat se, shledat se} }}
Rovněž do této skupiny patří tvar příčestí minulého v mužském rodě. Tento tvar je označen zkratkou \textbf{pp}. 

\entry{drekka} {
...
\textbf{drukkinn} {\small{ pp }} {$\rightarrow$} \textbf{drukkinn}}
V tomto případě se slovní tvar drukkinn vyskytuje ve slovníku jako heslové slovo a je použit odkaz na toto slovo. V online verzi lze navštívit slovní tvar kliknutím na hypertextový odkaz.

4.3.2	 Příkladová spojení
Příkladová spojení  jednotlivých významů jsou uváděna za významem, ke kterému patří. Příkladová spojení ilustrují komunikační chování těchto slov v příslušných významech a v různých syntagmatech, např. sloveso s určitou předložkou ap. 

\entry{leit/a} {\small{ v (-aði) gen}
...
\textbf{leita að e-u} \small{dat} {hledat (co)}
	\textit{leita að lyklunum} \textit{hledat klíče}  }

4.3.3	 Ustálená slovní spojení 
Tento druh slovních spojení je charakterizován většinou přeneseným významem ve slovním spojení a tato spojení jsou uváděna za posledním významem, viz. 6. Řazení významů v definici. 

\textbf{leita e-s dyrum og dyngjum} {hledat (co/ koho) úplně všude/ po všech čertech}

4.3.4	 Rčení a přísloví 

\entry{eik} {\small{ f (-ar/-ur, -ur)}
...
\textbf{Eplið fellur sjaldan langt frá eikinni.} \footnotesize{přís.} {Jablko nepadá daleko od stromu.}}

v plném znění je slovní spojení  halda á einhverju, přičemž einhverju zastupuje neživotné podstatné jméno v 3. pádě.
Znak \textbf{"/"} vyjadřuje různé možnosti vyjádření slovního spojení

\entry{blín|a} {\small{ v (-di, -t)}
...
\textbf{blína á e-ð/e-n} {koukat se na (co/koho)} }
se může přečíst jako blína á e-ð nebo blína á e-n
Závorky jsou použity na tu část slovního spojení, které je možné vypustit.
	
\entry{\textsuperscript{1}ár} {\small{ f (-ar, -ar) }
...
\textbf{taka (of) djúpt í árinni} \footnotesize{přen.} {přehnat (co v tvrzení)}}
se může přečíst jako taka djúpt í árinni nebo taka of djúpt í árinni

\section{Vícevýznamovost}

\subsection{Jednotlivá slova}
Velké množství islandských slov má více než jeden význam. Odlišné významy jsou uvedeny každý zvlášť a rozlišeny arabskými číslicemi a islandskými synonymy. 

\entry{land} {\small{ n (lands, lönd) }
\textbf{1.} \textit{(þurrlendi)} {souš, pevnina, země} 
\textbf{2.} \textit{(árbakki)} {břeh }
\textbf{3.} \textit{(ríki)} {země, stát }
\textbf{4.} \textit{(landareign)} {pozemek} }

\subsection{Slovní spojení}
V případě, že slovní spojení uvnitř definice má více významů, jsou jednotlivé významy odlišeny písmeny a., b., c.

\entry{drag} {\small{ n (drags, drög) }
{mokřina, podmoklý terén} 
\textbf{drög}\textbf{ a.} \textit{(uppsprettur)} {\small{ pl}} {prameny (řeky ap.)} 
\textbf{drög} \textbf{b.} \textit{(undirbúningur)} {\small{ pl }}{náčtr, návrh}}

\section{Řazení významů v definici }
V definici jsou nejříve seřazeny významy slova a odlišeny arabskými číslicemi.

\subsection{V rámci významu}
Ustálená slovní spojení jsou uváděna v rámci jednotlivého významu, pokud se s ním významově váží.   

\entry{tím|i} {\small{ m (-a, -ar) }
\textbf{1.} \textit{(tíð)} {čas} 
\textbf{í þann tíma} \textit{(\textsuperscript{2}þá)} {pak} 
\textbf{2.} \textit{(klukkustund)} {hodina (šedesát minut)} }

\subsection{Za posledním očíslovaným významem}
Pokud se slovní spojení nijak silně neváže k žádnému významu, je řazeno po posledním očíslovaném významu. Ustálená slovní spojení jsou uvedena značkou {$\diamondsuit$}

\entry{finna} {\small{ v (finn, fann, fundum, fundið) acc }
\textbf{1.} \textit{(uppgötva)} {najít, nalézt} 
...
\textbf{3.} \textit{(skynja)} {cítit, vnímat} 
...
{$\diamondsuit$}
\textbf{finna að} \textit{(gagnrýna)} {kritizovat, nacházet chyby }
...}

\subsection{Způsob řazení slovních spojení:}
Slovní spojení, které se nachází po posledním očíslovaném významu, jsou řazeny podle předložky, se kterou se pojí. Pro orientaci v rozsáhlých definicích je uvedeno heslové slovo \textbf{"+"} předložka před slovním spojením. 

\entry{\textsuperscript{2}koma} {\small{ v (kem, kom, komum, komið) dat}
...
{\small{koma + úr}}
\textbf{e-ð er komið úr móð} {(co) vychází z módy}
{\small{koma + út}}
\textbf{koma út} {vyjít, objevit se }
\textbf{koma upp úr kafinu}{ objasnit se, vyjasnit se}
...}
Potom následují slovní spojení mediopasiva řazená opět podle předložky, se kterou se pojí. 

\entry{\textsuperscript{2}koma} {\small{ v (kem, kom, komum, komið) dat}
... 	
{\small{komast}}
\textbf{komast} {\small{ refl}} {přijet, dostat se} 
...}

Poté následuje příčestí minulé v mužském rodě. V případě, že se příčestí minulé vyskytuje jako heslové slovo ve slovníku, je uveden odkaz na toto heslové slovo.

\entry{\textsuperscript{2}koma} {\small{ v (kem, kom, komum, komið) dat}
...
{\small{kominn}} 
\textbf{kominn af góðu fólki} {být z dobré rodiny}
...}

Poté následují slovní spojení, která neobsahují předložky nebo které nebylo možné zařadit.

\entry{\textsuperscript{2}koma} {\small{ v (kem, kom, komum, komið) dat}
... 
{\small{Slovní spojení}}

\textbf{koma e-u heim og saman} \textit{(samrýma)} {uvést v soulad} 
...}

Nakonec přicházejí přísloví nebo rčení. 

\entry{\textsuperscript{2}koma} {\small{ v (kem, kom, komum, komið) dat}
...
{\small{Přísloví}}

\textbf{Ekki (eigi) fellur eik við fyrsta högg.} \footnotesize{přís.} {Jedna vlaštovka jaro nedělá.} }

\section{Užití synonym a antonym}

\subsection{Synonyma}
Před opisem významu heslového slova jsou uváděny v závorkách synonyma nebo synonymní slovní spojení. Islandská synonyma slouží a) islandskému uživateli k rozlišení českých významů b) českému uživateli  jako dodatečná informace o významu slova. V online verzi se přechod na význam synonyma uskutečňuje kliknutím na hypertextový odkaz. 

\entry{staða} {\small{ f (stöðu, stöður) }
\textbf{1.} \textit{(ástand)} {situace} 
\textbf{2.} \textit{(starf)} {pozice (ve firmě ap.), profese}}
V případě, že je uveden synonymní slovní spojení je v závorce se znakem \textbf{"*"} za tímto spojením uveden odkaz na heslové slovo ve slovníku. V online verzi je tento odkaz realizován hypertextovým odkazem a není pro to nutné odkaz uvádět.
V tištěné verzi

\entry{bygging} {\small{ f (-ar, -ar) }
\textbf{1.}\textit{(það að byggja)(* byggja)} {(vý)stavba, stavení} 
\textbf{2.} \textit{(hús)} {budova }}

\subsection{Antonyma}
Antonyma jsou uváděna za opisem významu heslového slova se značkou \textbf{x}. Antonyma pomáhají vymezovat význam slova a také slouží k větší provázanosti slov. V online verzi je přenesení na antonymum realizován kliknutím na hypertextový odkaz.

\entry{kaldur} {\small{ adj (f köld) }
\textbf{1.} {studený, chladný} \textit{x (heitur)}}

\section{Oborové a stylové charakteristiky}
K upřesnění významu heslového slova slouží oborové a stylové charakteristiky.

\subsection{Oborové charakteristiky}
Oborové charakteristiky plní více funkcí
a) Oborové charakteristiky zařazují islandské slovo do určitého oboru a naznačují použití danného slova v jazyce.

\entry{berg } {\small{n (-s, -)}
\textbf{1.} \footnotesize{ geol. } {hornina, kámen} 
\textbf{2. } \textit{(klöpp)} { skalní stěna } }
b) Oborové charakteristiky vymezují význam českého slova,

\entry{mús} {\small{ f (músar, mýs)}
\textbf{1.} \footnotesize{ zool.} { myš } (Mus)
\textbf{2.} \footnotesize{ poč.} { myš } }

\subsection{Stylové charakteristiky}
Stylové charakteristiky jsou uváděny v případu, kdy význam heslového slova je možné použít pouze při určité příležitosti nebo je význam jazykově zabarven.

\entry{dís} {\small{ f (-ar, -ir) }
...
\textbf{2.} \footnotesize{básn.} {sestra} }

\section{Syntax}
Informace o syntaxi se slovníku vyskytuje ve třech různých zápisech. 1) Informace uvedené zkratkou 2) Mrtvé příklady 3) Ilustrační příklady. 
Jaká informace o syntaxi je ve slovníku obsažena a co je možno vyčíst z informace?

Informace o syntaxi popisují
a) zda-li je sloveso tranzitivní nebo intranzitivní a s jakým(i) pádem (pády) se pojí, pokud je tranzitivní
b) zda-li je podmět a předmět slovesa živý či neživý
c) zda-li je sloveso osobní či neosobní a v jakém pádu je podmět, pokud je sloveso neosobní
d) zda-li je podmět gramatický? (gerfifrumlag)
e) zda-li je sloveso použito v mediopasivním tvaru
f) informace o významu - nejčastější příklady, které se pojí se slovesem
g) s kterými příslovci nebo předložkami se sloveso pojí
h) samostatná definice pro příčestí minulé nebo přítomné
\subsection{Informace o syntaxi uvedené zkratkou (viz. oddíl 2. Slovní druhy)}
Zde máme na mysli informace o pádě (nom, acc, dat, gen), slovesu v mediopasivním tvaru (refl), neosobním slovesu (impers), slovesu počasí (met), čísle (sg, pl) ap.
\subsection{Mrtvý příklad}
Mrtvým příkladem nazýváme příklad, kdy předmět je zastoupen zájmenem neurčitým v pádě, s kterým se slovo pojí a slovesa je zpravidla v infinitivu. V případě, že podmět není v 1. pádě (např. u sloves neosobních), je sloveso uvedeno v 3. osobě jednotného čísla přítomného času.
Ve slovních spojeních používáme zkratky pro tvary islandského slova einhver (někdo) a eitthvað (něco). Tyto zkratky vyjadřují a) pád a b) životnost nebo neživotnost předmětu
Vysvětlení zkratek: 
\textbf{e-ð}  - eitthvað (4. pád, neživotný předmět), \textbf{e-n}  - einhvern (4. pád, životný předmět), \textbf{e-u} -einhverju (3. pád, neživotný předmět), \textbf{e-m} – einhverjum (3. pád, životný předmět),  \textbf{e-s} einhvers (2. pád, neživotný předmět), \textbf{e-rs} einhvers (2. pád, životný předmět)

\entry{
leit|a } {\small{v (-aði) gen}
...
{\small{leita + að}}
\textbf{leita að e-u/ e-m} {hledat (co/ koho)}
	\textit{
leita að lyklunum} \textit{ hledat klíče  }}
Příklad tedy přečteme jako leita að einhverju (hledat co) a leita að einhverjum (hledat koho). V tomto případě se tedy sloveso pojí s 3. pádem a předmět může být životný i neživotný.

\subsection{Ilustrační příklad}
Ve slovníku se vyskytuje velké množství ilustračních příkladů a jejich překladů. Ilustrační příklad se liší od příkladu mrtvého tím, že místo zájmen neurčitých se vyskytuje frekventované slovo v bězném projevu. Ilustrační příklady slouží k ilustraci použití
heslového slova. Snahou bylo uvést příklady, které jsou časté v běžné řeči.
Ilustračním příkladem může být a)(složené) slovo, b) slovní spojení, c) věta 

9.3.1	 Slovo jako ilustrační příklad 

\entry{
hett/a } {\small{f (-u, -ur)}
\textbf{1. } {kapuce}
\textit{hettuúlpa} \textit{ bunda s kapucí  }}

9.3.2	 Slovní spojení jako ilustrační příklad 

\entry{
tím|i  } {\small{m (-a, -ar)}
...
\textbf{2. } \textit{ (klukkustund) }{hodina (šedesát minut)}
\textit{tveggja tíma gangur} \textit{ dvouhodinový pochod  }}

9.3.3	 Věta jako ilustrační příklad 

\entry{
mæt|a   } {\small{v (-ti, -t) dat}
...
\textbf{mætast } \textit{ (ná saman) }
{\small{ refl }}{setkat se}
\textit{Þau mættust á miðri leið.} \textit{  Potkali se na půli cesty. }} 

\section{Fotografie, obrázky a ilustrace}
V online verzi je publikováno značné množství fotografií zejména rostlin a živočichů ale i fotografií, které dokumentují místopisně heslová slova (např. knihovnu, poštu, řeku nebo islandské kulinářské speciality (různé druhy koláčů, dortů, koblih ap.)).

Fotografie z oblasti biologie a botaniky jsou dvojího druhu. Buď a) odkazují hypertextově na www.biolib.cz (Databázi rostlin a živočichů), kde je možné se dozvědět více o daném druhu rostliny nebo živočicha nebo b) po kliknutí na miniaturní fotografii se otevře detailní fotografie ve vyskakovacím okně (kliknutím poza fotografii se okno s fotografií automaticky zavře).
Všechny fotografie, které se vyskytují ve slovníku, jsou publikovány pod veřejnými licencemi a jsou uvedeny s názvem autora a licence.

\par\begin{center}\setlength\fboxsep{0pt}\setlength\fboxrule{0.5pt}\fbox{\includegraphics[width=6cm]{/home/chejnik/Dokumenty/web/HVALUR-JOINED/www/images/uploaded_files/ds_image_posthus_0_2.jpg}}\end{center}
\par\begin{center}\footnotesize {Autor:hvalur.org Licence: CC Unported Licence}\end{center}

\section{Výslovnost slov }
V hranatých závorkách je uveden IPA zápis výslovnosti heslového slova (např. heslové slovo úlfur - [ulv{\textscy}r]). Dvojtečka označuje dlouhou samohlásku nebo souhlásku (např. renna - [r{\textepsilon}n{\textlengthmark}a]). V IPA zápise nejsou zaznačeny hlavní a vedlejší přízvuky.

\clearpage
\makeatletter\@openrightfalse
\chapter{Zkratky}
\@openrighttrue\makeatother 
\begin{description}\itemsep2pt
\item[\small{acc}] \small{4. pád}
\item[\small{adj}] \small{přídavné jméno}
\item[\small{adv}] \small{příslovce}
\item[\small{astro.}] \small{astronomie, výzkum vesmíru}
\item[\small{biol.}] \small{biologie}
\item[\small{bot.}] \small{botanika}
\item[\small{básn.}] \small{básnický, poetický výraz}
\item[\small{chem.}] \small{chemie}
\item[\small{comp}] \small{stupňování, 2. stupeň}
\item[\small{conj}] \small{spojka}
\item[\small{cykl.}] \small{cyklistika, jízdní kola}
\item[\small{dat}] \small{3. pád}
\item[\small{def}] \small{tvar určitý}
\item[\small{dem}] \small{zájmeno ukazovací}
\item[\small{dět.}] \small{dětsky}
\item[\small{ekon.}] \small{ekonomika, obchod}
\item[\small{elek.}] \small{elektřina}
\item[\small{f}] \small{podstatné jméno, rod ženský}
\item[\small{filos.}] \small{filosofie, logika}
\item[\small{form.}] \small{formálně}
\item[\small{fyz.}] \small{fyzika}
\item[\small{gen}] \small{2. pád}
\item[\small{geog.}] \small{zeměpis, geografie}
\item[\small{geol.}] \small{geologie}
\item[\small{han.}] \small{hanlivě, pejorativně}
\item[\small{hist.}] \small{historie}
\item[\small{hovor.}] \small{hovorově}
\item[\small{hrub.}] \small{hrubě}
\item[\small{hud.}] \small{hudba}
\item[\small{impers}] \small{sloveso neosobní}
\item[\small{indef}] \small{zájmeno neurčité}
\item[\small{inter}] \small{citoslovce}
\item[\small{jaz.}] \small{jazykověda}
\item[\small{kulin.}] \small{kulinářství, vaření}
\item[\small{let.}] \small{letectví}
\item[\small{lit.}] \small{literatura, vydavatelství}
\item[\small{m}] \small{podstatné jméno, rod mužský}
\item[\small{mat.}] \small{matematika}
\item[\small{med. }] \small{lékařství}
\item[\small{met}] \small{sloveso počasí}
\item[\small{meteo.}] \small{meteorologie}
\item[\small{n}] \small{podstatné jméno, rod střední}
\item[\small{neform.}] \small{neformálně}
\item[\small{neo.}] \small{neologismus}
\item[\small{nom}] \small{1. pád}
\item[\small{num}] \small{číslovka}
\item[\small{náb.}] \small{náboženství}
\item[\small{nám.}] \small{námořnictví, rybolov}
\item[\small{ord}] \small{řadová číslovka}
\item[\small{part}] \small{částice}
\item[\small{pers}] \small{osoba/ zájmeno osobní}
\item[\small{pol.}] \small{politika, politologie}
\item[\small{poss}] \small{zájmeno přivlastňovací}
\item[\small{pov.}] \small{lidové pověsti, folkloristika}
\item[\small{poč.}] \small{informatika, počítače}
\item[\small{pp}] \small{příčestí minulé}
\item[\small{predp}] \small{předpona}
\item[\small{prep}] \small{předložka}
\item[\small{pron}] \small{zájmeno}
\item[\small{prop}] \small{vlastní jméno}
\item[\small{práv.}] \small{právnictví, soudnictví}
\item[\small{psych.}] \small{psychologie}
\item[\small{přen.}] \small{přeneseně}
\item[\small{přís.}] \small{přísloví}
\item[\small{refl}] \small{mediopasivum}
\item[\small{slang.}] \small{slang}
\item[\small{sport.}] \small{sport}
\item[\small{stav.}] \small{stavebnictví, architektura}
\item[\small{sup}] \small{stupňování, 3. stupeň}
\item[\small{techn.}] \small{technika, mechanika}
\item[\small{v}] \small{sloveso}
\item[\small{voj.}] \small{vojenství}
\item[\small{zast.}] \small{zastarale}
\item[\small{zdrob.}] \small{zdrobněle}
\item[\small{zem.}] \small{zemědělství}
\item[\small{zkr}] \small{zkratka}
\item[\small{zool.}] \small{zoologie}
\item[\small{zool.}] \small{zoologie}
\item[\small{říd.}] \small{řídce}
\item[\small{škol.}] \small{školství}
\end{description}
\clearpage
\makeatletter\@openrightfalse
\chapter{Islandsko-český studijní slovník}
\@openrighttrue\makeatother
\clearpage
\pagestyle{dictstyle}
\dictchar{A}\entry{{a.m.k.}}{{\color{darkgreen}{\small{ zkr}}}{\textsl{\textbf{ að minnsta kosti}}} $\rightarrow$       kostur}
\entry{{a.n.}}{{\color{darkgreen}{\small{ zkr}}}{\textsl{\textbf{ að neðan}}}\foreignlanguage{czech}{{ pod, dole}}}
\entry{{a.n.l.}}{{\color{darkgreen}{\small{ zkr}}}{\textsl{\textbf{ að nokkru leyti}}}\foreignlanguage{czech}{{ do určité míry, částečně}}}
\entry{{a.o.}}{{\color{darkgreen}{\small{ zkr}}}{\textsl{\textbf{ að ofan}}}\foreignlanguage{czech}{{ nahoře, nad}}}
\entry{{ab. fn.}}{{\color{darkgreen}{\small{ zkr}}}{\textsl{\textbf{ afturbeygt fornafn}}} $\rightarrow$       afturbeygður}
\entry{{aborr|i}}{{\textipa{[{a}{\textsubring{b}}{\textopeno}{r}{\textlengthmark}{\textsci}]}}{\color{darkgreen}{\small{ m}}}{\small{ (-a, -ar)}}\foreignlanguage{czech}{{\footnotesize{ zool.}}}\foreignlanguage{czech}{{ okoun, okoun říční}}{\textit{ Perca fluviatilis}}\par\begin{center}\setlength\fboxsep{0pt}\setlength\fboxrule{0.5pt}\fbox{\includegraphics[width=6cm]{13418.jpg}}\end{center}\par\begin{center}\footnotesize {Autor: Kesl Michael Licence: COPYRIGHT/CC-BY-NC}\end{center}}
\entry{{\textsuperscript{1}}{að}}{{\textipa{[{a}{\textlengthmark}{\texttheta}]}}{\color{darkgreen}{\small{ prep/ adv}}}{\small{ dat}}{ 1.}\foreignlanguage{czech}{{ k}}{\footnotesize {\foreignlanguage{czech}{ (o pohybu k nějakému místu)}}} $\triangleright$ {\textit{\textbf{ Hann gekk að steininum.}}}{\textit{\foreignlanguage{czech}{ Šel ke kamenu.}}}{ 2.}{\footnotesize {\foreignlanguage{czech}{ (o změně (z jedné věci na druhou))}}} $\triangleright$ {\textit{\textbf{ Vatnið verður að ís.}}}{\textit{\foreignlanguage{czech}{ Z vody se stává led.}}},{ 3.}{\footnotesize {\foreignlanguage{czech}{ (o pobytu na místě)}}} $\triangleright$ {\textit{\textbf{ sitja að búi sínu}}},{ 4.}\foreignlanguage{czech}{{ o, k}}{\footnotesize {\foreignlanguage{czech}{ (o čase)}}},{\textsl{\textbf{ að jólum}}}\foreignlanguage{czech}{{ o vánocích}},{\textsl{\textbf{ að lokum}}}\foreignlanguage{czech}{{ nakonec}},{ 5.}\foreignlanguage{czech}{{ kvůli, z}}{\footnotesize {\foreignlanguage{czech}{ (o důvodu, motivu)}}} $\triangleright$ {\textit{\textbf{ hvítur að lit}}}{\textit{\foreignlanguage{czech}{ bílý}}},{ 6.}{\footnotesize {\foreignlanguage{czech}{ (ve spojeních s některými slovesy)}}} $\triangleright$ {\textit{\textbf{ leita að e-u}}}{\textit{\foreignlanguage{czech}{ hledat (co)}}},{\textsl{\textbf{ að kvöldi}}}\foreignlanguage{czech}{{ večerem, k večeru}},}
\entry{{\textsuperscript{2}}{að}}{{\textipa{[{a}{\textlengthmark}{\texttheta}]}}{\color{darkgreen}{\small{ conj}}}{ 1.}\foreignlanguage{czech}{{ že}} $\triangleright$ {\textit{\textbf{ Hann sagði að hann kæmi aftur á morgun.}}}{\textit{\foreignlanguage{czech}{ Řekl, že přijde zase zítra.}}}{ 2.}\foreignlanguage{czech}{{ (o účelu) aby}} $\triangleright$ {\textit{\textbf{ Ég var sendur (til) að kaupa mjólk.}}}{\textit{\foreignlanguage{czech}{ Poslali mě, abych koupil mléko.}}},}
\entry{{\textsuperscript{3}}{að}}{{\textipa{[{a}{\textlengthmark}{\texttheta}]}}{\color{darkgreen}{\small{ part}}}\foreignlanguage{czech}{{\footnotesize{ jaz.}}}\foreignlanguage{czech}{{ částice značící infinitiv}}}
\entry{{aðal}}{{\textipa{[{a}{\textlengthmark}{ð}{a}{\textsubring{l}}]}}{\color{darkgreen}{\small{ n}}}{\small{ (-s)}}{\textit{ (einkenni)}}\foreignlanguage{czech}{{ charakteristický/ hlavní rys}}}
\entry{{aðal-}}{{\textipa{[{a}{\textlengthmark}{ð}{a}{l}]}}{\color{darkgreen}{\small{ predp}}}\foreignlanguage{czech}{{ hlavní, centrální}} $\triangleright$ {\textit{\textbf{ aðalmál, aðalpersóna, aðalhöfundur}}}{\textit{\foreignlanguage{czech}{ hlavní záležitost, hlavní postava, hlavní spisovatel}}}}
\entry{{aðal··atriði}}{{\textipa{[{a}{\textlengthmark}{ð}{a}{\textsubring{l}}{a}{\textsubring{d}}{r}{\textsci}{ð}{\textsci}]}}{\color{darkgreen}{\small{ n}}}{\small{ (-s, -)}}\foreignlanguage{czech}{{ hlavní věc, to nejdůležitější}} $\triangleright$ {\textit{\textbf{ Aðalatriðið er að enginn slasaðist.}}}{\textit{\foreignlanguage{czech}{ Nejdůležitější je, že se nikdo nezranil.}}}}
\entry{{aðal··á·hersl|a}}{{\textipa{[{a}{\textlengthmark}{ð}{a}{\textsubring{l}}{au}{h}{\textepsilon}{\textsubring{r}}{s}{\textsubring{d}}{l}{a}]}}{\color{darkgreen}{\small{ f}}}{\small{ (-u, -ur)}}{ 1.}{\textit{ (höfuðáhersla)}}\foreignlanguage{czech}{{ hlavní důraz}}{ 2.}\foreignlanguage{czech}{{\footnotesize{ jaz.}}}\foreignlanguage{czech}{{ hlavní přízvuk}} $\triangleright$ {\textit{\textbf{ Aðaláhersla er á fyrsta atkvæði.}}}{\textit{\foreignlanguage{czech}{ Hlavní přízvuk je na první slabice.}}},}
\entry{{aðal··braut}}{{\textipa{[{a}{ð}{a}{\textsubring{l}}{\textsubring{b}}{r}{\ae i}{\textsubring{d}}]}}{\color{darkgreen}{\small{ f}}}{\small{ (-ar, -ir)}}{ 1.}{\textit{ (helsta leið)}}\foreignlanguage{czech}{{ hlavní ulice, třída}}{ 2.}\foreignlanguage{czech}{{ hlavní silnice (v dopravním ruchu)}} $\triangleright$ {\textit{\textbf{ aðalbrautarréttur}}},}
\entry{{aðal··fund|ur}}{{\textipa{[{a}{ð}{a}{\textsubring{l}}{f}{\textscy}{n}{\textsubring{d}}{\textscy}{r}]}}{\color{darkgreen}{\small{ m}}}{\small{ (-ar, -ir)}}\foreignlanguage{czech}{{ výroční schůze}}}
\entry{{aðal··|gata}}{{\textipa{[{a}{ð}{a}{\textsubring{l}}{\r{g}}{a}{\textsubring{d}}{a}]}}{\color{darkgreen}{\small{ f}}}{\small{ (-götu, -götur)}}\foreignlanguage{czech}{{ hlavní třída, hlavní ulice}}}
\entry{{aðal··hending}}{{\textipa{[{a}{\textlengthmark}{ð}{a}{\textsubring{l}}{}{h}{\textepsilon}{n}{\textsubring{d}}{i}{}{\r{g}}]}}{\color{darkgreen}{\small{ f}}}{\small{ (-ar, -ar)}}\foreignlanguage{czech}{{\footnotesize{ jaz.}}}{\footnotesize {\foreignlanguage{czech}{ rým ve stejném verši, v kterém se rýmují souhláska a po ní následující samohláska}}}}
\entry{{aðal··hlut·verk}}{{\textipa{[{a}{ð}{a}{\textsubring{l}}{\textscy}{\textsubring{d}}{v}{\textepsilon}{\textsubring{r}}{\r{g}}]}}{\color{darkgreen}{\small{ n}}}{\small{ (-s, -)}}\foreignlanguage{czech}{{ hlavní role}} $\triangleright$ {\textit{\textbf{ aðalhlutverk í leikriti}}}{\textit{\foreignlanguage{czech}{ hlavní role v divadelní hře}}}}
\entry{{aðal|l}}{{\textipa{[{a}{\textlengthmark}{ð}{a}{\textsubring{d}}{\textsubring{l}}]}}{\color{darkgreen}{\small{ m}}}{\small{ (-s)}}{ 1.}{\textit{ (yfirstétt í lénsveldi)}}\foreignlanguage{czech}{{ šlechta, aristokracie}}{ 2.}{\textit{ (meginhluti)}}\foreignlanguage{czech}{{ jádro, hlavní část}},}
\entry{{aðal··lega}}{{\textipa{[{a}{ð}{a}{l}{\textepsilon}{\textbabygamma}{a}]}}{\color{darkgreen}{\small{ adv}}}\foreignlanguage{czech}{{ hlavně}}}
\entry{{aðal··|maður}}{{\textipa{[{a}{ð}{a}{\textsubring{l}}{m}{a}{ð}{\textscy}{r}]}}{\color{darkgreen}{\small{ m}}}{\small{ (-manns, -menn)}}\foreignlanguage{czech}{{ hlavní/ ústřední postava}} $\triangleright$ {\textit{\textbf{ aðalmaður í framkvæmdum}}}{\textit{\foreignlanguage{czech}{ ústřední postava v realizaci}}}}
\entry{{aðal··nám·skrá}}{{\textipa{[{a}{ð}{a}{\textsubring{l}}{n}{au}{m}{}{s}{u}{p}{}{}{}{}{s}{u}{p}{}{s}{\r{g}}{r}{au}]}}{\color{darkgreen}{\small{ f}}}{\small{ (-r, -r)}}\foreignlanguage{czech}{{\footnotesize{ škol.}}}\foreignlanguage{czech}{{ hlavní učební plán}}}
\entry{{aðal··persón|a}}{{\textipa{[{a}{ð}{a}{\textsubring{l}}{p\textsuperscript{h}}{\textepsilon}{\textsubring{r}}{s}{ou}{n}{a}]}}{\color{darkgreen}{\small{ f}}}{\small{ (-u, -ur)}}\foreignlanguage{czech}{{\footnotesize{ lit.}}}\foreignlanguage{czech}{{ protagonista, hlavní postava}}}
\entry{{aðal··setning}}{{\textipa{[{a}{ð}{a}{\textsubring{l}}{s}{\textepsilon}{h}{\textsubring{d}}{n}{i}{}{\r{g}}]}}{\color{darkgreen}{\small{ f}}}{\small{ (-ar, -ar)}}\foreignlanguage{czech}{{\footnotesize{ jaz.}}}\foreignlanguage{czech}{{ hlavní věta}}}
\entry{{aðal··skipu·lag}}{{\textipa{[{a}{ð}{a}{\textsubring{l}}{s}{\r{\textObardotlessj}}{\textsci}{\textsubring{b}}{\textscy}{l}{a}{\textbabygamma}]}}{\color{darkgreen}{\small{ n}}}{\small{ (-s)}}\foreignlanguage{czech}{{ ???}}}
\entry{{aðal··|sögn}}{{\textipa{[{a}{ð}{a}{\textsubring{l}}{s}{\ae}{\r{g}}{\textsubring{n}}]}}{\color{darkgreen}{\small{ f}}}{\small{ (-sagnar, -sagnir)}}\foreignlanguage{czech}{{\footnotesize{ jaz.}}}\foreignlanguage{czech}{{ hlavní sloveso}}}
\entry{{aðal··tenging}}{{\textipa{[{a}{ð}{a}{\textsubring{l}}{t\textsuperscript{h}}{ei}{\textltailn}{\r{\textObardotlessj}}{i}{}{\r{g}}]}}{\color{darkgreen}{\small{ f}}}{\small{ (-ar, -ar)}}\foreignlanguage{czech}{{\footnotesize{ jaz.}}}\foreignlanguage{czech}{{ spojka souřadící}}}
\entry{{að··blástur}}{{\textipa{[{a}{ð}{\textsubring{b}}{l}{au}{s}{\textsubring{d}}{\textscy}{r}]}}{\color{darkgreen}{\small{ m}}}{\small{ (-s)}}\foreignlanguage{czech}{{\footnotesize{ jaz.}}}\foreignlanguage{czech}{{ preaspirace, předpřídech}}}
\entry{{að··búð}}{{\textipa{[{a}{ð}{\textsubring{b}}{u}{\texttheta}]}}{\color{darkgreen}{\small{ f}}}{\small{ (-ar)}}{\textit{ (kjör)}}\foreignlanguage{czech}{{ podmínky, okolnosti}}}
\entry{{að··búnað|ur}}{{\textipa{[{a}{ð}{\textsubring{b}}{u}{n}{a}{ð}{\textscy}{r}]}}{\color{darkgreen}{\small{ m}}}{\small{ (-ar)}}{ 1.}{\textit{ (aðhlynning)}}\foreignlanguage{czech}{{ péče, opatrování}}{ 2.}{\textit{ (kjör)}}\foreignlanguage{czech}{{ podmínky}} $\triangleright$ {\textit{\textbf{ njóta góðs aðbúnaðar}}}{\textit{\foreignlanguage{czech}{ těšit se z dobrých podmínek}}},}
\entry{{að··dá|andi}}{{\textipa{[{a}{ð}{\textsubring{d}}{au}{a}{n}{\textsubring{d}}{\textsci}]}}{\color{darkgreen}{\small{ m}}}{\small{ (-anda, -endur)}}\foreignlanguage{czech}{{ obdivovatel, fanoušek}} $\triangleright$ {\textit{\textbf{ vera mikill aðdáandi hans}}}{\textit{\foreignlanguage{czech}{ být jeho velkým fanouškem}}}}
\entry{{að·dáan··legur}}{{\textipa{[{a}{ð}{\textsubring{d}}{au}{a}{n}{l}{\textepsilon}{\textbabygamma}{\textscy}{r}]}}{\color{darkgreen}{\small{ adj}}}\foreignlanguage{czech}{{ obdivuhodný}}}
\entry{{að··dáun}}{{\textipa{[{a}{ð}{\textsubring{d}}{au}{\textscy}{n}]}}{\color{darkgreen}{\small{ f}}}{\small{ (-ar)}}\foreignlanguage{czech}{{ obdiv, údiv, úžas}} $\triangleright$ {\textit{\textbf{ blind aðdáun}}}{\textit{\foreignlanguage{czech}{ slepý obdiv}}}}
\entry{{að·dáunar··verður}}{{\textipa{[{a}{ð}{\textsubring{d}}{au}{\textscy}{n}{a}{r}{v}{\textepsilon}{r}{ð}{\textscy}{r}]}}{\color{darkgreen}{\small{ adj}}}\foreignlanguage{czech}{{ obdivuhodný}}}
\entry{{að··dragand|i}}{{\textipa{[{a}{ð}{\textsubring{d}}{r}{a}{\textbabygamma}{a}{n}{\textsubring{d}}{\textsci}]}}{\color{darkgreen}{\small{ m}}}{\small{ (-a)}}{ 1.}\foreignlanguage{czech}{{ předchozí příčíny}} $\triangleright$ {\textit{\textbf{ Málið á sér sögulegan aðdraganda.}}}{ 2.}{\textit{ (undirbúningur)}}\foreignlanguage{czech}{{ přípravy}},}
\entry{{að·dráttar··afl}}{{\textipa{[{a}{ð}{\textsubring{d}}{r}{au}{h}{\textsubring{d}}{a}{r}{a}{\textsubring{b}}{\textsubring{l}}]}}{\color{darkgreen}{\small{ n}}}{\small{ (-s)}}\foreignlanguage{czech}{{ přitažlivost}}{\textsl{\textbf{ aðdráttarafl jarðar}}}\foreignlanguage{czech}{{\footnotesize{ fyz.}}}\foreignlanguage{czech}{{ gravitační síla}},}
\entry{{að·dráttar··lins|a}}{{\textipa{[{a}{ð}{\textsubring{d}}{r}{au}{h}{\textsubring{d}}{a}{r}{l}{\textsci}{n}{s}{a}]}}{\color{darkgreen}{\small{ f}}}{\small{ (-u, -ur)}}\foreignlanguage{czech}{{\footnotesize{ fyz.}}}\foreignlanguage{czech}{{ teleobjektiv}}}
\entry{{að··drótt|un}}{{\textipa{[{a}{ð}{\textsubring{d}}{r}{ou}{h}{\textsubring{d}}{\textscy}{n}]}}{\color{darkgreen}{\small{ f}}}{\small{ (-unar, -anir)}}\foreignlanguage{czech}{{ narážka, naznačení}} $\triangleright$ {\textit{\textbf{ aðdróttanir um morð}}}}
\entry{{aðeins}}{{\textipa{[{a}{\textlengthmark}{ð}{ei}{n}{s}]}}{\color{darkgreen}{\small{ adv}}}{\textit{ (bara)}}\foreignlanguage{czech}{{ pouze, jen(om)}} $\triangleright$ {\textit{\textbf{ aðeins í dag}}}{\textit{\foreignlanguage{czech}{ jenom dneska}}}{\textsl{\textbf{ því aðeins}}}\foreignlanguage{czech}{{ pouze pokud}},}
\entry{{að··|fall}}{{\textipa{[{a}{ð}{f}{a}{\textsubring{d}}{\textsubring{l}}]}}{\color{darkgreen}{\small{ n}}}{\small{ (-falls, -föll)}}{\textit{ (flóð)}}\foreignlanguage{czech}{{ příliv}}{\textit{ (útfall)}}}
\entry{{að·fanga·dags··kvöld}}{{\textipa{[{a}{ð}{f}{au}{}{\r{g}}{a}{\textsubring{d}}{a}{x}{s}{k\textsuperscript{h}}{v}{\ae}{l}{\textsubring{d}}]}}{\color{darkgreen}{\small{ n}}}{\small{ (-s, -)}}\foreignlanguage{czech}{{ večer Štědrého dne}}}
\entry{{að·fanga··dag|ur}}{{\textipa{[{a}{ð}{f}{au}{}{\r{g}}{a}{\textsubring{d}}{a}{\textbabygamma}{\textscy}{r}]}}{\color{darkgreen}{\small{ m}}}{\small{ (-s, -ar)}}\foreignlanguage{czech}{{ Štědrý den}} $\triangleright$ {\textit{\textbf{ Aðfangadagurinn er 24. desember.}}}{\textit{\foreignlanguage{czech}{ Štědrý den je 24. prosince.}}}}
\entry{{að·fara··|nótt}}{{\textipa{[{a}{ð}{f}{a}{r}{a}{n}{ou}{h}{\textsubring{d}}]}}{\color{darkgreen}{\small{ f}}}{\small{ (-nætur, -nætur)}}\foreignlanguage{czech}{{ předešlá noc}} $\triangleright$ {\textit{\textbf{ aðfaranótt mánudags}}}}
\entry{{að··ferð}}{{\textipa{[{a}{ð}{f}{\textepsilon}{r}{ð}]}}{\color{darkgreen}{\small{ f}}}{\small{ (-ar, -ir)}}{\textit{ (verklag)}}\foreignlanguage{czech}{{ postup, metoda}} $\triangleright$ {\textit{\textbf{ mín aðferð}}}}
\entry{{að·ferða··fræði}}{{\textipa{[{a}{ð}{f}{\textepsilon}{r}{ð}{a}{f}{r}{a}{i}{ð}{\textsci}]}}{\color{darkgreen}{\small{ f}}}{\small{ indecl}}\foreignlanguage{czech}{{ metodologie}}}
\entry{{að··finnsl|a}}{{\textipa{[{a}{ð}{f}{\textsci}{n}{s}{\textsubring{d}}{l}{a}]}}{\color{darkgreen}{\small{ f}}}{\small{ (-u, -ur)}}{\textit{ (umvöndun)}}\foreignlanguage{czech}{{ kritika, výtka}}}
\entry{{að··flug}}{{\textipa{[{a}{ð}{f}{l}{\textscy}{\textbabygamma}]}}{\color{darkgreen}{\small{ n}}}{\small{ (-s, -)}}\foreignlanguage{czech}{{\footnotesize{ let.}}}\foreignlanguage{czech}{{ přistávací manévry}}}
\entry{{að·flutnings··|gjald}}{{\textipa{[{a}{ð}{f}{l}{\textscy}{h}{\textsubring{d}}{n}{i}{}{s}{\r{\textObardotlessj}}{a}{l}{\textsubring{d}}]}}{\color{darkgreen}{\small{ n}}}{\small{ (-gjalds, -gjöld)}}\foreignlanguage{czech}{{ dovozní clo}}}
\entry{{að··flutning|ur}}{{\textipa{[{a}{ð}{f}{l}{\textscy}{h}{\textsubring{d}}{n}{i}{}{\r{g}}{\textscy}{r}]}}{\color{darkgreen}{\small{ m}}}{\small{ (-s, -ar)}}\foreignlanguage{czech}{{ import, dovoz}}}
\entry{{að··fluttur}}{{\textipa{[{a}{ð}{f}{l}{\textscy}{h}{\textsubring{d}}{\textscy}{r}]}}{\color{darkgreen}{\small{ adj}}}{\textit{ (innfluttur)}}\foreignlanguage{czech}{{ dovozový, importovaný}}}
\entry{{að·fram··kominn}}{{\textipa{[{a}{ð}{f}{r}{a}{m}{k\textsuperscript{h}}{\textopeno}{m}{\textsci}{n}]}}{\color{darkgreen}{\small{ adj}}}\foreignlanguage{czech}{{ polomrtvý, více mrtvý než živý}}}
\entry{{að··föng}}{{\textipa{[{a}{ð}{f}{\ae i}{}{\r{g}}]}}{\color{darkgreen}{\small{ n}}}{\small{ pl}}{ 1.}\foreignlanguage{czech}{{ ???}} $\triangleright$ {\textit{\textbf{ aðfangabók}}}{ 2.}\foreignlanguage{czech}{{\footnotesize{ ekon.}}}\foreignlanguage{czech}{{ potřeby, pomocný materiál}},}
\entry{{að··|för}}{{\textipa{[{a}{ð}{f}{\ae}{r}]}}{\color{darkgreen}{\small{ f}}}{\small{ (-farar, -farir)}}{ 1.}{\textit{ (árás)}}\foreignlanguage{czech}{{ útok, napadení}}{\textsl{\textbf{ aðför að e-m}}}\foreignlanguage{czech}{{ útok na (koho)}},{ 2.}{\textit{ (atferli)}}\foreignlanguage{czech}{{ (za)chování}} $\triangleright$ {\textit{\textbf{ ljótar aðfarir}}}{\textit{\foreignlanguage{czech}{ nepěkné zachování}}},}
\entry{{að·gangs··eyri|r}}{{\textipa{[{a}{ð}{\r{g}}{au}{}{s}{ei}{r}{\textsci}{r}]}}{\color{darkgreen}{\small{ m}}}{\small{ (-s)}}\foreignlanguage{czech}{{ vstupné, vstupní poplatek}}}
\entry{{að·gangs··harður}}{{\textipa{[{a}{ð}{\r{g}}{au}{}{s}{h}{a}{r}{ð}{\textscy}{r}]}}{\color{darkgreen}{\small{ adj}}}{\textit{ (ýtinn)}}\foreignlanguage{czech}{{ ctižádostivý, ambiciózní}}}
\entry{{að·gangs··kort}}{{\textipa{[{a}{ð}{\r{g}}{au}{}{s}{k\textsuperscript{h}}{\textopeno}{\textsubring{r}}{\textsubring{d}}]}}{\color{darkgreen}{\small{ n}}}{\small{ (-s, -)}}\foreignlanguage{czech}{{ vstupenka}}}
\entry{{að··gang|ur}}{{\textipa{[{a}{ð}{\r{g}}{au}{}{\r{g}}{\textscy}{r}]}}{\color{darkgreen}{\small{ m}}}{\small{ (-s, -ar)}}\foreignlanguage{czech}{{ vstup, přístup}} $\triangleright$ {\textit{\textbf{ fá frjálsan aðgang að gögnunum}}}{\textit{\foreignlanguage{czech}{ mít volný přístup k materiálům}}}}
\entry{{að··gát}}{{\textipa{[{a}{ð}{\r{g}}{au}{\textsubring{d}}]}}{\color{darkgreen}{\small{ f}}}{\small{ (-ar)}}{\textit{ (aðgæsla)}}\foreignlanguage{czech}{{ opatrnost, obezřetnost}} $\triangleright$ {\textit{\textbf{ sýna fulla aðgát}}}}
\entry{{að··gengi}}{{\textipa{[{a}{ð}{\r{\textObardotlessj}}{ei}{\textltailn}{\r{\textObardotlessj}}{\textsci}]}}{\color{darkgreen}{\small{ n}}}{\small{ (-s)}}{\textit{ (aðgangur)}}\foreignlanguage{czech}{{ vstup, přístup}}}
\entry{{að·gengi··legur}}{{\textipa{[{a}{ð}{\r{\textObardotlessj}}{ei}{\textltailn}{\r{\textObardotlessj}}{\textsci}{l}{\textepsilon}{\textbabygamma}{\textscy}{r}]}}{\color{darkgreen}{\small{ adj}}}{ 1.}\foreignlanguage{czech}{{ přístupný (vchod ap.)}} $\triangleright$ {\textit{\textbf{ aðgengilegur staður}}}{ 3.}{\textit{ (sanngjarn)}}\foreignlanguage{czech}{{ přijatelný, přípustný, akceptovatelný}} $\triangleright$ {\textit{\textbf{ aðgengilegt tilboð}}},{ 2.}{\textit{ (auðnotaður)}}\foreignlanguage{czech}{{ přístupný (myšlenka ap.), snadno pochopitelný}} $\triangleright$ {\textit{\textbf{ aðgengilegir skilmálar}}},}
\entry{{að··gerð}}{{\textipa{[{a}{ð}{\r{\textObardotlessj}}{\textepsilon}{r}{ð}]}}{\color{darkgreen}{\small{ f}}}{\small{ (-ar, -ir)}}{ 1.}{\textit{ (ráðstafanir)}}\foreignlanguage{czech}{{ opatření, krok}}{ 2.}{\textit{ (viðgerð)}}\foreignlanguage{czech}{{ oprava}},{ 3.}\foreignlanguage{czech}{{ operace, chirurgický zákrok}} $\triangleright$ {\textit{\textbf{ Nonni litli fór í mikla aðgerð vegna hjartagalla.}}},{ 4.}{\textit{ (framkvæmd)}}\foreignlanguage{czech}{{ operace, akce, zásah}},{ 5.}\foreignlanguage{czech}{{ zpracování ryb}} $\triangleright$ {\textit{\textbf{ Margir unnu í aðgerð í frystihúsinu.}}}{\textit{\foreignlanguage{czech}{ Hodně lidí pracovalo při zpracování ryb v mrazírně.}}},}
\entry{{að·gerða··laus}}{{\textipa{[{a}{ð}{\r{\textObardotlessj}}{\textepsilon}{r}{ð}{a}{l}{\ae i}{s}]}}{\color{darkgreen}{\small{ adj}}}{\textit{ (afskiptahægur)}}\foreignlanguage{czech}{{ pasivní, nečinný}} $\triangleright$ {\textit{\textbf{ standa þarna aðgerðalaus}}}}
\entry{{að·gerða··leysi}}{{\textipa{[{a}{ð}{\r{\textObardotlessj}}{\textepsilon}{r}{ð}{a}{l}{ei}{s}{\textsci}]}}{\color{darkgreen}{\small{ n}}}{\small{ (-s)}}\foreignlanguage{czech}{{ nečinnost}}}
\entry{{að·gerða··lítill}}{{\textipa{[{a}{ð}{\r{\textObardotlessj}}{\textepsilon}{r}{ð}{a}{l}{i}{\textsubring{d}}{\textsci}{\textsubring{d}}{\textsubring{l}}]}}{\color{darkgreen}{\small{ adj}}}\foreignlanguage{czech}{{ nečinný, záhalčivý}}}
\entry{{að··greining}}{{\textipa{[{a}{ð}{\r{g}}{r}{ei}{n}{i}{}{\r{g}}]}}{\color{darkgreen}{\small{ f}}}{\small{ (-ar, -ar)}}\foreignlanguage{czech}{{ odlišení, rozlišení}}}
\entry{{að··gæsl|a}}{{\textipa{[{a}{ð}{\r{\textObardotlessj}}{a}{i}{s}{\textsubring{d}}{l}{a}]}}{\color{darkgreen}{\small{ f}}}{\small{ (-u)}}{\textit{ (eftirlit)}}\foreignlanguage{czech}{{ opatrnost, obezřetnost}} $\triangleright$ {\textit{\textbf{ sýna miklu aðgæslu}}}{\textit{ (óaðgæsla)}}}
\entry{{að··gæt|a}}{{\textipa{[{a}{ð}{\r{\textObardotlessj}}{a}{i}{\textsubring{d}}{a}]}}{\color{darkgreen}{\small{ v}}}{\small{ (-ti, -t)}}{\small{ acc}}{\textit{ (gefa gaum að)}}\foreignlanguage{czech}{{ prozkoumat, ověřit, zjistit}} $\triangleright$ {\textit{\textbf{ aðgæta hvort gluggarnir eru lokaðir}}}{\textit{\foreignlanguage{czech}{ zjistit zda-li jsou zavřená okna}}}}
\entry{{að··gætinn}}{{\textipa{[{a}{ð}{\r{\textObardotlessj}}{a}{i}{\textsubring{d}}{\textsci}{n}]}}{\color{darkgreen}{\small{ adj}}}\foreignlanguage{czech}{{ opatrný, obezřetný}}}
\entry{{að··gætni}}{{\textipa{[{a}{ð}{\r{\textObardotlessj}}{a}{i}{h}{\textsubring{d}}{n}{\textsci}]}}{\color{darkgreen}{\small{ f}}}{\small{ indecl}}{\textit{ (aðgát)}}\foreignlanguage{czech}{{ opatrnost}}}
\entry{{að·göngu··mið|i}}{{\textipa{[{a}{ð}{\r{g}}{\ae i}{}{\r{g}}{\textscy}{m}{\textsci}{ð}{\textsci}]}}{\color{darkgreen}{\small{ m}}}{\small{ (-a, -ar)}}\foreignlanguage{czech}{{ vstupenka}} $\triangleright$ {\textit{\textbf{ aðgöngumiði að sýningunni}}}{\textit{\foreignlanguage{czech}{ vstupenka na výstavu}}}}
\entry{{að··haf|ast}}{{\textipa{[{a}{\textlengthmark}{\texttheta}{a}{v}{a}{s}{\textsubring{d}}]}}{\color{darkgreen}{\small{ v}}}{\small{ (-ðist, -st)}}{\small{ refl}}\foreignlanguage{czech}{{ udělat, dělat}} $\triangleright$ {\textit{\textbf{ aðhafast ekkert á kvöldin}}}{\textit{\foreignlanguage{czech}{ nic večír nedělat}}}}
\entry{{að··hald}}{{\textipa{[{a}{\textlengthmark}{\texttheta}{a}{l}{\textsubring{d}}]}}{\color{darkgreen}{\small{ n}}}{\small{ (-s)}}{ 1.}{\textit{ (stuðningur)}}\foreignlanguage{czech}{{ podpora, pomoc}}{ 2.}{\textit{ (það að halda aftur af sér)}}\foreignlanguage{czech}{{ omezení (jídla ap.), držení se zpět}} $\triangleright$ {\textit{\textbf{ Ég er í aðhaldi.}}}{\textit{\foreignlanguage{czech}{ Omezuju se (v jídle ap.).}}},{ 3.}{\textit{ (eftirlit)}}\foreignlanguage{czech}{{ dohled, dozor}},{ 4.}{\textit{ (girðing)}}\foreignlanguage{czech}{{ oplocení}},}
\entry{{að··hlát|ur}}{{\textipa{[{a}{ð}{\textsubring{l}}{au}{\textsubring{d}}{\textscy}{r}]}}{\color{darkgreen}{\small{ m}}}{\small{ (-urs/-rar, -rar)}}\foreignlanguage{czech}{{ výsměch, posměch}} $\triangleright$ {\textit{\textbf{ verða að aðhlátri}}}}
\entry{{að·hláturs··efni}}{{\textipa{[{a}{ð}{\textsubring{l}}{au}{\textsubring{d}}{\textscy}{\textsubring{r}}{s}{\textepsilon}{\textsubring{b}}{n}{\textsci}]}}{\color{darkgreen}{\small{ n}}}{\small{ (-s)}}\foreignlanguage{czech}{{ předmět výsměchu}}}
\entry{{að··hlynning}}{{\textipa{[{a}{ð}{\textsubring{l}}{\textsci}{n}{i}{}{\r{g}}]}}{\color{darkgreen}{\small{ f}}}{\small{ (-ar)}}\foreignlanguage{czech}{{ péče, ošetřování}} $\triangleright$ {\textit{\textbf{ veita sjúklingnum aðhlynningu}}}}
\entry{{að··hyll|ast}}{{\textipa{[{a}{\textlengthmark}{\texttheta}{\textsci}{\textsubring{d}}{l}{a}{s}{\textsubring{d}}]}}{\color{darkgreen}{\small{ v}}}{\small{ (-tist, -st)}}{\small{ refl acc}}{ 1.}\foreignlanguage{czech}{{ schvalovat, schválit}} $\triangleright$ {\textit{\textbf{ aðhyllast stefnu flokksins}}}{\textit{\foreignlanguage{czech}{ schvalovat směr strany}}}{ 2.}\foreignlanguage{czech}{{ vyznávat (víru)}} $\triangleright$ {\textit{\textbf{ aðhyllast lúterstrú}}}{\textit{\foreignlanguage{czech}{ vyznávat luteránskou víru}}},}
\entry{{að··hæf|a}}{{\textipa{[{a}{\textlengthmark}{\texttheta}{a}{i}{v}{a}]}}{\color{darkgreen}{\small{ v}}}{\small{ (-ði, -t)}}{\small{ acc}}{\textit{ (aðlaga)}}\foreignlanguage{czech}{{ přizpůsobit, adaptovat}}{\textsl{\textbf{ aðhæfa e-ð að e-u}}}\foreignlanguage{czech}{{ přizpůsobit (co čemu)}},}
\entry{{aðild}}{{\textipa{[{a}{\textlengthmark}{ð}{\textsci}{l}{\textsubring{d}}]}}{\color{darkgreen}{\small{ f}}}{\small{ (-ar)}}\foreignlanguage{czech}{{ účast, členství, zapojení}} $\triangleright$ {\textit{\textbf{ aðild að félaginu}}}}
\entry{{aðildar··fé·|lag}}{{\textipa{[{a}{ð}{\textsci}{l}{\textsubring{d}}{a}{\textsubring{r}}{f}{j}{\textepsilon}{l}{a}{\textbabygamma}]}}{\color{darkgreen}{\small{ n}}}{\small{ (-lags, -lög)}}\foreignlanguage{czech}{{ ???}}}
\entry{{aðildar··ríki}}{{\textipa{[{a}{ð}{\textsci}{l}{\textsubring{d}}{a}{r}{i}{\r{\textObardotlessj}}{\textsci}]}}{\color{darkgreen}{\small{ n}}}{\small{ (-s, -)}}\foreignlanguage{czech}{{ členská země}}}
\entry{{aðil|i}}{{\textipa{[{a}{\textlengthmark}{ð}{\textsci}{l}{\textsci}]}}{\color{darkgreen}{\small{ m}}}{\small{ (-a/-ja, -ar/-jar)}}\foreignlanguage{czech}{{ osoba, jednací strana}}}
\entry{{að··kallandi}}{{\textipa{[{a}{ð}{k\textsuperscript{h}}{a}{\textsubring{d}}{l}{a}{n}{\textsubring{d}}{\textsci}]}}{\color{darkgreen}{\small{ adj}}}{\small{ indecl}}\foreignlanguage{czech}{{ naléhavý, urgentní}} $\triangleright$ {\textit{\textbf{ aðkallandi verkefni}}}}
\entry{{að··|kast}}{{\textipa{[{a}{ð}{k\textsuperscript{h}}{a}{s}{\textsubring{d}}]}}{\color{darkgreen}{\small{ n}}}{\small{ (-kasts, -köst)}}\foreignlanguage{czech}{{ jízlivé poznámky, posměšky}}{\textsl{\textbf{ verða fyrir aðkasti}}}\foreignlanguage{czech}{{ být zesměšňován}},}
\entry{{að··kenning}}{{\textipa{[{a}{ð}{c\textsuperscript{h}}{\textepsilon}{n}{i}{}{\r{g}}]}}{\color{darkgreen}{\small{ f}}}{\small{ (-ar)}}{\textit{ (snertur)}}\foreignlanguage{czech}{{ náběh, náznak}} $\triangleright$ {\textit{\textbf{ fá aðkenningu af slagi}}}{\textit{\foreignlanguage{czech}{ dostat náběh na mrtvici}}}}
\entry{{að··keyptur}}{{\textipa{[{a}{ð}{c\textsuperscript{h}}{ei}{f}{\textsubring{d}}{\textscy}{r}]}}{\color{darkgreen}{\small{ adj}}}\foreignlanguage{czech}{{ koupený jinde (zboží ap.)}} $\triangleright$ {\textit{\textbf{ aðkeyptur varningur}}}}
\entry{{að··kom|a}}{{\textipa{[{a}{ð}{k\textsuperscript{h}}{\textopeno}{m}{a}]}}{\color{darkgreen}{\small{ f}}}{\small{ (-u)}}{ 1.}{\textit{ (nálgun)}}\foreignlanguage{czech}{{ přiblížení, přístup}} $\triangleright$ {\textit{\textbf{ 
}}}{ 2.}{\textit{ (viðtaka)}}\foreignlanguage{czech}{{ uvítání, přivítání}},{ 3.}{\textsl{\textbf{ ljót aðkoma}}}\foreignlanguage{czech}{{ příšerný pohled}},}
\entry{{að·komu··|maður}}{{\textipa{[{a}{ð}{k\textsuperscript{h}}{\textopeno}{m}{\textscy}{m}{a}{ð}{\textscy}{r}]}}{\color{darkgreen}{\small{ m}}}{\small{ (-manns, -menn)}}\foreignlanguage{czech}{{ cizinec, přespolní}}}
\entry{{að··krepptur}}{{\textipa{[{a}{ð}{k\textsuperscript{h}}{r}{\textepsilon}{f}{\textsubring{d}}{\textscy}{r}]}}{\color{darkgreen}{\small{ adj}}}{ 1.}{\textit{ (inniluktur)}}\foreignlanguage{czech}{{ uzavřený}} $\triangleright$ {\textit{\textbf{ aðkrepptur dalur}}}{\textit{\foreignlanguage{czech}{ uzavřené údolí}}}{ 2.}{\textit{ (þröngur)}}\foreignlanguage{czech}{{ stísněný, úzký}},{ 3.}\foreignlanguage{czech}{{ sklíčený, skleslý}},}
\entry{{að··laðandi}}{{\textipa{[{a}{ð}{l}{a}{ð}{a}{n}{\textsubring{d}}{\textsci}]}}{\color{darkgreen}{\small{ adj}}}{\small{ indecl}}\foreignlanguage{czech}{{ přitažlivý, atraktivní, vábivý, čarovný}} $\triangleright$ {\textit{\textbf{ aðlaðandi stúlka}}}{\textit{\foreignlanguage{czech}{ přitažlivá dívka}}}}
\entry{{að··lag|a}}{{\textipa{[{a}{ð}{l}{a}{\textbabygamma}{a}]}}{\color{darkgreen}{\small{ v}}}{\small{ (-aði)}}{\small{ acc}}\foreignlanguage{czech}{{ přizpůsobit, adaptovat}} $\triangleright$ {\textit{\textbf{ aðlaga framleiðsluna nýjum kröfum}}}{\textit{\foreignlanguage{czech}{ přizpůsobit výrobu novým požadavkům}}}{\textsl{\textbf{ aðlaga sig e-u}}}\foreignlanguage{czech}{{ přizpůsobit se (čemu)}} $\triangleright$ {\textit{\textbf{ aðlaga sig aðstæðum}}}{\textit{\foreignlanguage{czech}{ přizpůsobit se situaci}}},{\textsl{\textbf{ aðlagast}}}{\footnotesize{ refl}}\foreignlanguage{czech}{{ přizpůsobit se}} $\triangleright$ {\textit{\textbf{ aðlagast hópnum}}}{\textit{\foreignlanguage{czech}{ přizpůsobit se skupině}}},}
\entry{{að··leiðsl|a}}{{\textipa{[{a}{ð}{l}{ei}{ð}{s}{\textsubring{d}}{l}{a}]}}{\color{darkgreen}{\small{ f}}}{\small{ (-u)}}\foreignlanguage{czech}{{\footnotesize{ filos.}}}\foreignlanguage{czech}{{ indukce}}}
\entry{{að··|lögun}}{{\textipa{[{a}{ð}{l}{\ae}{\textbabygamma}{\textscy}{n}]}}{\color{darkgreen}{\small{ f}}}{\small{ (-lögunar, -laganir)}}\foreignlanguage{czech}{{ přizpůsobení, adaptace}} $\triangleright$ {\textit{\textbf{ aðlögun að nýjum siðum}}}{\textit{\foreignlanguage{czech}{ přizpůsobení se novým zvykům}}}{\textsl{\textbf{ aðlögun að nýju loftalagi}}}\foreignlanguage{czech}{{ aklimatizace (na jiné podnebí ap.)}},}
\entry{{að··njótandi}}{{\textipa{[{a}{ð}{n}{j}{ou}{\textsubring{d}}{a}{n}{\textsubring{d}}{\textsci}]}}{\color{darkgreen}{\small{ adj}}}{\small{ indecl}}{\textsl{\textbf{ verða e-s aðnjótandi}}}\foreignlanguage{czech}{{ těšit se z (čeho), účastnící se (čeho)}} $\triangleright$ {\textit{\textbf{ verða þeirrar gleði aðnjótandi}}}}
\entry{{að··rennsli}}{{\textipa{[{a}{ð}{r}{\textepsilon}{n}{s}{\textsubring{d}}{l}{\textsci}]}}{\color{darkgreen}{\small{ n}}}{\small{ (-s, -)}}\foreignlanguage{czech}{{ přítok}} $\triangleright$ {\textit{\textbf{ aðrennsli vatnsins}}}{\textit{\foreignlanguage{czech}{ přítok jezera}}}}
\entry{{að··setur}}{{\textipa{[{a}{ð}{s}{\textepsilon}{\textsubring{d}}{\textscy}{r}]}}{\color{darkgreen}{\small{ n}}}{\small{ (-s, -)}}\foreignlanguage{czech}{{ sídlo, rezidence}} $\triangleright$ {\textit{\textbf{ Stofnunin hefur aðsetur í borginni.}}}{\textit{\foreignlanguage{czech}{ Instituce sídlí ve městě.}}}}
\entry{{að··sig}}{{\textipa{[{a}{ð}{s}{\textsci}{\textbabygamma}]}}{\color{darkgreen}{\small{ n}}}{\small{ (-s)}}{\textsl{\textbf{ vera í aðsigi}}}\foreignlanguage{czech}{{ blížit se rychle}}}
\entry{{að··skil|ja}}{{\textipa{[{a}{ð}{s}{\r{\textObardotlessj}}{\textsci}{l}{j}{a}]}}{\color{darkgreen}{\small{ v}}}{\small{ (acc) (-di, -ið)}}\foreignlanguage{czech}{{ rozdělit, separovat}} $\triangleright$ {\textit{\textbf{ Systkinin voru aðskilin á unga aldri og sjást nú loks eftir tuttugu ár.}}}}
\entry{{að·skiljan··legur}}{{\textipa{[{a}{ð}{s}{\r{\textObardotlessj}}{\textsci}{l}{j}{a}{n}{l}{\textepsilon}{\textbabygamma}{\textscy}{r}]}}{\color{darkgreen}{\small{ adj}}}{ 1.}\foreignlanguage{czech}{{ oddělitelný}}{ 2.}{\textit{ (mismunandi)}}\foreignlanguage{czech}{{ různorodý, různý}},}
\entry{{að·skilnaðar··stefn|a}}{{\textipa{[{a}{ð}{s}{\r{\textObardotlessj}}{\textsci}{l}{n}{a}{ð}{a}{\textsubring{r}}{s}{\textsubring{d}}{\textepsilon}{\textsubring{b}}{n}{a}]}}{\color{darkgreen}{\small{ f}}}{\small{ (-u)}}\foreignlanguage{czech}{{ apartheid, segregace}}}
\entry{{að··skilnað|ur}}{{\textipa{[{a}{ð}{s}{\r{\textObardotlessj}}{\textsci}{l}{n}{a}{ð}{\textscy}{r}]}}{\color{darkgreen}{\small{ m}}}{\small{ (-ar)}}\foreignlanguage{czech}{{ odloučení, separace}} $\triangleright$ {\textit{\textbf{ aðskilnaður frá ástvinum}}}}
\entry{{að··skorinn}}{{\textipa{[{a}{ð}{s}{\r{g}}{\textopeno}{r}{\textsci}{n}]}}{\color{darkgreen}{\small{ adj}}}\foreignlanguage{czech}{{ přiléhavý (oblek ap.)}} $\triangleright$ {\textit{\textbf{ aðskorinn kjóll}}}}
\entry{{aðskota··dýr}}{{\textipa{[{a}{ð}{s}{\r{g}}{\textopeno}{\textsubring{d}}{a}{\textsubring{d}}{i}{r}]}}{\color{darkgreen}{\small{ n}}}{\small{ (-s, -)}}\foreignlanguage{czech}{{ vetřelec, nevítaný host}}}
\entry{{aðskota··hlut|ur}}{{\textipa{[{a}{ð}{s}{\r{g}}{\textopeno}{\textsubring{d}}{a}{\textsubring{l}}{\textscy}{\textsubring{d}}{\textscy}{r}]}}{\color{darkgreen}{\small{ m}}}{\small{ (-ar, -ir)}}\foreignlanguage{czech}{{ cizí předmět (smítko v oku ap.)}} $\triangleright$ {\textit{\textbf{ aðskotahlutur í auga}}}}
\entry{{að··sog}}{{\textipa{[{a}{ð}{s}{\textopeno}{\textbabygamma}]}}{\color{darkgreen}{\small{ n}}}{\small{ (-s)}}\foreignlanguage{czech}{{ ???}}}
\entry{{að··sókn}}{{\textipa{[{a}{ð}{s}{ou}{h}{\r{g}}{\textsubring{n}}]}}{\color{darkgreen}{\small{ f}}}{\small{ (-ar)}}\foreignlanguage{czech}{{ návštěvnost, počet návštěvníků, účast}} $\triangleright$ {\textit{\textbf{ Það er mikil aðsókn að sýningunni.}}}}
\entry{{aðsóps··mikill}}{{\textipa{[{a}{ð}{s}{ou}{\textsubring{b}}{s}{m}{\textsci}{\r{\textObardotlessj}}{\textsci}{\textsubring{d}}{\textsubring{l}}]}}{\color{darkgreen}{\small{ adj}}}{\textit{ (skörulegur)}}\foreignlanguage{czech}{{ působivý, impozantní}}}
\entry{{að··spurður}}{{\textipa{[{a}{ð}{s}{\textsubring{b}}{\textscy}{r}{ð}{\textscy}{r}]}}{\color{darkgreen}{\small{ adj}}}\foreignlanguage{czech}{{ tázaný, dotazovaný}} $\triangleright$ {\textit{\textbf{ aðspurður svaraði hann ...}}}}
\entry{{að··staða}}{{\textipa{[{a}{ð}{s}{\textsubring{d}}{a}{ð}{a}]}}{\color{darkgreen}{\small{ f}}}{\small{ (aðstöðu)}}\foreignlanguage{czech}{{ podmínky, situace}} $\triangleright$ {\textit{\textbf{ góð aðstaða}}}{\textit{\foreignlanguage{czech}{ dobrá situace}}}}
\entry{{að··stand|andi}}{{\textipa{[{a}{ð}{s}{\textsubring{d}}{a}{n}{\textsubring{d}}{a}{n}{\textsubring{d}}{\textsci}]}}{\color{darkgreen}{\small{ m}}}{\small{ (-anda, -endur)}}{ 1.}\foreignlanguage{czech}{{ mužský příbuzný}}{ 2.}\foreignlanguage{czech}{{ podporovatel, přívrženec, mecenáš}} $\triangleright$ {\textit{\textbf{ aðstandendur sýningarinnar}}},}
\entry{{að··stoð}}{{\textipa{[{a}{ð}{s}{\textsubring{d}}{\textopeno}{\texttheta}]}}{\color{darkgreen}{\small{ f}}}{\small{ (-ar)}}{\textit{ (hjálp)}}\foreignlanguage{czech}{{ (malá) pomoc}} $\triangleright$ {\textit{\textbf{ Aðstoð lögreglunnar var mikils virði.}}}{\textit{\foreignlanguage{czech}{ Pomoc policie měla velkou hodnotu.}}}}
\entry{{að··stoð|a}}{{\textipa{[{a}{ð}{s}{\textsubring{d}}{\textopeno}{ð}{a}]}}{\color{darkgreen}{\small{ v}}}{\small{ (-aði)}}{\small{ acc}}{\textit{ (hjálpa)}}\foreignlanguage{czech}{{ pomoct, pomáhat, asistovat}} $\triangleright$ {\textit{\textbf{ Get ég aðstoðað þig?}}}{\textit{\foreignlanguage{czech}{ Čím Vám mohu posloužit? Můžu Ti pomoct?}}}}
\entry{{að·stoðar··for·stjór|i}}{{\textipa{[{a}{ð}{s}{\textsubring{d}}{\textopeno}{ð}{a}{\textsubring{r}}{f}{\textopeno}{\textsubring{r}}{s}{\textsubring{d}}{j}{ou}{r}{\textsci}]}}{\color{darkgreen}{\small{ m}}}{\small{ (-a, -ar)}}\foreignlanguage{czech}{{ zástupce ředitele}}}
\entry{{að·stoðar··|maður}}{{\textipa{[{a}{ð}{s}{\textsubring{d}}{\textopeno}{ð}{a}{r}{m}{a}{ð}{\textscy}{r}]}}{\color{darkgreen}{\small{ m}}}{\small{ (-manns, -menn)}}\foreignlanguage{czech}{{ asistent, pomocník}}}
\entry{{að··streymi}}{{\textipa{[{a}{ð}{s}{\textsubring{d}}{r}{ei}{m}{\textsci}]}}{\color{darkgreen}{\small{ n}}}{\small{ (-s)}}\foreignlanguage{czech}{{ přítok, tok (vody, lidí ap.)}}}
\entry{{að··stæður}}{{\textipa{[{a}{ð}{s}{\textsubring{d}}{a}{i}{ð}{\textscy}{r}]}}{\color{darkgreen}{\small{ f}}}{\small{ pl}}{\textit{ (kringumstæður)}}\foreignlanguage{czech}{{ podmínky, okolnosti}}}
\entry{{að·stöðu··|gjald}}{{\textipa{[{a}{ð}{s}{\textsubring{d}}{\ae}{ð}{\textscy}{\r{\textObardotlessj}}{a}{l}{\textsubring{d}}]}}{\color{darkgreen}{\small{ n}}}{\small{ (-gjalds, -gjöld)}}\foreignlanguage{czech}{{\footnotesize{ ekon.}}}\foreignlanguage{czech}{{ daň z nákladů}}}
\entry{{að··súg|ur}}{{\textipa{[{a}{ð}{s}{u}{\textscy}{r}]}}{\color{darkgreen}{\small{ m}}}{\small{ (-s)}}{\textit{ (ráðast á e-n)}}{\textsl{\textbf{ gera aðsúg að e-m}}}\foreignlanguage{czech}{{ napadnout, přepadnout (koho)}}}
\entry{{að··svif}}{{\textipa{[{a}{ð}{s}{v}{\textsci}{f}]}}{\color{darkgreen}{\small{ n}}}{\small{ (-s)}}\foreignlanguage{czech}{{ malátnost, mdloby}} $\triangleright$ {\textit{\textbf{ fá aðsvif}}}}
\entry{{að··var|a}}{{\textipa{[{a}{ð}{v}{a}{r}{a}]}}{\color{darkgreen}{\small{ v}}}{\small{ (-aði)}}{\small{ acc}}\foreignlanguage{czech}{{ varovat}} $\triangleright$ {\textit{\textbf{ Pabbi aðvarar börnin.}}}{\textit{\foreignlanguage{czech}{ Táta varuje děti.}}}}
\entry{{að··venta}}{{\textipa{[{a}{ð}{v}{\textepsilon}{\textsubring{n}}{\textsubring{d}}{a}]}}{\color{darkgreen}{\small{ f}}}{\small{ (-u)}}\foreignlanguage{czech}{{ Advent}}}
\entry{{að··vífandi}}{{\textipa{[{a}{ð}{v}{i}{v}{a}{n}{\textsubring{d}}{\textsci}]}}{\color{darkgreen}{\small{ adj}}}{\small{ indecl}}\foreignlanguage{czech}{{ blížící se, nadcházející (často náhodně)}}}
\entry{{að··|vörun}}{{\textipa{[{a}{ð}{v}{\ae}{r}{\textscy}{n}]}}{\color{darkgreen}{\small{ f}}}{\small{ (-vörunar, -varanir)}}\foreignlanguage{czech}{{ výstraha,  varování}} $\triangleright$ {\textit{\textbf{ aðvörun um hættu}}}}
\entry{{að··þrengdur}}{{\textipa{[{a}{ð}{\texttheta}{r}{\textepsilon}{}{\textsubring{d}}{\textscy}{r}]}}{\color{darkgreen}{\small{ adj}}}\foreignlanguage{czech}{{ v těžké situaci, pod silným (psychickým) tlakem}}}
\entry{{af}}{{\textipa{[{a}{\textlengthmark}{f}]}}{\color{darkgreen}{\small{ prep/ adv}}}{\small{ dat}}{ 1.}\foreignlanguage{czech}{{ z, od}}{\footnotesize {\foreignlanguage{czech}{ (o pohybu z nějakého místa)}}}{\textsl{\textbf{ fara af stað}}}\foreignlanguage{czech}{{ vyrazit}},{\textsl{\textbf{ koma af fundi}}}\foreignlanguage{czech}{{ přijít ze schůze}},{\textsl{\textbf{ taka af sér skóna}}}\foreignlanguage{czech}{{ sundat si boty}},{\textsl{\textbf{ vakna af svefni}}}\foreignlanguage{czech}{{ probudit se ze spánku}},{ 2.}\foreignlanguage{czech}{{ po}}{\footnotesize {\foreignlanguage{czech}{ (o čase)}}},{\textsl{\textbf{ vera af barnsaldri}}}\foreignlanguage{czech}{{ vyrůst z dětského věku}},{\textsl{\textbf{ af dagsetri}}}\foreignlanguage{czech}{{ po západu slunce}},{\textsl{\textbf{ hver af öðrum}}}\foreignlanguage{czech}{{ jeden po druhém}},{\textit{ (strax)}}{\textsl{\textbf{ af stundu}}}\foreignlanguage{czech}{{ hned}},{ 3.}{\footnotesize {\foreignlanguage{czech}{ (o důvodu, příčině)}}},{\textsl{\textbf{ gera e-ð af skyldurækni}}}\foreignlanguage{czech}{{ dělat (co) svědomitě}},{\textsl{\textbf{ af hverju?}}}\foreignlanguage{czech}{{ proč}},{\textsl{\textbf{ af því, af því að}}}\foreignlanguage{czech}{{ protože}} $\triangleright$ {\textit{\textbf{ Hann tekur lýsi af því að hann vill ekki fá kvef.}}}{\textit{\foreignlanguage{czech}{ Bere rybí olej, protože nechce nastydnout.}}},{ 4.}{\footnotesize {\foreignlanguage{czech}{ (o přemístění (čeho), části (čeho))}}},{\textsl{\textbf{ hluti af þessu}}}\foreignlanguage{czech}{{ část čeho}},{\textsl{\textbf{ kominn af góðu fólki}}}\foreignlanguage{czech}{{ být z dobré rodiny}},{\textsl{\textbf{ kaupa e-ð af e-m}}}\foreignlanguage{czech}{{ koupit (co) od (koho)}},{ 5.}{\footnotesize {\foreignlanguage{czech}{ (o původci děje)}}} $\triangleright$ {\textit{\textbf{ Greinin er samin af henni.}}}{\textit{\foreignlanguage{czech}{ Napsala ten článek.}}},{\textit{ (því er lokið)}}{\textsl{\textbf{ það er af}}}\foreignlanguage{czech}{{ ???}},{\textsl{\textbf{ það er af og frá}}}\foreignlanguage{czech}{{ to je vyloučené}},{ 6.}{\footnotesize {\foreignlanguage{czech}{ (v ustálených spojeních)}}},{\textit{ (stundum)}}{\textsl{\textbf{ af og til}}}\foreignlanguage{czech}{{ občas, někdy}},}
\entry{{afa··|bróðir}}{{\textipa{[{a}{\textlengthmark}{v}{a}{\textsubring{b}}{r}{ou}{ð}{\textsci}{r}]}}{\color{darkgreen}{\small{ m}}}{\small{ (-bróður, -bræður)}}\foreignlanguage{czech}{{ prastrýc}}}
\entry{{afar}}{{\textipa{[{a}{\textlengthmark}{v}{a}{r}]}}{\color{darkgreen}{\small{ adv}}}{\textit{ (mjög)}}\foreignlanguage{czech}{{ velmi, hodně}}}
\entry{{afar··kostir}}{{\textipa{[{a}{v}{a}{\textsubring{r}}{k\textsuperscript{h}}{\textopeno}{s}{\textsubring{d}}{\textsci}{r}]}}{\color{darkgreen}{\small{ m}}}{\small{ pl}}\foreignlanguage{czech}{{ těžké podmínky}} $\triangleright$ {\textit{\textbf{ setja honum afarkosti}}}}
\entry{{afa··syst|ir}}{{\textipa{[{a}{\textlengthmark}{v}{a}{s}{\textsci}{s}{\textsubring{d}}{\textsci}{r}]}}{\color{darkgreen}{\small{ f}}}{\small{ (-ur, -ur)}}\foreignlanguage{czech}{{ prateta}}}
\entry{{af··bak|a}}{{\textipa{[{a}{f}{\textsubring{b}}{a}{\r{g}}{a}]}}{\color{darkgreen}{\small{ v}}}{\small{ (-aði)}}{\small{ acc}}\foreignlanguage{czech}{{ překroutit, zkomolit (význam ap.)}} $\triangleright$ {\textit{\textbf{ afbaka ummæli hennar}}}}
\entry{{af··boð|a}}{{\textipa{[{a}{f}{\textsubring{b}}{\textopeno}{ð}{a}]}}{\color{darkgreen}{\small{ v}}}{\small{ (-aði)}}{\small{ acc}}{\textit{ (aflýsa)}}\foreignlanguage{czech}{{ zrušit, odvolat, odhlásit}} $\triangleright$ {\textit{\textbf{ afboða samkomuna}}}}
\entry{{af··borg|un}}{{\textipa{[{a}{f}{\textsubring{b}}{\textopeno}{r}{\r{g}}{\textscy}{n}]}}{\color{darkgreen}{\small{ f}}}{\small{ (-unar, -anir)}}\foreignlanguage{czech}{{ půjčka}} $\triangleright$ {\textit{\textbf{ kaupa e-ð á (með) afborgunum}}}\foreignlanguage{czech}{{\footnotesize{ ekon}}}\foreignlanguage{czech}{{ amortizace}},}
\entry{{af·borgunar··tím|i}}{{\textipa{[{a}{f}{\textsubring{b}}{\textopeno}{\textsubring{r}}{\r{g}}{\textscy}{n}{a}{r}{t\textsuperscript{h}}{i}{m}{\textsci}]}}{\color{darkgreen}{\small{ m}}}{\small{ (-a)}}\foreignlanguage{czech}{{\footnotesize{ ekon.}}}\foreignlanguage{czech}{{ doba amortizace}}}
\entry{{af··brigði}}{{\textipa{[{a}{f}{\textsubring{b}}{r}{\textsci}{\textbabygamma}{ð}{\textsci}]}}{\color{darkgreen}{\small{ n}}}{\small{ (-s, -)}}{ 1.}{\textit{ (útgáfa)}}\foreignlanguage{czech}{{ varianta, obměna}} $\triangleright$ {\textit{\textbf{ Sagan er þekkt í tveimur afbrigđum.}}}{\textsl{\textbf{ með afbrigðum}}}\foreignlanguage{czech}{{ velmi, opravdu}} $\triangleright$ {\textit{\textbf{ Myndin er með afbrigðum skemmtileg.}}},{ 2.}{\textit{ (undantekning)}}\foreignlanguage{czech}{{ vyjímka, abnormalita}} $\triangleright$ {\textit{\textbf{ afbrigði frá þingsköpum}}},}
\entry{{af·brigði··legur}}{{\textipa{[{a}{f}{\textsubring{b}}{r}{\textsci}{\textbabygamma}{ð}{\textsci}{l}{\textepsilon}{\textbabygamma}{\textscy}{r}]}}{\color{darkgreen}{\small{ adj}}}{ 1.}\foreignlanguage{czech}{{ abnormální, anomální}} $\triangleright$ {\textit{\textbf{ afbrigðileg kynhegðun}}}{ 2.}\foreignlanguage{czech}{{ neobvyklý, nepravidelný}} $\triangleright$ {\textit{\textbf{ afbrigðilegar aðstæður}}},}
\entry{{af··brot}}{{\textipa{[{a}{f}{\textsubring{b}}{r}{\textopeno}{\textsubring{d}}]}}{\color{darkgreen}{\small{ n}}}{\small{ (-s, -)}}\foreignlanguage{czech}{{ delikvence, přestupek; zločin}} $\triangleright$ {\textit{\textbf{ fremja alvarlegt afbrot}}}}
\entry{{af·brota··|maður}}{{\textipa{[{a}{f}{\textsubring{b}}{r}{\textopeno}{\textsubring{d}}{a}{m}{a}{ð}{\textscy}{r}]}}{\color{darkgreen}{\small{ m}}}{\small{ (-manns, -menn)}}\foreignlanguage{czech}{{ delikvent, recidivista}}}
\entry{{af··brýði}}{{\textipa{[{a}{f}{\textsubring{b}}{r}{i}{ð}{\textsci}]}}{\color{darkgreen}{\small{ f}}}{\small{ indecl}}\foreignlanguage{czech}{{ žárlivost}}}
\entry{{af·brýði··|samur}}{{\textipa{[{a}{f}{\textsubring{b}}{r}{i}{ð}{\textsci}{s}{a}{m}{\textscy}{r}]}}{\color{darkgreen}{\small{ adj}}}{\small{ (f -söm)}}\foreignlanguage{czech}{{ žárlivý}}}
\entry{{af·brýði··semi}}{{\textipa{[{a}{f}{\textsubring{b}}{r}{i}{ð}{\textsci}{s}{\textepsilon}{m}{\textsci}]}}{\color{darkgreen}{\small{ f}}}{\small{ indecl}}\foreignlanguage{czech}{{ žárlivost}} $\triangleright$ {\textit{\textbf{ finna til afbrýðisemi gagnvart honum}}}}
\entry{{af··|bökun}}{{\textipa{[{a}{f}{\textsubring{b}}{\ae}{\r{g}}{\textscy}{n}]}}{\color{darkgreen}{\small{ f}}}{\small{ (-bökunar, -bakanir)}}\foreignlanguage{czech}{{ překroucení, zkomolení}}}
\entry{{af··dal|ur}}{{\textipa{[{a}{f}{\textsubring{d}}{a}{l}{\textscy}{r}]}}{\color{darkgreen}{\small{ m}}}{\small{ (-s, -ir)}}\foreignlanguage{czech}{{ vedlejší údolí (vycházející z hlavního údolí)}}}
\entry{{af·dráttar··laus}}{{\textipa{[{a}{f}{\textsubring{d}}{r}{au}{h}{\textsubring{d}}{a}{r}{l}{\ae i}{s}]}}{\color{darkgreen}{\small{ adj}}}\foreignlanguage{czech}{{ bezpodmínečný, naprostý}}}
\entry{{af··drep}}{{\textipa{[{a}{f}{\textsubring{d}}{r}{\textepsilon}{\textsubring{b}}]}}{\color{darkgreen}{\small{ n}}}{\small{ (-s, -)}}{\textit{ (skjól)}}\foreignlanguage{czech}{{ úkryt, útočiště}}}
\entry{{af··drif}}{{\textipa{[{a}{f}{\textsubring{d}}{r}{\textsci}{f}]}}{\color{darkgreen}{\small{ n}}}{\small{ pl}}{ 1.}{\textit{ (forlög)}}\foreignlanguage{czech}{{ osud}} $\triangleright$ {\textit{\textbf{ óttast um afdrif fjallgöngumannanna}}}{ 2.}{\textit{ (úrslit)}}\foreignlanguage{czech}{{ výsledek}},}
\entry{{af·drifa··ríkur}}{{\textipa{[{a}{f}{\textsubring{d}}{r}{\textsci}{v}{a}{r}{i}{\r{g}}{\textscy}{r}]}}{\color{darkgreen}{\small{ adj}}}\foreignlanguage{czech}{{ osudový, s velkými následky}} $\triangleright$ {\textit{\textbf{ afdrifarík ákvörðun}}}}
\entry{{af··dæmi}}{{\textipa{[{a}{f}{\textsubring{d}}{a}{i}{m}{\textsci}]}}{\color{darkgreen}{\small{ n}}}{\small{ (-s, -)}}{\textsl{\textbf{ með afdæmum}}}\foreignlanguage{czech}{{ mimořádně}}}
\entry{{af··ferm|a}}{{\textipa{[{a}{f}{f}{\textepsilon}{r}{m}{a}]}}{\color{darkgreen}{\small{ v}}}{\small{ (-di, -t)}}{\small{ acc}}\foreignlanguage{czech}{{ vyložit, složit, vylodit}} $\triangleright$ {\textit{\textbf{ Björgunarsveitarmenn afferma gáminn.}}}{\textit{\foreignlanguage{czech}{ Záchranáři vykládají kontejner.}}}{\textit{ (ferma)}}}
\entry{{af··|flytja}}{{\textipa{[{a}{f}{f}{l}{\textsci}{\textsubring{d}}{j}{a}]}}{\color{darkgreen}{\small{ v}}}{\small{ (-flutti, -flutt)}}{\small{ acc}}{\textit{ (rangfæra)}}\foreignlanguage{czech}{{ překroutit, zkomolit}}}
\entry{{af··föll}}{{\textipa{[{a}{f}{f}{\ae}{\textsubring{d}}{\textsubring{l}}]}}{\color{darkgreen}{\small{ n}}}{\small{ pl}}{ 1.}{\textit{ (afsláttur)}}\foreignlanguage{czech}{{ sleva}}{ 2.}{\textit{ (tap)}}\foreignlanguage{czech}{{ úbytek}},}
\entry{{af··|gamall}}{{\textipa{[{a}{f}{\r{g}}{a}{m}{a}{\textsubring{d}}{\textsubring{l}}]}}{\color{darkgreen}{\small{ adj}}}{\small{ (f -gömul)}}\foreignlanguage{czech}{{ velmi starý}}}
\entry{{af··gang|ur}}{{\textipa{[{a}{f}{\r{g}}{au}{}{\r{g}}{\textscy}{r}]}}{\color{darkgreen}{\small{ m}}}{\small{ (-s, -ar)}}\foreignlanguage{czech}{{ zbytek}} $\triangleright$ {\textit{\textbf{ Það er afgangur af matnum.}}}{\textit{\foreignlanguage{czech}{ Zbylo jídlo.}}}}
\entry{{af··gerandi}}{{\textipa{[{a}{f}{\r{\textObardotlessj}}{\textepsilon}{r}{a}{n}{\textsubring{d}}{\textsci}]}}{\color{darkgreen}{\small{ adj}}}{\small{ indecl}}\foreignlanguage{czech}{{ přesvědčivý}} $\triangleright$ {\textit{\textbf{ afgerandi svar}}}}
\entry{{af··gir|ða}}{{\textipa{[{a}{f}{\r{\textObardotlessj}}{\textsci}{r}{ð}{a}]}}{\color{darkgreen}{\small{ v}}}{\small{ (-ti, -t)}}{\small{ acc}}{ 1.}\foreignlanguage{czech}{{ ohradit, obehnat (plotem ap.)}} $\triangleright$ {\textit{\textbf{ afgirða tún}}}{ 2.}\foreignlanguage{czech}{{ odříznout (přerušit spojení)}} $\triangleright$ {\textit{\textbf{ Dalurinn er afgirtur á vetrum vegna snjóa.}}},}
\entry{{af··glap|i}}{{\textipa{[{a}{f}{\r{g}}{l}{a}{\textsubring{b}}{\textsci}]}}{\color{darkgreen}{\small{ m}}}{\small{ (-a, -ar)}}{\textit{ (bjálfi)}}\foreignlanguage{czech}{{ blázen, hlupák}}}
\entry{{af··glöp}}{{\textipa{[{a}{f}{\r{g}}{l}{\ae}{\textsubring{b}}]}}{\color{darkgreen}{\small{ n}}}{\small{ pl}}\foreignlanguage{czech}{{ pošetilost, trapas}} $\triangleright$ {\textit{\textbf{ Henni hafa orðið á hroðaleg afglöp.}}}}
\entry{{af··grei|ða}}{{\textipa{[{a}{f}{\r{g}}{r}{ei}{ð}{a}]}}{\color{darkgreen}{\small{ v}}}{\small{ (-ddi, -tt)}}{\small{ acc}}{ 1.}{\textit{ (þjóna)}}\foreignlanguage{czech}{{ obsluhovat, obsloužit, prodávat (v obchodě ap.)}} $\triangleright$ {\textit{\textbf{ afgreiða viðskiptavin}}}{\textit{\foreignlanguage{czech}{ obsloužit zákazníka}}}{ 2.}{\textit{ (senda)}}\foreignlanguage{czech}{{ odbavit, připravit k vydání}} $\triangleright$ {\textit{\textbf{ afgreiða skip í höfn}}}{\textit{\foreignlanguage{czech}{ odbavit loď v přístavu}}},{ 3.}\foreignlanguage{czech}{{ rozhodnout, určit}} $\triangleright$ {\textit{\textbf{ afgreiða mál}}}{\textit{\foreignlanguage{czech}{ rozhodnout záležitost}}},}
\entry{{af··greiðsl|a}}{{\textipa{[{a}{f}{\r{g}}{r}{ei}{ð}{s}{\textsubring{d}}{l}{a}]}}{\color{darkgreen}{\small{ f}}}{\small{ (-u, -ur)}}{ 1.}{\textit{ (\textsuperscript{1}þjónusta)}}\foreignlanguage{czech}{{ služba, servis}} $\triangleright$ {\textit{\textbf{ afgreiðsla á pöntunum}}}{ 2.}\foreignlanguage{czech}{{ expedice (zboží ap.)}} $\triangleright$ {\textit{\textbf{ póstafgreiðsla, bensínaafgreiðsla}}},}
\entry{{af·greiðslu··frest|ur}}{{\textipa{[{a}{f}{\r{g}}{r}{ei}{ð}{s}{\textsubring{d}}{l}{\textscy}{f}{r}{\textepsilon}{s}{\textsubring{d}}{\textscy}{r}]}}{\color{darkgreen}{\small{ m}}}{\small{ (-s, -ir)}}\foreignlanguage{czech}{{ doba dodání}}}
\entry{{af·greiðslu··|kona}}{{\textipa{[{a}{f}{\r{g}}{r}{ei}{ð}{s}{\textsubring{d}}{l}{\textscy}{k\textsuperscript{h}}{\textopeno}{n}{a}]}}{\color{darkgreen}{\small{ f}}}{\small{ (-u, -ur)}}\foreignlanguage{czech}{{ prodavačka}}}
\entry{{af·greiðslu··|maður}}{{\textipa{[{a}{f}{\r{g}}{r}{ei}{ð}{s}{\textsubring{d}}{l}{\textscy}{m}{a}{ð}{\textscy}{r}]}}{\color{darkgreen}{\small{ m}}}{\small{ (-manns, -menn)}}\foreignlanguage{czech}{{ prodavač}}}
\entry{{af·greiðslu··stúlk|a}}{{\textipa{[{a}{f}{\r{g}}{r}{ei}{ð}{s}{\textsubring{d}}{l}{\textscy}{s}{\textsubring{d}}{u}{\textsubring{l}}{\r{g}}{a}]}}{\color{darkgreen}{\small{ f}}}{\small{ (-u, -ur)}}\foreignlanguage{czech}{{ prodavačka}}}
\entry{{af··hend|a}}{{\textipa{[{a}{\textlengthmark}{f}{\textepsilon}{n}{\textsubring{d}}{a}]}}{\color{darkgreen}{\small{ v}}}{\small{ (-ti, -t)}}{\small{ acc}}{ 1.}\foreignlanguage{czech}{{ předat, odevzdat}} $\triangleright$ {\textit{\textbf{ Skólastjórinn afhenti nemendum verðlaun fyrir góðan námsárangur.}}}{\textit{\foreignlanguage{czech}{ Ŕeditel předal žákům ocenění za dobré školní výsledky.}}}{ 2.}\foreignlanguage{czech}{{ doručit, dodat}} $\triangleright$ {\textit{\textbf{ afhenda honum bréfið}}}{\textit{\foreignlanguage{czech}{ doručit mu dopis}}},}
\entry{{af··hending}}{{\textipa{[{a}{\textlengthmark}{f}{\textepsilon}{n}{\textsubring{d}}{i}{}{\r{g}}]}}{\color{darkgreen}{\small{ f}}}{\small{ (-ar)}}\foreignlanguage{czech}{{ dodání, doručení}}}
\entry{{af·hendingar··dag|ur}}{{\textipa{[{a}{\textlengthmark}{f}{\textepsilon}{n}{\textsubring{d}}{i}{}{\r{g}}{a}{r}{\textsubring{d}}{a}{\textbabygamma}{\textscy}{r}]}}{\color{darkgreen}{\small{ m}}}{\small{ (-s, -ar)}}\foreignlanguage{czech}{{ datum doručení}}}
\entry{{af··hjúp|a}}{{\textipa{[{a}{f}{\c{c}}{u}{\textsubring{b}}{a}]}}{\color{darkgreen}{\small{ v}}}{\small{ (-aði)}}{\small{ acc}}\foreignlanguage{czech}{{ odhalit, odkrýt}} $\triangleright$ {\textit{\textbf{ afhjúpa misferli}}}}
\entry{{af··hjúp|un}}{{\textipa{[{a}{f}{\c{c}}{u}{\textsubring{b}}{\textscy}{n}]}}{\color{darkgreen}{\small{ f}}}{\small{ (-unar)}}\foreignlanguage{czech}{{ odhalení, odkrytí}}}
\entry{{af··hroð}}{{\textipa{[{a}{f}{\textsubring{r}}{\textopeno}{\texttheta}]}}{\color{darkgreen}{\small{ n}}}{\small{ (-s)}}{\textit{ (tjón)}}\foreignlanguage{czech}{{ škoda, ztráta}} $\triangleright$ {\textit{\textbf{ gjalda afhroð}}}}
\entry{{af··huga}}{{\textipa{[{a}{\textlengthmark}{f}{\textscy}{\textbabygamma}{a}]}}{\color{darkgreen}{\small{ adj}}}{\small{ indecl (dat)}}{\textit{ (missa áhuga á e-u)}}{\textsl{\textbf{ verða e-u afhuga}}}\foreignlanguage{czech}{{ ztratit zájem o (co)}} $\triangleright$ {\textit{\textbf{ verða starfinu afhuga}}}}
\entry{{af··hý|ða}}{{\textipa{[{a}{\textlengthmark}{f}{i}{ð}{a}]}}{\color{darkgreen}{\small{ v}}}{\small{ (-ddi, -tt)}}{\small{ acc}}{\textit{ (flysja)}}\foreignlanguage{czech}{{ oloupat, oškrabat}} $\triangleright$ {\textit{\textbf{ afhýða kartöflur}}}{\textit{\foreignlanguage{czech}{ oloupat brambory}}}}
\entry{{af|i}}{{\textipa{[{a}{\textlengthmark}{v}{\textsci}]}}{\color{darkgreen}{\small{ m}}}{\small{ (-a, -ar)}}\foreignlanguage{czech}{{ dědeček, děda}} $\triangleright$ {\textit{\textbf{ Palli á tvo afa.}}}{\textit{\foreignlanguage{czech}{ Palli má dva dědečky.}}}}
\entry{{af··kast|a}}{{\textipa{[{a}{f}{k\textsuperscript{h}}{a}{s}{\textsubring{d}}{a}]}}{\color{darkgreen}{\small{ v}}}{\small{ (-aði)}}{\small{ dat}}\foreignlanguage{czech}{{ provést, udělat}} $\triangleright$ {\textit{\textbf{ afkasta miklu}}}}
\entry{{af·kasta··get|a}}{{\textipa{[{a}{f}{k\textsuperscript{h}}{a}{s}{\textsubring{d}}{a}{\r{\textObardotlessj}}{\textepsilon}{\textsubring{d}}{a}]}}{\color{darkgreen}{\small{ f}}}{\small{ (-u)}}\foreignlanguage{czech}{{ produktivita}}}
\entry{{af·kasta··|maður}}{{\textipa{[{a}{f}{k\textsuperscript{h}}{a}{s}{\textsubring{d}}{a}{m}{a}{ð}{\textscy}{r}]}}{\color{darkgreen}{\small{ m}}}{\small{ (-manns, -menn)}}\foreignlanguage{czech}{{ pilný, usilovný pracovník}}}
\entry{{af·kasta··mikill}}{{\textipa{[{a}{f}{k\textsuperscript{h}}{a}{s}{\textsubring{d}}{a}{m}{\textsci}{\r{\textObardotlessj}}{\textsci}{\textsubring{d}}{\textsubring{l}}]}}{\color{darkgreen}{\small{ adj}}}\foreignlanguage{czech}{{ výkonný, efektivní, produktivní}} $\triangleright$ {\textit{\textbf{ afkastamikill listamaður}}}}
\entry{{af·kára··legur}}{{\textipa{[{a}{f}{k\textsuperscript{h}}{au}{r}{a}{l}{\textepsilon}{\textbabygamma}{\textscy}{r}]}}{\color{darkgreen}{\small{ adj}}}{\textit{ (hlægilegur)}}\foreignlanguage{czech}{{ bizarní, prapodivný}} $\triangleright$ {\textit{\textbf{ afkáralegur hugmynd}}}}
\entry{{af··kim|i}}{{\textipa{[{a}{f}{c\textsuperscript{h}}{\textsci}{m}{\textsci}]}}{\color{darkgreen}{\small{ m}}}{\small{ (-a, -ar)}}{\textit{ (skot)}}\foreignlanguage{czech}{{ kout, koutek}}}
\entry{{af··klæ|ða}}{{\textipa{[{a}{f}{k\textsuperscript{h}}{l}{a}{i}{ð}{a}]}}{\color{darkgreen}{\small{ v}}}{\small{ (-ddi, -tt)}}{\small{ acc}}{\textit{ (fara úr)}}\foreignlanguage{czech}{{ sundat, svléknout}} $\triangleright$ {\textit{\textbf{ afklæða sig}}}{\textit{ (klæða)}}}
\entry{{af··kom|a}}{{\textipa{[{a}{f}{k\textsuperscript{h}}{\textopeno}{m}{a}]}}{\color{darkgreen}{\small{ f}}}{\small{ (-u)}}{ 1.}{\textit{ (arður)}}\foreignlanguage{czech}{{ zisk, výnos}} $\triangleright$ {\textit{\textbf{ reyna að tryggja afkomu fyrirtækisins með víðtækum ráðstöfunum.}}}{ 2.}{\textit{ (efnahagur)}}\foreignlanguage{czech}{{ finanční postavení}},}
\entry{{af··kom|andi}}{{\textipa{[{a}{f}{k\textsuperscript{h}}{\textopeno}{m}{a}{n}{\textsubring{d}}{\textsci}]}}{\color{darkgreen}{\small{ m}}}{\small{ (-anda, -endur)}}\foreignlanguage{czech}{{ potomek}}}
\entry{{af··kvæmi}}{{\textipa{[{a}{f}{k\textsuperscript{h}}{v}{a}{i}{m}{\textsci}]}}{\color{darkgreen}{\small{ n}}}{\small{ (-s, -)}}{ 1.}\foreignlanguage{czech}{{ dítě, potomek}}{ 2.}{\textit{ (afsprengi)}}\foreignlanguage{czech}{{ potomek, mládě}} $\triangleright$ {\textit{\textbf{ Sum dýr eiga mörg afkvæmi.}}}{\textit{\foreignlanguage{czech}{ Některá zvířata mají hodně mláďat.}}},}
\entry{{af··köst}}{{\textipa{[{a}{f}{k\textsuperscript{h}}{\ae}{s}{\textsubring{d}}]}}{\color{darkgreen}{\small{ n}}}{\small{ pl}}\foreignlanguage{czech}{{ výkon, produktivita}} $\triangleright$ {\textit{\textbf{ Afköstin aukast.}}}{\textit{\foreignlanguage{czech}{ Výkon se zvětšuje.}}}}
\entry{{afl}}{{\textipa{[{a}{\textsubring{b}}{\textsubring{l}}]}}{\color{darkgreen}{\small{ n}}}{\small{ (afls, öfl)}}{ 1.}\foreignlanguage{czech}{{ (lidská) síla, elán, potence}} $\triangleright$ {\textit{\textbf{ margra manna afl}}}{\textit{\foreignlanguage{czech}{ síla mnoha lidí}}}{ 2.}{\textit{ (ofbeldi)}}\foreignlanguage{czech}{{ síla, násilí}} $\triangleright$ {\textit{\textbf{ taka e-ð með afli}}}{\textit{\foreignlanguage{czech}{ vzít si (co) silou}}},{ 3.}{\textit{ (kraftur)}}\foreignlanguage{czech}{{ síla, energie}} $\triangleright$ {\textit{\textbf{ öfl náttúrunnar}}}{\textit{\foreignlanguage{czech}{ síly přírody}}},{ 4.}{\textit{ (gildi)}}\foreignlanguage{czech}{{ platnost}},{ 5.}\foreignlanguage{czech}{{\footnotesize{ fyz.}}}\foreignlanguage{czech}{{ výkon}},}
\entry{{afl.}}{{\color{darkgreen}{\small{ zkr}}}{\textsl{\textbf{ afleiddur}}}\foreignlanguage{czech}{{ odvozený}}}
\entry{{afla}}{{\textipa{[{a}{\textsubring{b}}{l}{a}]}}{\color{darkgreen}{\small{ v}}}{\small{ (gen) (-aði)}}{ 1.}{\textit{ (fiska)}}\foreignlanguage{czech}{{ rybařit, chytat/lovit ryby}} $\triangleright$ {\textit{\textbf{ Bátarnir hafa aflað vel að undanförnu.}}}{ 2.}{\textsl{\textbf{ afla sér e-s}}}\foreignlanguage{czech}{{ pořídit si (co), získat (co)}} $\triangleright$ {\textit{\textbf{ afla sér þekkingar}}},}
\entry{{afla··brest|ur}}{{\textipa{[{a}{\textsubring{b}}{l}{a}{\textsubring{b}}{r}{\textepsilon}{s}{\textsubring{d}}{\textscy}{r}]}}{\color{darkgreen}{\small{ m}}}{\small{ (-s)}}\foreignlanguage{czech}{{ žádný úlovek (ryb ap.)}}}
\entry{{\textsuperscript{1}}{af··|laga}}{{\textipa{[{a}{f}{l}{a}{\textbabygamma}{a}]}}{\color{darkgreen}{\small{ f}}}{\small{ (-lögu, -lögur)}}\foreignlanguage{czech}{{ zásoba}} $\triangleright$ {\textit{\textbf{ eiga e-ð aflögu}}}}
\entry{{\textsuperscript{2}}{af··lag|a}}{{\textipa{[{a}{f}{l}{a}{\textbabygamma}{a}]}}{\color{darkgreen}{\small{ v}}}{\small{ (-aði)}}{\textit{ (bæla)}}\foreignlanguage{czech}{{ zkazit, pokazit}} $\triangleright$ {\textit{\textbf{ aflaga hárgreiðsluna}}}}
\entry{{afla··|mark}}{{\textipa{[{a}{\textsubring{b}}{l}{a}{m}{a}{\textsubring{r}}{\r{g}}]}}{\color{darkgreen}{\small{ n}}}{\small{ (-marks, -mörk)}}\foreignlanguage{czech}{{ lovné kvóty}}}
\entry{{af··|langur}}{{\textipa{[{a}{f}{l}{au}{}{\r{g}}{\textscy}{r}]}}{\color{darkgreen}{\small{ adj}}}{\small{ (f -löng)}}\foreignlanguage{czech}{{ obdélníkový}}}
\entry{{af··lausn}}{{\textipa{[{a}{f}{l}{\ae i}{s}{\textsubring{d}}{\textsubring{n}}]}}{\color{darkgreen}{\small{ f}}}{\small{ (-ar)}}\foreignlanguage{czech}{{ rozhřešení}}}
\entry{{af··lát}}{{\textipa{[{a}{f}{l}{au}{\textsubring{d}}]}}{\color{darkgreen}{\small{ n}}}{\small{ (-s, -)}}{\textit{ (hlé)}}\foreignlanguage{czech}{{ přestávka, pauza}}{\textsl{\textbf{ án afláts}}}\foreignlanguage{czech}{{ neustále, nepřetržitě}},}
\entry{{af··|leggja}}{{\textipa{[{a}{f}{l}{\textepsilon}{\r{\textObardotlessj}}{a}]}}{\color{darkgreen}{\small{ v}}}{\small{ (-lagði, -lagt)}}{\small{ acc}}\foreignlanguage{czech}{{ přestat (s čím), opustit (od čeho)}}{\textsl{\textbf{ afleggja vana}}}\foreignlanguage{czech}{{ opustit zvyk}},}
\entry{{af··leggjar|i}}{{\textipa{[{a}{f}{l}{\textepsilon}{\r{\textObardotlessj}}{a}{r}{\textsci}]}}{\color{darkgreen}{\small{ m}}}{\small{ (-a, -ar)}}{ 1.}\foreignlanguage{czech}{{ postranní cesta}}{ 2.}{\textit{ (græðlingur)}}\foreignlanguage{czech}{{\footnotesize{ bot.}}}\foreignlanguage{czech}{{ řízek, odřezek (od rostliny)}},}
\entry{{af··leiddur}}{{\textipa{[{a}{f}{l}{ei}{\textsubring{d}}{\textscy}{r}]}}{\color{darkgreen}{\small{ adj}}}\foreignlanguage{czech}{{ odvozený}}{\textsl{\textbf{ afleitt orð}}}\foreignlanguage{czech}{{\footnotesize{ jaz.}}}\foreignlanguage{czech}{{ odvozené slovo}},}
\entry{{af··leiðing}}{{\textipa{[{a}{f}{l}{ei}{ð}{i}{}{\r{g}}]}}{\color{darkgreen}{\small{ f}}}{\small{ (-ar, -ar)}}{\textit{ (árangur)}}\foreignlanguage{czech}{{ důsledek, následek}} $\triangleright$ {\textit{\textbf{ orsök og afleiðing}}}}
\entry{{af·leiðingar··setning}}{{\textipa{[{a}{f}{l}{ei}{ð}{i}{}{\r{g}}{a}{\textsubring{r}}{s}{\textepsilon}{h}{\textsubring{d}}{n}{i}{}{\r{g}}]}}{\color{darkgreen}{\small{ f}}}{\small{ (-ar, -ar)}}\foreignlanguage{czech}{{\footnotesize{ jaz.}}}\foreignlanguage{czech}{{ vedlejší věta příslovečná účelová}}}
\entry{{af·leiðingar··tenging}}{{\textipa{[{a}{f}{l}{ei}{ð}{i}{}{\r{g}}{a}{\textsubring{r}}{t\textsuperscript{h}}{ei}{\textltailn}{\r{\textObardotlessj}}{i}{}{\r{g}}]}}{\color{darkgreen}{\small{ f}}}{\small{ (-ar, -ar)}}\foreignlanguage{czech}{{\footnotesize{ jaz.}}}\foreignlanguage{czech}{{ spojka účelová}}}
\entry{{af··leiðsl|a}}{{\textipa{[{a}{f}{l}{ei}{ð}{s}{\textsubring{d}}{l}{a}]}}{\color{darkgreen}{\small{ f}}}{\small{ (-u)}}{ 1.}\foreignlanguage{czech}{{\footnotesize{ filos.}}}\foreignlanguage{czech}{{ dedukce}}{ 2.}\foreignlanguage{czech}{{\footnotesize{ jaz.}}}\foreignlanguage{czech}{{ derivace, odvozování}},}
\entry{{af··leit|ur}}{{\textipa{[{a}{f}{l}{ei}{\textsubring{d}}{\textscy}{r}]}}{\color{darkgreen}{\small{ adj}}}{\textit{ (slæmur)}}\foreignlanguage{czech}{{ špatný, nedobrý}} $\triangleright$ {\textit{\textbf{ vera afleitur bílstjóri}}}{\textit{\foreignlanguage{czech}{ nebýt dobrým řidičem}}}}
\entry{{af··leysing}}{{\textipa{[{a}{f}{l}{ei}{s}{i}{}{\r{g}}]}}{\color{darkgreen}{\small{ f}}}{\small{ (-ar, -ar)}}\foreignlanguage{czech}{{ náhrada, vystřídání}}{\textsl{\textbf{ vinna í afleysingum}}}\foreignlanguage{czech}{{ pracovat na výpomoc}},}
\entry{{af··lé|tta}}{{\textipa{[{a}{f}{l}{j}{\textepsilon}{h}{\textsubring{d}}{a}]}}{\color{darkgreen}{\small{ v}}}{\small{ (-tti, -tt)}}{\textit{ (frelsa)}}\foreignlanguage{czech}{{ ulevit, ulehčit}}}
\entry{{afl··gjaf|i}}{{\textipa{[{a}{\textsubring{b}}{\textsubring{l}}{\r{\textObardotlessj}}{a}{v}{\textsci}]}}{\color{darkgreen}{\small{ m}}}{\small{ (-a, -ar)}}\foreignlanguage{czech}{{ zdroj (energie)}}}
\entry{{afl|i}}{{\textipa{[{a}{\textsubring{b}}{l}{\textsci}]}}{\color{darkgreen}{\small{ m}}}{\small{ (-a)}}{ 1.}{\textit{ (afrakstur)}}\foreignlanguage{czech}{{ užitek, prospěch}}{ 4.}{\textit{ (fjöldi)}}\foreignlanguage{czech}{{ množství}} $\triangleright$ {\textit{\textbf{ liðsafli}}}{\textit{\foreignlanguage{czech}{ množství lidí}}},{ 3.}\foreignlanguage{czech}{{ úlovek (rybaření)}} $\triangleright$ {\textit{\textbf{ verka aflann}}}{\textit{\foreignlanguage{czech}{ konzervovat úlovek}}},{ 2.}{\textit{ (birgðir)}}\foreignlanguage{czech}{{ zásoby}},}
\entry{{af··lóga}}{{\textipa{[{a}{f}{l}{ou}{a}]}}{\color{darkgreen}{\small{ adj}}}{\small{ indecl}}\foreignlanguage{czech}{{ sešlý, vetchý}} $\triangleright$ {\textit{\textbf{ aflóga hross}}}}
\entry{{afl·rauna··|maður}}{{\textipa{[{a}{\textsubring{b}}{\textsubring{l}}{r}{\ae i}{n}{a}{m}{a}{ð}{\textscy}{r}]}}{\color{darkgreen}{\small{ m}}}{\small{ (-manns, -menn)}}\foreignlanguage{czech}{{ silák, vzpěrač}}}
\entry{{afls··mun|ur}}{{\textipa{[{a}{\textsubring{b}}{\textsubring{l}}{s}{m}{\textscy}{n}{\textscy}{r}]}}{\color{darkgreen}{\small{ m}}}{\small{ (-ar)}}\foreignlanguage{czech}{{ rozdíl v síle}} $\triangleright$ {\textit{\textbf{ neyta aflsmunar}}}}
\entry{{afl··vak|i}}{{\textipa{[{a}{\textsubring{b}}{\textsubring{l}}{v}{a}{\r{\textObardotlessj}}{\textsci}]}}{\color{darkgreen}{\small{ m}}}{\small{ (-a, -ar)}}\foreignlanguage{czech}{{ hnací síla, motiv}}}
\entry{{af··lýs|a}}{{\textipa{[{a}{f}{l}{i}{s}{a}]}}{\color{darkgreen}{\small{ v}}}{\small{ (-ti, -t)}}{\small{ dat}}{\textit{ (afboða)}}\foreignlanguage{czech}{{ odvolat, zrušit}} $\triangleright$ {\textit{\textbf{ aflýsa fundinum}}}}
\entry{{af·lögu··fær}}{{\textipa{[{a}{f}{l}{\ae}{\textbabygamma}{\textscy}{f}{a}{i}{r}]}}{\color{darkgreen}{\small{ adj}}}\foreignlanguage{czech}{{ štědrý}} $\triangleright$ {\textit{\textbf{ vera aflögufær}}}}
\entry{{af··mark|a}}{{\textipa{[{a}{f}{m}{a}{\textsubring{r}}{\r{g}}{a}]}}{\color{darkgreen}{\small{ v}}}{\small{ (-aði)}}{ 1.}{\textit{ (setja takmörk)}}\foreignlanguage{czech}{{ ohraničit, vytyčit}}{ 2.}{\textit{ (skilgreina)}}\foreignlanguage{czech}{{ vymezit, definovat}} $\triangleright$ {\textit{\textbf{ afmarka hugtök}}}{\textit{\foreignlanguage{czech}{ definovat koncept}}},}
\entry{{af··má}}{{\textipa{[{a}{f}{m}{au}]}}{\color{darkgreen}{\small{ v}}}{\small{ (-ði, -ð)}}{\small{ acc}}{ 1.}{\textit{ (þurrka út)}}\foreignlanguage{czech}{{ vymazat, zničit, zahladit (stopy ap.)}} $\triangleright$ {\textit{\textbf{ afmá verksummerki}}}{ 2.}{\textit{ (drepa)}}\foreignlanguage{czech}{{ zabít}} $\triangleright$ {\textit{\textbf{ afmá stjúpu sína}}},}
\entry{{af·máan··legur}}{{\textipa{[{a}{f}{m}{au}{a}{n}{l}{\textepsilon}{\textbabygamma}{\textscy}{r}]}}{\color{darkgreen}{\small{ adj}}}\foreignlanguage{czech}{{ smazatelný, odstranitelný}}}
\entry{{af··mynd|a}}{{\textipa{[{a}{f}{m}{\textsci}{n}{\textsubring{d}}{a}]}}{\color{darkgreen}{\small{ v}}}{\small{ (-aði)}}{\small{ acc}}{\textit{ (afskræma)}}\foreignlanguage{czech}{{ znetvořit, zohyzdit}} $\triangleright$ {\textit{\textbf{ afmynda e-n}}}}
\entry{{af··myndaður}}{{\textipa{[{a}{f}{m}{\textsci}{n}{\textsubring{d}}{a}{ð}{\textscy}{r}]}}{\color{darkgreen}{\small{ adj}}}\foreignlanguage{czech}{{ znetvořený, zohyzdění}} $\triangleright$ {\textit{\textbf{ vera afmyndaður í framan}}}}
\entry{{af··mynd|un}}{{\textipa{[{a}{f}{m}{\textsci}{n}{\textsubring{d}}{\textscy}{n}]}}{\color{darkgreen}{\small{ f}}}{\small{ (-unar, -anir)}}\foreignlanguage{czech}{{ znetvoření, zohyzdění}}}
\entry{{af··mæli}}{{\textipa{[{a}{f}{m}{a}{i}{l}{\textsci}]}}{\color{darkgreen}{\small{ n}}}{\small{ (-s, -)}}{\textit{ (fæðingardagur)}}\foreignlanguage{czech}{{ narozeniny}} $\triangleright$ {\textit{\textbf{ Palli á afmæli.}}}{\textit{\foreignlanguage{czech}{ Palli má narozeniny.}}}}
\entry{{af·mælis··|barn}}{{\textipa{[{a}{f}{m}{a}{i}{l}{\textsci}{s}{\textsubring{b}}{a}{r}{\textsubring{d}}{\textsubring{n}}]}}{\color{darkgreen}{\small{ n}}}{\small{ (-barns, -börn)}}\foreignlanguage{czech}{{ narozeninový oslavenec}}}
\entry{{af·mælis··dag|ur}}{{\textipa{[{a}{f}{m}{a}{i}{l}{\textsci}{s}{\textsubring{d}}{a}{\textbabygamma}{\textscy}{r}]}}{\color{darkgreen}{\small{ m}}}{\small{ (-s, -ar)}}\foreignlanguage{czech}{{ narozeninový den}}}
\entry{{af·mælis··|gjöf}}{{\textipa{[{a}{f}{m}{a}{i}{l}{\textsci}{s}{\r{\textObardotlessj}}{\ae}{f}]}}{\color{darkgreen}{\small{ f}}}{\small{ (-gjafar, -gjafir)}}\foreignlanguage{czech}{{ narozeninový dárek}} $\triangleright$ {\textit{\textbf{ fá afmælisgjöf frá henni}}}}
\entry{{af·mælis··veisl|a}}{{\textipa{[{a}{f}{m}{a}{i}{l}{\textsci}{s}{v}{ei}{s}{\textsubring{d}}{l}{a}]}}{\color{darkgreen}{\small{ f}}}{\small{ (-u, -ur)}}\foreignlanguage{czech}{{ narozeninová oslava}}}
\entry{{af··nám}}{{\textipa{[{a}{f}{n}{au}{m}]}}{\color{darkgreen}{\small{ n}}}{\small{ (-s)}}\foreignlanguage{czech}{{ zrušení, odvolání}} $\triangleright$ {\textit{\textbf{ afnám laga}}}}
\entry{{af··neit|a}}{{\textipa{[{a}{f}{n}{ei}{\textsubring{d}}{a}]}}{\color{darkgreen}{\small{ v}}}{\small{ (-aði)}}{\small{ dat}}{ 1.}{\textit{ (hafna)}}\foreignlanguage{czech}{{ popřít, odmítnout}} $\triangleright$ {\textit{\textbf{ afneita þessum ásökunum}}}{ 2.}\foreignlanguage{czech}{{ zříci se}} $\triangleright$ {\textit{\textbf{ afneita guði}}},}
\entry{{af··neit|un}}{{\textipa{[{a}{f}{n}{ei}{\textsubring{d}}{\textscy}{n}]}}{\color{darkgreen}{\small{ f}}}{\small{ (-unar, -anir)}}\foreignlanguage{czech}{{ odřeknutí se, vzdání se}} $\triangleright$ {\textit{\textbf{ trúarafneitun}}}}
\entry{{af··|nema}}{{\textipa{[{a}{f}{n}{\textepsilon}{m}{a}]}}{\color{darkgreen}{\small{ v}}}{\small{ (-nem, -nam, -námum, -numið)}}{\small{ acc}}{\textit{ (fella niður)}}\foreignlanguage{czech}{{ zrušit, odvolat, prohlásit za neplatné (zákon ap.)}} $\triangleright$ {\textit{\textbf{ afnema regluna}}}}
\entry{{af··not}}{{\textipa{[{a}{f}{n}{\textopeno}{\textsubring{d}}]}}{\color{darkgreen}{\small{ n}}}{\small{ pl}}{\textit{ (gagn)}}\foreignlanguage{czech}{{ využití, užitek}} $\triangleright$ {\textit{\textbf{ afnot af húsinu}}}}
\entry{{af·nota··|gjald}}{{\textipa{[{a}{f}{n}{\textopeno}{\textsubring{d}}{a}{\r{\textObardotlessj}}{a}{l}{\textsubring{d}}]}}{\color{darkgreen}{\small{ n}}}{\small{ (-gjalds, -gjöld)}}\foreignlanguage{czech}{{\footnotesize{ ekon.}}}\foreignlanguage{czech}{{ poplatek za licenci}}}
\entry{{af·nota··rétt|ur}}{{\textipa{[{a}{f}{n}{\textopeno}{\textsubring{d}}{a}{r}{j}{\textepsilon}{h}{\textsubring{d}}{\textscy}{r}]}}{\color{darkgreen}{\small{ m}}}{\small{ (-ar)}}\foreignlanguage{czech}{{ užívací právo}}}
\entry{{af··pant|a}}{{\textipa{[{a}{f}{p\textsuperscript{h}}{a}{\textsubring{n}}{\textsubring{d}}{a}]}}{\color{darkgreen}{\small{ v}}}{\small{ (-aði)}}{\small{ acc}}\foreignlanguage{czech}{{ zrušit (objednávku ap.)}} $\triangleright$ {\textit{\textbf{ afpanta gistinguna}}}{\textit{\foreignlanguage{czech}{ zrušit ubytování}}}}
\entry{{af··plán|a}}{{\textipa{[{a}{f}{p\textsuperscript{h}}{l}{au}{n}{a}]}}{\color{darkgreen}{\small{ v}}}{\small{ (-aði)}}{\small{ acc}}\foreignlanguage{czech}{{ odpykat, odčinit}} $\triangleright$ {\textit{\textbf{ afplána þungan fangelsisdóm}}}}
\entry{{af··rakstur}}{{\textipa{[{a}{f}{r}{a}{x}{s}{\textsubring{d}}{\textscy}{r}]}}{\color{darkgreen}{\small{ m}}}{\small{ (-s)}}{ 1.}{\textit{ (framleiðsla)}}\foreignlanguage{czech}{{ výrobek, produkt}}{ 2.}{\textit{ (hagnaður)}}\foreignlanguage{czech}{{ výnos, zisk}},}
\entry{{af··|ráða}}{{\textipa{[{a}{f}{r}{au}{ð}{a}]}}{\color{darkgreen}{\small{ v}}}{\small{ (-ræð, -réði, -réðum, -ráðið )}}{\small{ acc}}\foreignlanguage{czech}{{ rozhodnout (se), učinit rozhodnutí}}}
\entry{{af··rek}}{{\textipa{[{a}{f}{r}{\textepsilon}{\r{g}}]}}{\color{darkgreen}{\small{ n}}}{\small{ ( -s, -)}}\foreignlanguage{czech}{{ úsilí, výkon}} $\triangleright$ {\textit{\textbf{ vinna mikið afrek}}}}
\entry{{af·reks··verk}}{{\textipa{[{a}{f}{r}{\textepsilon}{\r{g}}{s}{v}{\textepsilon}{\textsubring{r}}{\r{g}}]}}{\color{darkgreen}{\small{ n}}}{\small{ ( -s, -)}}\foreignlanguage{czech}{{ hrdinský čin}}}
\entry{{af··rétt}}{{\textipa{[{a}{f}{r}{j}{\textepsilon}{h}{\textsubring{d}}]}}{\color{darkgreen}{\small{ m}}}{\small{ (-ar, -ir)}}\foreignlanguage{czech}{{ horská pastva/ pastvina}}}
\entry{{af··réttar|i}}{{\textipa{[{a}{f}{r}{j}{\textepsilon}{h}{\textsubring{d}}{a}{r}{\textsci}]}}{\color{darkgreen}{\small{ m}}}{\small{ (-a, -ar)}}\foreignlanguage{czech}{{ posilující prostředek}}}
\entry{{af··rétt|ur}}{{\textipa{[{a}{f}{r}{j}{\textepsilon}{h}{\textsubring{d}}{\textscy}{r}]}}{\color{darkgreen}{\small{ m}}}{\small{ (-ar, -ir)}} $\rightarrow$       afrétt}
\entry{{af··rit}}{{\textipa{[{a}{f}{r}{\textsci}{\textsubring{d}}]}}{\color{darkgreen}{\small{ n}}}{\small{ ( -s, -)}}{ 1.}\foreignlanguage{czech}{{ kopie}} $\triangleright$ {\textit{\textbf{ afrit af skjalinu}}}{ 2.}\foreignlanguage{czech}{{ opis, přepis}},}
\entry{{af··rit|a}}{{\textipa{[{a}{f}{r}{\textsci}{\textsubring{d}}{a}]}}{\color{darkgreen}{\small{ v}}}{\small{ (-aði)}}{\small{ acc}}{ 1.}{\textit{ (taka afrit)}}\foreignlanguage{czech}{{ opsat, přepsat (dopis ap.)}} $\triangleright$ {\textit{\textbf{ afrita bréf}}}{\textit{\foreignlanguage{czech}{ přepsat dopis}}}{\textsl{\textbf{ afrita til öryggis}}}\foreignlanguage{czech}{{\footnotesize{ poč.}}}\foreignlanguage{czech}{{ (za)zálohovat}},{ 2.}\foreignlanguage{czech}{{\footnotesize{ poč.}}}\foreignlanguage{czech}{{ (z)kopírovat (soubor ap.)}} $\triangleright$ {\textit{\textbf{ Hann afritar valinn texta.}}}{\textit{\foreignlanguage{czech}{ Kopíruje vybraný text.}}},}
\entry{{Afrík|a}}{{\textipa{[{a}{v}{r}{i}{\r{g}}{a}]}}{\color{darkgreen}{\small{ f}}}{\small{ (-u)}}\foreignlanguage{czech}{{\footnotesize{ geog.}}}\foreignlanguage{czech}{{ Afrika}} $\triangleright$ {\textit{\textbf{ Í Afríku búa Afríkumenn.}}}{\textit{\foreignlanguage{czech}{ V Africe žijí Afričané.}}}}
\entry{{Afríku··|maður}}{{\textipa{[{a}{v}{r}{i}{\r{g}}{\textscy}{m}{a}{ð}{\textscy}{r}]}}{\color{darkgreen}{\small{ m}}}{\small{ (-manns, -menn)}}\foreignlanguage{czech}{{ Afričan}}}
\entry{{af··ræk|ja}}{{\textipa{[{a}{f}{r}{a}{i}{\r{\textObardotlessj}}{a}]}}{\color{darkgreen}{\small{ v}}}{\small{ (-ti, -t)}}{\small{ acc}}\foreignlanguage{czech}{{ zanedbat, opomenout}} $\triangleright$ {\textit{\textbf{ afrækja starfið}}}}
\entry{{af··röddun}}{{\textipa{[{a}{f}{r}{\ae}{\textsubring{d}}{\textscy}{n}]}}{\color{darkgreen}{\small{ f}}}{\small{ (-ar)}}\foreignlanguage{czech}{{\footnotesize{ jaz.}}}\foreignlanguage{czech}{{ ???}}}
\entry{{af··sak|a}}{{\textipa{[{a}{f}{s}{a}{\r{g}}{a}]}}{\color{darkgreen}{\small{ v}}}{\small{ (-aði)}}{\small{ acc}}{\textit{ (biðjast afsökunar)}}\foreignlanguage{czech}{{ omluvit, omlouvat}} $\triangleright$ {\textit{\textbf{ afsaka hegðun hans}}}{\textit{\foreignlanguage{czech}{ omluvit jeho chování}}}{\textsl{\textbf{ afsaka sig}}}\foreignlanguage{czech}{{ omluvit se}},{\textsl{\textbf{ afsakið!}}}\foreignlanguage{czech}{{ promiň! promiňte! pardon!}},}
\entry{{af·sakan··legur}}{{\textipa{[{a}{f}{s}{a}{\r{g}}{a}{n}{l}{\textepsilon}{\textbabygamma}{\textscy}{r}]}}{\color{darkgreen}{\small{ adj}}}\foreignlanguage{czech}{{ omluvitelný}}}
\entry{{af··|sal}}{{\textipa{[{a}{f}{s}{a}{\textsubring{l}}]}}{\color{darkgreen}{\small{ n}}}{\small{ (-sals, -söl)}}\foreignlanguage{czech}{{\footnotesize{ ekon.}}}\foreignlanguage{czech}{{ doklad (majetkový), vlastnická listina}} $\triangleright$ {\textit{\textbf{ afsal á eigninni}}}}
\entry{{af··sal|a}}{{\textipa{[{a}{f}{s}{a}{l}{a}]}}{\color{darkgreen}{\small{ v}}}{\small{ (-aði)}}{\textit{ (fórna)}}{\textsl{\textbf{ afsala sér}}}\foreignlanguage{czech}{{ vzdát se, zříci se}} $\triangleright$ {\textit{\textbf{ afsala öllum réttindum
}}}}
\entry{{af··sann|a}}{{\textipa{[{a}{f}{s}{a}{n}{a}]}}{\color{darkgreen}{\small{ v}}}{\small{ (-aði)}}{\textit{ (hrekja)}}\foreignlanguage{czech}{{ vyvrátit (tvrzení ap.), dokázat nesprávnost}} $\triangleright$ {\textit{\textbf{ Þessi vitneskja afsannar kenningu hans.
}}}}
\entry{{af··|segja}}{{\textipa{[{a}{f}{s}{ei}{j}{a}]}}{\color{darkgreen}{\small{ v}}}{\small{ (-sagði, -sagt)}}{\small{ acc}}{\textit{ (neita)}}\foreignlanguage{czech}{{ odmítnout}}}
\entry{{af·skap··lega}}{{\textipa{[{a}{f}{s}{\r{g}}{a}{\textsubring{b}}{l}{\textepsilon}{\textbabygamma}{a}]}}{\color{darkgreen}{\small{ adv}}}\foreignlanguage{czech}{{ velmi, extrémně}}}
\entry{{af·skap··legur}}{{\textipa{[{a}{f}{s}{\r{g}}{a}{\textsubring{b}}{l}{\textepsilon}{\textbabygamma}{\textscy}{r}]}}{\color{darkgreen}{\small{ adj}}}{\textit{ (ofsalegur)}}\foreignlanguage{czech}{{ skvělý, nádherný}}}
\entry{{af··skekktur}}{{\textipa{[{a}{f}{s}{\r{\textObardotlessj}}{\textepsilon}{x}{\textsubring{d}}{\textscy}{r}]}}{\color{darkgreen}{\small{ adj}}}\foreignlanguage{czech}{{ odlehlý, izolovaný}}}
\entry{{af·skipta··laus}}{{\textipa{[{a}{f}{s}{\r{\textObardotlessj}}{\textsci}{f}{\textsubring{d}}{a}{l}{\ae i}{s}]}}{\color{darkgreen}{\small{ adj}}}\foreignlanguage{czech}{{ lhostejný, netečný}} $\triangleright$ {\textit{\textbf{ vera afskiptalaus um málið}}}}
\entry{{af·skipta··leysi}}{{\textipa{[{a}{f}{s}{\r{\textObardotlessj}}{\textsci}{f}{\textsubring{d}}{a}{l}{ei}{s}{\textsci}]}}{\color{darkgreen}{\small{ n}}}{\small{ (-s)}}\foreignlanguage{czech}{{ lhostejnost, pasivita}}}
\entry{{af·skipta··|samur}}{{\textipa{[{a}{f}{s}{\r{\textObardotlessj}}{\textsci}{f}{\textsubring{d}}{a}{s}{a}{m}{\textscy}{r}]}}{\color{darkgreen}{\small{ adj}}}{\small{ (f -söm)}}\foreignlanguage{czech}{{ vlezlý, dotěrný}}{\textit{ (fáskiptinn)}}}
\entry{{af·skipta··semi}}{{\textipa{[{a}{f}{s}{\r{\textObardotlessj}}{\textsci}{f}{\textsubring{d}}{a}{s}{\textepsilon}{m}{\textsci}]}}{\color{darkgreen}{\small{ f}}}{\small{ indecl}}\foreignlanguage{czech}{{ vlezlost}}}
\entry{{af··skipti}}{{\textipa{[{a}{f}{s}{\r{\textObardotlessj}}{\textsci}{f}{\textsubring{d}}{\textsci}]}}{\color{darkgreen}{\small{ n}}}{\small{ pl}}{\textit{ (þátttaka)}}\foreignlanguage{czech}{{ zapojení, angažování, účastnění}} $\triangleright$ {\textit{\textbf{ hafa engin afskipti af málinu
}}}}
\entry{{af··skipt|ur}}{{\textipa{[{a}{f}{s}{\r{\textObardotlessj}}{\textsci}{f}{\textsubring{d}}{\textscy}{r}]}}{\color{darkgreen}{\small{ adj}}}\foreignlanguage{czech}{{ opomíjený, přehlížený}}}
\entry{{af··skrif|a}}{{\textipa{[{a}{f}{s}{\r{g}}{r}{\textsci}{v}{a}]}}{\color{darkgreen}{\small{ v}}}{\small{ (-aði)}}{\small{ acc}}{ 1.}{\textit{ (afrita)}}\foreignlanguage{czech}{{ kopírovat, přepsat}}{ 2.}{\textit{ (fyrna)}}\foreignlanguage{czech}{{\footnotesize{ ekon.}}}\foreignlanguage{czech}{{ znehodnotit, devalvovat}},{ 3.}\foreignlanguage{czech}{{ odepsat (vyloučit ze seznamu ap.)}},}
\entry{{af··skrift}}{{\textipa{[{a}{f}{s}{\r{g}}{r}{\textsci}{f}{\textsubring{d}}]}}{\color{darkgreen}{\small{ f}}}{\small{ (-ar, -ir)}}{ 1.}\foreignlanguage{czech}{{ kopie, přepis}}{ 2.}\foreignlanguage{czech}{{\footnotesize{ ekon.}}}\foreignlanguage{czech}{{ odpis, amortizace}},}
\entry{{af··skræm|a}}{{\textipa{[{a}{f}{s}{\r{g}}{r}{a}{i}{m}{a}]}}{\color{darkgreen}{\small{ v}}}{\small{ (-di, -t)}}{\small{ acc}}\foreignlanguage{czech}{{ znetvořit, zohyzdit, zohavit}} $\triangleright$ {\textit{\textbf{ afskræma listaverkið}}}}
\entry{{af··skræmi}}{{\textipa{[{a}{f}{s}{\r{g}}{r}{a}{i}{m}{\textsci}]}}{\color{darkgreen}{\small{ n}}}{\small{ ( -s, -)}}{ 1.}{\textit{ (\textsuperscript{1}herfa)}}\foreignlanguage{czech}{{ ošklivý obraz nebo člověk}}{ 2.}{\textit{ (skrípamynd)}}\foreignlanguage{czech}{{ karikatura}},}
\entry{{af··skræming}}{{\textipa{[{a}{f}{s}{\r{g}}{r}{a}{i}{m}{i}{}{\r{g}}]}}{\color{darkgreen}{\small{ f}}}{\small{ (-ar, -ar)}}{\textit{ (afmyndun)}}\foreignlanguage{czech}{{ znetvoření, zohyzdění, zohavení}}}
\entry{{af··|slappaður}}{{\textipa{[{a}{f}{s}{\textsubring{d}}{l}{a}{h}{\textsubring{b}}{a}{ð}{\textscy}{r}]}}{\color{darkgreen}{\small{ adj}}}{\small{ (f -slöppuð)}}\foreignlanguage{czech}{{\footnotesize{ neform.}}}\foreignlanguage{czech}{{ uvolněný, v pohodě}}}
\entry{{af··slátt|ur}}{{\textipa{[{a}{f}{s}{\textsubring{d}}{l}{au}{h}{\textsubring{d}}{\textscy}{r}]}}{\color{darkgreen}{\small{ m}}}{\small{ (-ar)}}\foreignlanguage{czech}{{\footnotesize{ ekon}}}\foreignlanguage{czech}{{ sleva}} $\triangleright$ {\textit{\textbf{ kaupa e-ð með afslætti}}}}
\entry{{af··slöpp|un}}{{\textipa{[{a}{f}{s}{\textsubring{d}}{l}{\ae}{h}{\textsubring{b}}{\textscy}{n}]}}{\color{darkgreen}{\small{ f}}}{\small{ (-ar)}}\foreignlanguage{czech}{{\footnotesize{ neform.}}}\foreignlanguage{czech}{{ uvolnění, odpočinek, relaxace}}}
\entry{{af··sprengi}}{{\textipa{[{a}{f}{s}{\textsubring{b}}{r}{ei}{\textltailn}{\r{\textObardotlessj}}{\textsci}]}}{\color{darkgreen}{\small{ n}}}{\small{ ( -s, -)}}{\textit{ (afkvæmi)}}\foreignlanguage{czech}{{ potomek}}}
\entry{{af··spurn}}{{\textipa{[{a}{f}{s}{\textsubring{b}}{\textscy}{r}{\textsubring{d}}{\textsubring{n}}]}}{\color{darkgreen}{\small{ f}}}{\small{ (-ar)}}{\textit{ (umtal)}}\foreignlanguage{czech}{{ pověst, doslech, povídá se}}{\textsl{\textbf{ þekkja e-n af afspurn}}}\foreignlanguage{czech}{{ znát (koho) z doslechu}},}
\entry{{af··|staða}}{{\textipa{[{a}{f}{s}{\textsubring{d}}{a}{ð}{a}]}}{\color{darkgreen}{\small{ f}}}{\small{ (-stöðu)}}{ 1.}{\textit{ (viðhorf)}}\foreignlanguage{czech}{{ postoj}} $\triangleright$ {\textit{\textbf{ afstaða til hugmyndar hennar}}}{\textit{\foreignlanguage{czech}{ postoj k jejím myšlenkám}}}{ 2.}{\textit{ (lega)}}\foreignlanguage{czech}{{ poloha, umístění}} $\triangleright$ {\textit{\textbf{ innbyrðis afstaða landanna}}}{\textit{\foreignlanguage{czech}{ vzájemná poloha zemí}}},{\textsl{\textbf{ taka afstöðu til e-s}}}\foreignlanguage{czech}{{ zaujmout k (vůči) (čemu) postoj}},}
\entry{{afstrakt}}{{\textipa{[{a}{f}{s}{\textsubring{d}}{r}{a}{x}{\textsubring{d}}]}}{\color{darkgreen}{\small{ adj}}}{\small{ indecl}}\foreignlanguage{czech}{{ abstraktní}}}
\entry{{afstrakt··list}}{{\textipa{[{a}{f}{s}{\textsubring{d}}{r}{a}{x}{\textsubring{d}}{l}{\textsci}{s}{\textsubring{d}}]}}{\color{darkgreen}{\small{ f}}}{\small{ (-ar)}}\foreignlanguage{czech}{{ abstraktní umění}}}
\entry{{af··stýr|a}}{{\textipa{[{a}{f}{s}{\textsubring{d}}{i}{r}{a}]}}{\color{darkgreen}{\small{ v}}}{\small{ (-ði, -t)}}{\small{ dat}}\foreignlanguage{czech}{{ předcházet, předejít, zabránit}} $\triangleright$ {\textit{\textbf{ afstýra slysi}}}}
\entry{{af··stæða}}{{\textipa{[{a}{f}{s}{\textsubring{d}}{a}{i}{ð}{a}]}}{\color{darkgreen}{\small{ f}}}{\small{ (-u, -ur)}}\foreignlanguage{czech}{{ ???}}}
\entry{{af··stæði}}{{\textipa{[{a}{f}{s}{\textsubring{d}}{a}{i}{ð}{\textsci}]}}{\color{darkgreen}{\small{ n}}}{\small{ (-s)}}\foreignlanguage{czech}{{ relativita}}}
\entry{{af·stæðis··kenning}}{{\textipa{[{a}{f}{s}{\textsubring{d}}{a}{i}{ð}{\textsci}{s}{c\textsuperscript{h}}{\textepsilon}{n}{i}{}{\r{g}}]}}{\color{darkgreen}{\small{ f}}}{\small{ (-ar)}}\foreignlanguage{czech}{{\footnotesize{ fyz.}}}\foreignlanguage{czech}{{ teorie relativity}}}
\entry{{af··stæður}}{{\textipa{[{a}{f}{s}{\textsubring{d}}{a}{i}{ð}{\textscy}{r}]}}{\color{darkgreen}{\small{ adj}}}{\textit{ (viðmiðaður)}}\foreignlanguage{czech}{{ relativní}}}
\entry{{af··svar}}{{\textipa{[{a}{f}{s}{v}{a}{r}]}}{\color{darkgreen}{\small{ n}}}{\small{ (-s)}}{\textit{ (neitun)}}\foreignlanguage{czech}{{ odmítnutí, odepření, odřeknutí}} $\triangleright$ {\textit{\textbf{ veita honum afsvar við beiðninni}}}}
\entry{{af··|sögn}}{{\textipa{[{a}{f}{s}{\ae}{\r{g}}{\textsubring{n}}]}}{\color{darkgreen}{\small{ f}}}{\small{ (-sagnar, -sagnir)}}{ 1.}{\textit{ (neitun)}}\foreignlanguage{czech}{{ odmítnutí, neuznání}}{ 2.}\foreignlanguage{czech}{{ abdikace, vzdání se (úřadu ap.)}} $\triangleright$ {\textit{\textbf{ afsögn starfs}}},}
\entry{{af··|sökun}}{{\textipa{[{a}{f}{s}{\ae}{\r{g}}{\textscy}{n}]}}{\color{darkgreen}{\small{ f}}}{\small{ (-sökunar, -sakanir)}}{\textit{ (fyrirgefning)}}\foreignlanguage{czech}{{ omluva}} $\triangleright$ {\textit{\textbf{ biðjast kurteislega afsökunar á gerðum sínum}}}{\textit{\foreignlanguage{czech}{ omluvit se zdvořile za své činy}}}{\textsl{\textbf{ biðjast afsökunar}}}{\footnotesize{ refl}}\foreignlanguage{czech}{{ omluvit se}} $\triangleright$ {\textit{\textbf{ Hann baðst afsökunar.}}}{\textit{\foreignlanguage{czech}{ Omluvil se.}}},}
\entry{{af··|sönnun}}{{\textipa{[{a}{f}{s}{\ae}{n}{\textscy}{n}]}}{\color{darkgreen}{\small{ f}}}{\small{ (-sönnunar, -sannanir)}}\foreignlanguage{czech}{{ popření, vyvrácení}}}
\entry{{\textsuperscript{1}}{af··|taka}}{{\textipa{[{a}{f}{t\textsuperscript{h}}{a}{\r{g}}{a}]}}{\color{darkgreen}{\small{ f}}}{\small{ (-töku, -tökur)}}{\textit{ (líflát)}}\foreignlanguage{czech}{{ poprava}} $\triangleright$ {\textit{\textbf{ aftakan fór fram við sólarupprás.}}}}
\entry{{\textsuperscript{2}}{af··|taka}}{{\textipa{[{a}{f}{t\textsuperscript{h}}{a}{\r{g}}{a}]}}{\color{darkgreen}{\small{ v}}}{\small{ (-tek, -tók, -tókum, -tekið)}}{\small{ acc}}{ 1.}{\textit{ (afnema)}}\foreignlanguage{czech}{{ zrušit, prohlásit za neplatné}} $\triangleright$ {\textit{\textbf{ aftaka lög}}}{ 2.}{\textit{ (þverneita)}}\foreignlanguage{czech}{{ (rázně) popřít, zapřít}} $\triangleright$ {\textit{\textbf{ aftaka með öllu að}}},{ 3.}{\textit{ (taka af lífi)}}\foreignlanguage{czech}{{\footnotesize{ zast.}}}\foreignlanguage{czech}{{ popravit}} $\triangleright$ {\textit{\textbf{ aftaka e-n}}},}
\entry{{aftan}}{{\textipa{[{a}{f}{\textsubring{d}}{a}{n}]}}{\color{darkgreen}{\small{ prep/ adv}}}{\small{ gen}}{ 1.}{\textit{ (að baki)}}\foreignlanguage{czech}{{ za (čím) (místně)}}{ 2.}{\textit{ (á eftir e-u)}}\foreignlanguage{czech}{{ za (čím) (časově), po (čem)}},{ 3.}{\textit{ (að aftanverðu)}}\foreignlanguage{czech}{{ zezadu, vzadu, na zadní části (čeho)}},{\textsl{\textbf{ fyrir aftan, aftan við}}}{\footnotesize{ acc}}\foreignlanguage{czech}{{ (vzadu) za}} $\triangleright$ {\textit{\textbf{ Fyrir aftan húsið er bílskúr.}}}{\textit{\foreignlanguage{czech}{ (Vzadu) za domem je garáž.}}},}
\entry{{aftan··verður}}{{\textipa{[{a}{f}{\textsubring{d}}{a}{n}{}{s}{u}{p}{}{v}{}{}{s}{u}{p}{}{v}{\textepsilon}{r}{ð}{\textscy}{r}]}}{\color{darkgreen}{\small{ adj}}}\foreignlanguage{czech}{{ týkající se zadní části, zadní}}}
\entry{{aftari}}{{\textipa{[{a}{f}{\textsubring{d}}{a}{r}{\textsci}]}}{\color{darkgreen}{\small{ adj}}}{\small{ comp}}\foreignlanguage{czech}{{ ten druhý, druhý v pořadí (ze dvou)}}{\textit{ (fremri)}}}
\entry{{af··teng|ja}}{{\textipa{[{a}{f}{t\textsuperscript{h}}{ei}{\textltailn}{\r{\textObardotlessj}}{a}]}}{\color{darkgreen}{\small{ v}}}{\small{ (-di, -t)}}{\small{ acc}}\foreignlanguage{czech}{{ odpojit}}}
\entry{{aftr|a}}{{\textipa{[{a}{f}{\textsubring{d}}{r}{a}]}}{\color{darkgreen}{\small{ v}}}{\small{ (-aði)}}{\small{ dat}}{\textit{ (hindra)}}\foreignlanguage{czech}{{ (za)bránit, odradit, zastavit}} $\triangleright$ {\textit{\textbf{ Það getur ekkert aftrað mér frá því að fara í ferðina.}}}}
\entry{{aftur}}{{\textipa{[{a}{f}{\textsubring{d}}{\textscy}{r}]}}{\color{darkgreen}{\small{ adv}}}{ 1.}\foreignlanguage{czech}{{ zpátky, nazpátek, zpět}}{\textsl{\textbf{ ganga aftur á bak}}}\foreignlanguage{czech}{{ jít po zpátku}},{\textsl{\textbf{ líta aftur}}}\foreignlanguage{czech}{{ podívat se nazpátek}},{\textsl{\textbf{ ganga fram og aftur}}}\foreignlanguage{czech}{{ chodit sem a tam}},{\textsl{\textbf{ láta e-ð aftur}}}\foreignlanguage{czech}{{ dát (co) nazpátek}},{ 2.}{\textit{ (á ný)}}\foreignlanguage{czech}{{ znovu, zas(e)}} $\triangleright$ {\textit{\textbf{ aftur og aftur}}}{\textit{\foreignlanguage{czech}{ zas a zas}}},{ 3.}{\textit{ (í staðinn)}}\foreignlanguage{czech}{{ na místo (čeho), v náhradě za (co)}} $\triangleright$ {\textit{\textbf{ bæta e-m aftur tjón hans}}},{ 4.}{\footnotesize {\foreignlanguage{czech}{ (o něčem (polo)zapomenutém)}}},{\textsl{\textbf{ Hvað heitir hann nú aftur?}}}\foreignlanguage{czech}{{ Jak že se jmenuje? Jak se vlastně jmenuje?}},{ 5.}{\footnotesize {\foreignlanguage{czech}{ (o zavírání)}}},{\textsl{\textbf{ Hurðin er aftur.}}}\foreignlanguage{czech}{{ Dveře jsou zavřené.}},{\textsl{\textbf{ Láttu aftur hurðina!}}}\foreignlanguage{czech}{{ Zavři dveře!}},{ 6.}{\footnotesize {\foreignlanguage{czech}{ (s různými prep)}}},{\textsl{\textbf{ fara aftur í}}}\foreignlanguage{czech}{{ jet vzadu (na zadním sedadle)}},{\textsl{\textbf{ verða aftur úr}}}\foreignlanguage{czech}{{ nestíhat, být pozadu}},{\textsl{\textbf{ aftur úr því}}}\foreignlanguage{czech}{{ po (čem) (časově)}},}
\entry{{aftur··beygður}}{{\textipa{[{a}{f}{\textsubring{d}}{\textscy}{r}{\textsubring{b}}{ei}{\textbabygamma}{ð}{\textscy}{r}]}}{\color{darkgreen}{\small{ adj}}}\foreignlanguage{czech}{{ zvratný}}{\textsl{\textbf{ afturbeygð sögn}}}\foreignlanguage{czech}{{\footnotesize{ jaz.}}}\foreignlanguage{czech}{{ zvratné sloveso}} $\triangleright$ {\textit{\textbf{ (klæða sig)}}},{\textsl{\textbf{ afturbeygt fornafn}}}\foreignlanguage{czech}{{\footnotesize{ jaz.}}}\foreignlanguage{czech}{{ zvratné zájmeno}} $\triangleright$ {\textit{\textbf{ (sig)}}},{\textsl{\textbf{ afturbeygt eignarfornafn}}}\foreignlanguage{czech}{{\footnotesize{ jaz.}}}\foreignlanguage{czech}{{ zvratné zájmeno přivlastňovací}} $\triangleright$ {\textit{\textbf{ (sinn)}}},}
\entry{{aftur··|fótur}}{{\textipa{[{a}{f}{\textsubring{d}}{\textscy}{\textsubring{r}}{f}{ou}{\textsubring{d}}{\textscy}{r}]}}{\color{darkgreen}{\small{ m}}}{\small{ (-fótar, -fætur)}}\foreignlanguage{czech}{{ zadní nohy (zvláště u čtyřnohých živočichů)}}{\textsl{\textbf{ það kemur á afturfótunum út úr e-m}}}\foreignlanguage{czech}{{ leze to z (koho) jako z chlupaté deky}},}
\entry{{aftur··|för}}{{\textipa{[{a}{f}{\textsubring{d}}{\textscy}{\textsubring{r}}{f}{\ae}{r}]}}{\color{darkgreen}{\small{ f}}}{\small{ (-farar)}}{ 1.}{\textit{ (afturferð)}}\foreignlanguage{czech}{{ cesta zpět, zpáteční cesta}}{ 2.}{\textit{ (hnignun)}}\foreignlanguage{czech}{{ pokles, úpadek, sestup; zhoršení (zdraví ap.)}},}
\entry{{aftur··|ganga}}{{\textipa{[{a}{f}{\textsubring{d}}{\textscy}{r}{\r{g}}{au}{}{\r{g}}{a}]}}{\color{darkgreen}{\small{ f}}}{\small{ (-göngu, -göngur)}}{\textit{ (draugur)}}\foreignlanguage{czech}{{\footnotesize{ pov.}}}\foreignlanguage{czech}{{ duch, zjevení}} $\triangleright$ {\textit{\textbf{ Þjóðsögur segja frá afturgöngum.}}}{\textit{\foreignlanguage{czech}{ Lidové pověsti vyprávějí o duchách.}}}}
\entry{{aftur··hlut|i}}{{\textipa{[{a}{f}{\textsubring{d}}{\textscy}{\textsubring{r}}{\textsubring{l}}{\textscy}{\textsubring{d}}{\textsci}]}}{\color{darkgreen}{\small{ m}}}{\small{ (-a, -ar)}}{ 1.}\foreignlanguage{czech}{{ zadní část}}{ 2.}\foreignlanguage{czech}{{ zadek (u čtyřnohého zvířete včetně dvou zadních nohou)}},}
\entry{{aftur··hvarf}}{{\textipa{[{a}{f}{\textsubring{d}}{\textscy}{r}{k\textsuperscript{h}}{v}{a}{r}{v}]}}{\color{darkgreen}{\small{ n}}}{\small{ (-s)}}\foreignlanguage{czech}{{ obrat (v názoru ap.), návrat (k víře ap.)}}}
\entry{{aftur··kall|a}}{{\textipa{[{a}{f}{\textsubring{d}}{\textscy}{\textsubring{r}}{k\textsuperscript{h}}{a}{\textsubring{d}}{l}{a}]}}{\color{darkgreen}{\small{ v}}}{\small{ (-aði)}}{\small{ acc}}{\textit{ (rifta)}}\foreignlanguage{czech}{{ zrušit, odvolat, anulovat}} $\triangleright$ {\textit{\textbf{ afturkalla ákvörðun sína}}}{\textit{\foreignlanguage{czech}{ zrušit své rozhodnutí}}}}
\entry{{aftur··kipp|ur}}{{\textipa{[{a}{f}{\textsubring{d}}{\textscy}{\textsubring{r}}{c\textsuperscript{h}}{\textsci}{h}{\textsubring{b}}{\textscy}{r}]}}{\color{darkgreen}{\small{ m}}}{\small{ (-s)}}{ 1.}\foreignlanguage{czech}{{ zpětné trhnutí, škubnutí zpět}}{ 2.}{\textit{ (stöðvun)}}\foreignlanguage{czech}{{ zastavení, přerušení}} $\triangleright$ {\textit{\textbf{ afturkippur vexti}}},{ 3.}{\textit{ (hik)}}\foreignlanguage{czech}{{ (za)váhání, rozpaky}} $\triangleright$ {\textit{\textbf{ Þá kom afturkippur á mig.}}},}
\entry{{aftur··kræfur}}{{\textipa{[{a}{f}{\textsubring{d}}{\textscy}{\textsubring{r}}{k\textsuperscript{h}}{r}{a}{i}{v}{\textscy}{r}]}}{\color{darkgreen}{\small{ adj}}}\foreignlanguage{czech}{{ reklamovatelný}}}
\entry{{aftur··|köllun}}{{\textipa{[{a}{f}{\textsubring{d}}{\textscy}{\textsubring{r}}{k\textsuperscript{h}}{\ae}{\textsubring{d}}{l}{\textscy}{n}]}}{\color{darkgreen}{\small{ f}}}{\small{ (-köllunar, -kallanir)}}\foreignlanguage{czech}{{ odvolání, zrušení}} $\triangleright$ {\textit{\textbf{ afturköllun leyfis}}}}
\entry{{aftur··ljós}}{{\textipa{[{a}{f}{\textsubring{d}}{\textscy}{r}{l}{j}{ou}{s}]}}{\color{darkgreen}{\small{ n}}}{\small{ (-s, -)}}\foreignlanguage{czech}{{ zadní světla}}}
\entry{{aftur··reka}}{{\textipa{[{a}{f}{\textsubring{d}}{\textscy}{r}{\textepsilon}{\r{g}}{a}]}}{\color{darkgreen}{\small{ adj}}}{\small{ indecl}}{\textsl{\textbf{ gera e-n afturreka}}}\foreignlanguage{czech}{{ zamítnout (komu) přístup, poslat (koho) pryč bez vyslyšení}}}
\entry{{aftur··sæti}}{{\textipa{[{a}{f}{\textsubring{d}}{\textscy}{\textsubring{r}}{s}{a}{i}{\textsubring{d}}{\textsci}]}}{\color{darkgreen}{\small{ n}}}{\small{ ( -s, -)}}\foreignlanguage{czech}{{ zadní sedadlo}}}
\entry{{aftur··virkur}}{{\textipa{[{a}{f}{\textsubring{d}}{\textscy}{r}{v}{\textsci}{\textsubring{r}}{\r{g}}{\textscy}{r}]}}{\color{darkgreen}{\small{ adj}}}\foreignlanguage{czech}{{ retroaktivní, mající zpětný účinek}}}
\entry{{af··undinn}}{{\textipa{[{a}{\textlengthmark}{f}{\textscy}{n}{\textsubring{d}}{\textsci}{n}]}}{\color{darkgreen}{\small{ adj}}}{\textit{ (önugur)}}\foreignlanguage{czech}{{ mrzutý, nevrlý}}}
\entry{{afurð}}{{\textipa{[{a}{\textlengthmark}{v}{\textscy}{r}{ð}]}}{\color{darkgreen}{\small{ f}}}{\small{ (-ar, -ir)}}{\textit{ (framleiðsla)}}\foreignlanguage{czech}{{ výrobek, (zemědělský) produkt}} $\triangleright$ {\textit{\textbf{ framleiða verðmætar afurðir úr timbrinu}}}}
\entry{{afurða··lán}}{{\textipa{[{a}{\textlengthmark}{v}{\textscy}{r}{ð}{a}{l}{au}{n}]}}{\color{darkgreen}{\small{ n}}}{\small{ ( -s, -)}}\foreignlanguage{czech}{{ ??}}}
\entry{{af·vega··leiddur}}{{\textipa{[{a}{f}{v}{\textlengthmark}{\textepsilon}{\textbabygamma}{a}{l}{ei}{\textsubring{d}}{\textscy}{r}]}}{\color{darkgreen}{\small{ adj}}}\foreignlanguage{czech}{{ zcestný, zavádějící; zvrácený}} $\triangleright$ {\textit{\textbf{ afvegaleiddar stúlkur}}}}
\entry{{af·vega··lei|ða}}{{\textipa{[{a}{f}{v}{\textlengthmark}{\textepsilon}{\textbabygamma}{a}{l}{ei}{ð}{a}]}}{\color{darkgreen}{\small{ v}}}{\small{ (-ddi, -tt)}}{\small{ acc}}\foreignlanguage{czech}{{ zavést na zcestí, uvést v omyl, svést na nesprávnou cestu}}}
\entry{{af··vikinn}}{{\textipa{[{a}{f}{v}{\textlengthmark}{\textsci}{\r{\textObardotlessj}}{\textsci}{n}]}}{\color{darkgreen}{\small{ adj}}}\foreignlanguage{czech}{{ odlehlý, vzdálený, izolovaný (o místě ap.)}}}
\entry{{af··vopn|a}}{{\textipa{[{a}{f}{v}{\textlengthmark}{\textopeno}{h}{\textsubring{b}}{n}{a}]}}{\color{darkgreen}{\small{ v}}}{\small{ (-aði)}}\foreignlanguage{czech}{{ odzbrojit}} $\triangleright$ {\textit{\textbf{ afvopna hann}}}{\textit{\foreignlanguage{czech}{ odzbrojit ho}}}}
\entry{{af··vopnun}}{{\textipa{[{a}{f}{v}{\textlengthmark}{\textopeno}{h}{\textsubring{b}}{n}{\textscy}{n}]}}{\color{darkgreen}{\small{ f}}}{\small{ (-ar)}}\foreignlanguage{czech}{{ odzbrojení}}}
\entry{{af··vötnun}}{{\textipa{[{a}{f}{v}{\textlengthmark}{\ae}{h}{\textsubring{d}}{n}{\textscy}{n}]}}{\color{darkgreen}{\small{ f}}}{\small{ (-ar)}}\foreignlanguage{czech}{{ odvodnění, vysušení (ryby ap.)}}}
\entry{{af··þakk|a}}{{\textipa{[{a}{f}{\texttheta}{a}{h}{\r{g}}{a}]}}{\color{darkgreen}{\small{ v}}}{\small{ (-aði)}}{\small{ acc}}{\textit{ (hafna)}}\foreignlanguage{czech}{{ odmítnout, nepřijmout}} $\triangleright$ {\textit{\textbf{ afþakka gjöfina kurteislega}}}}
\entry{{af··þreying}}{{\textipa{[{a}{f}{\texttheta}{r}{ei}{i}{}{\r{g}}]}}{\color{darkgreen}{\small{ f}}}{\small{ (-ar)}}\foreignlanguage{czech}{{ pobavení, zábava}} $\triangleright$ {\textit{\textbf{ horfa á sjónvarpið sér til afþreyingar}}}}
\entry{{af··æt|a}}{{\textipa{[{a}{\textlengthmark}{f}{a}{i}{\textsubring{d}}{a}]}}{\color{darkgreen}{\small{ f}}}{\small{ (-u, -ur)}}\foreignlanguage{czech}{{ parazit}}}
\entry{{ag|a}}{{\textipa{[{a}{\textlengthmark}{\textbabygamma}{a}]}}{\color{darkgreen}{\small{ v}}}{\small{ (-aði)}}{\small{ acc}}{\textit{ (temja)}}\foreignlanguage{czech}{{ ukáznit, potrestat (disciplinárně)}} $\triangleright$ {\textit{\textbf{ aga börnin}}}}
\entry{{aga··laus}}{{\textipa{[{a}{\textlengthmark}{\textbabygamma}{a}{l}{\ae i}{s}]}}{\color{darkgreen}{\small{ adj}}}\foreignlanguage{czech}{{ neukázněný, nevychovaný}}}
\entry{{aga··legur}}{{\textipa{[{a}{\textlengthmark}{\textbabygamma}{a}{l}{\textepsilon}{\textbabygamma}{\textscy}{r}]}}{\color{darkgreen}{\small{ adj}}}{\textit{ (hrikalegur)}}\foreignlanguage{czech}{{ příšerný, hrozný}}}
\entry{{aga··leysi}}{{\textipa{[{a}{\textlengthmark}{\textbabygamma}{a}{l}{ei}{s}{\textsci}]}}{\color{darkgreen}{\small{ n}}}{\small{ (-s)}}\foreignlanguage{czech}{{ neukázněnost, nekázeň}}}
\entry{{ag|i}}{{\textipa{[{a}{i}{j}{\textlengthmark}{\textsci}]}}{\color{darkgreen}{\small{ m}}}{\small{ (-a)}}\foreignlanguage{czech}{{ kázeň, ukázněnost, disciplína}} $\triangleright$ {\textit{\textbf{ reyna að halda aga}}}}
\entry{{agn}}{{\textipa{[{a}{\r{g}}{\textsubring{n}}]}}{\color{darkgreen}{\small{ n}}}{\small{ (-s)}}{\textit{ (beita)}}\foreignlanguage{czech}{{ návnada, vnadidlo}}}
\entry{{agn··dofa}}{{\textipa{[{a}{\r{g}}{\textsubring{n}}{\textsubring{d}}{\textopeno}{v}{a}]}}{\color{darkgreen}{\small{ adj}}}{\small{ indecl}}{\textit{ (forviða)}}\foreignlanguage{czech}{{ ohromený, vyjevený}}}
\entry{{agn··hnú|i}}{{\textipa{[{a}{\r{g}}{\textsubring{n}}{\textsubring{n}}{u}{\textsci}]}}{\color{darkgreen}{\small{ m}}}{\small{ (-a, -ar)}} $\rightarrow$       agnúi}
\entry{{agnú|i}}{{\textipa{[{a}{\r{g}}{n}{u}{\textsci}]}}{\color{darkgreen}{\small{ m}}}{\small{ (-a, -ar)}}{ 1.}\foreignlanguage{czech}{{ úlovek na háčku (rybaření)}}{ 2.}{\textit{ (smástrákur)}}\foreignlanguage{czech}{{ malý kluk}},{ 3.}{\textit{ (tormerki)}}\foreignlanguage{czech}{{ překážky, potíže}},{ 4.}{\textit{ (ímugustur)}}\foreignlanguage{czech}{{ odpor, nechuť}} $\triangleright$ {\textit{\textbf{ hafa agnúa á e-m}}},}
\entry{{agúrk|a}}{{\textipa{[{a}{\textlengthmark}{\textbabygamma}{u}{\textsubring{r}}{\r{g}}{a}]}}{\color{darkgreen}{\small{ f}}}{\small{ (-u, -ur)}}\foreignlanguage{czech}{{\footnotesize{ bot.}}}\foreignlanguage{czech}{{ okurka}}{\textit{ Cucumis sativus}}\par\begin{center}\setlength\fboxsep{0pt}\setlength\fboxrule{0.5pt}\fbox{\includegraphics[height=6cm]{ds_image_agurka_0_2.jpg}}\end{center}\par\begin{center}\footnotesize {Autor: Stephen Ausmus Licence: Public Domain}\end{center}}
\entry{{aka}}{{\textipa{[{a}{\textlengthmark}{\r{g}}{a}]}}{\color{darkgreen}{\small{ v}}}{\small{ (ek, ók, ókum, ekið)}}{\small{ dat}}{ 1.}{\textit{ (keyra)}}\foreignlanguage{czech}{{ řídit (auto ap.)}} $\triangleright$ {\textit{\textbf{ Hann lærði ungur að aka dráttarvél.}}}{\textit{\foreignlanguage{czech}{ Naučil se řídit traktor, když byl mladý.}}}{ 2.}{\textit{ (róta)}}\foreignlanguage{czech}{{ shromažďovat, hromadit (peníze ap.)}} $\triangleright$ {\textit{\textbf{ aka saman peningum}}}{\textit{\foreignlanguage{czech}{ hromadit peníze}}},{ 3.}{\textit{ (hreyfa hægt)}}\foreignlanguage{czech}{{ (pomalu) hýbat se}} $\triangleright$ {\textit{\textbf{ aka sér í herðunum}}},}
\entry{{ak··braut}}{{\textipa{[{a}{\textlengthmark}{\r{g}}{\textsubring{b}}{r}{\ae i}{\textsubring{d}}]}}{\color{darkgreen}{\small{ f}}}{\small{ (-ar, -ir)}}\foreignlanguage{czech}{{ vozovka}}}
\entry{{akkeri}}{{\textipa{[{a}{h}{\r{\textObardotlessj}}{\textepsilon}{r}{\textsci}]}}{\color{darkgreen}{\small{ n}}}{\small{ (-s, -)}}\foreignlanguage{czech}{{\footnotesize{ nám.}}}\foreignlanguage{czech}{{ kotva}} $\triangleright$ {\textit{\textbf{ leggjast við akkeri}}}}
\entry{{akkorð}}{{\textipa{[{a}{h}{\r{g}}{\textopeno}{r}{ð}]}}{\color{darkgreen}{\small{ n}}}{\small{ (-s)}}{\textit{ (ákvæðisvinna)}}\foreignlanguage{czech}{{ úkolová práce}} $\triangleright$ {\textit{\textbf{ vinna upp á akkorð}}}}
\entry{{akkúrat}}{{\textipa{[{a}{h}{\r{g}}{u}{r}{a}{\textsubring{d}}]}}{\color{darkgreen}{\small{ adj}}}{\small{ indecl}}\foreignlanguage{czech}{{ přesný, doslovný}}}
\entry{{ak··rein}}{{\textipa{[{a}{\textlengthmark}{\r{g}}{r}{ei}{n}]}}{\color{darkgreen}{\small{ f}}}{\small{ (-ar, -ar)}}\foreignlanguage{czech}{{ jízdní pruh}}}
\entry{{akstur}}{{\textipa{[{a}{x}{s}{\textsubring{d}}{\textscy}{r}]}}{\color{darkgreen}{\small{ m}}}{\small{ (-s)}}\foreignlanguage{czech}{{ jízda (autem ap.)}}}
\entry{{ak|ur}}{{\textipa{[{a}{\textlengthmark}{\r{g}}{\textscy}{r}]}}{\color{darkgreen}{\small{ m}}}{\small{ (-urs, -rar)}}{\textit{ (ekra)}}\foreignlanguage{czech}{{ pole}}}
\entry{{Akur··eyr|i}}{{\textipa{[{a}{\textlengthmark}{\r{g}}{\textscy}{r}{ei}{r}{\textsci}]}}{\color{darkgreen}{\small{ f}}}{\small{ (-ar)}}\foreignlanguage{czech}{{\footnotesize{ geog.}}}\foreignlanguage{czech}{{ Akureyri}}\par\begin{center}\setlength\fboxsep{0pt}\setlength\fboxrule{0.5pt}\fbox{\includegraphics[width=6cm]{ds_image_akureyri_0_1.jpg}}\end{center}\par\begin{center}\footnotesize {Autor: hvalur.org Licence: CC Unported Licence}\end{center}}
\entry{{akur··hæn|a}}{{\textipa{[{a}{\textlengthmark}{\r{g}}{\textscy}{r}{h}{a}{i}{n}{a}]}}{\color{darkgreen}{\small{ f}}}{\small{ (-u, -ur)}}\foreignlanguage{czech}{{\footnotesize{ zool.}}}\foreignlanguage{czech}{{ koroptev, koroptev polní}}{\textit{ Perdix perdix}}\par\begin{center}\setlength\fboxsep{0pt}\setlength\fboxrule{0.5pt}\fbox{\includegraphics[width=6cm]{117600.jpg}}\end{center}\par\begin{center}\footnotesize {Autor: Došlý Martin Licence: COPYRIGHT/PD}\end{center}}
\entry{{akur··yrkj|a}}{{\textipa{[{a}{\textlengthmark}{\r{g}}{\textscy}{r}{\textsci}{\textsubring{r}}{\r{\textObardotlessj}}{a}]}}{\color{darkgreen}{\small{ f}}}{\small{ (-u)}}{\textit{ (kornyrkja)}}\foreignlanguage{czech}{{\footnotesize{ zem.}}}\foreignlanguage{czech}{{ zemedělství}}}
\entry{{al-}}{{\color{darkgreen}{\small{ predp}}}\foreignlanguage{czech}{{ naprosto, úplně}}}
\entry{{ala}}{{\textipa{[{a}{\textlengthmark}{l}{a}]}}{\color{darkgreen}{\small{ v}}}{\small{ (el, ól, ólum, alið)}}{\small{ acc}}{ 1.}{\textit{ (fæða)}}\foreignlanguage{czech}{{ (po)rodit}} $\triangleright$ {\textit{\textbf{ ala barn}}}{\textit{\foreignlanguage{czech}{ porodit dítě}}}{\textit{ (\textsuperscript{1}fóstra)}}{\textsl{\textbf{ ala upp}}}\foreignlanguage{czech}{{ vychovávat}} $\triangleright$ {\textit{\textbf{ ala upp barn}}}{\textit{\foreignlanguage{czech}{ vychovávat dítě}}},{\textsl{\textbf{ alast upp}}}{\footnotesize{ refl}}\foreignlanguage{czech}{{ vyrůstat}} $\triangleright$ {\textit{\textbf{ alast upp í sveitinni}}}{\textit{\foreignlanguage{czech}{ vyrůstat na venkově}}},{ 2.}{\textit{ (næra á mat)}}\foreignlanguage{czech}{{ živit, krmit}} $\triangleright$ {\textit{\textbf{ ala á mjólk}}}{\textit{\foreignlanguage{czech}{ krmit mlékem}}},}
\entry{{albatros}}{{\textipa{[{a}{l}{\textsubring{b}}{a}{\textsubring{d}}{r}{\textopeno}{s}]}}{\color{darkgreen}{\small{ m}}}{\small{ (-s, -ar)}} $\rightarrow$       albatrosi}
\entry{{albatros|i}}{{\textipa{[{a}{l}{\textsubring{b}}{a}{\textsubring{d}}{r}{\textopeno}{s}{\textsci}]}}{\color{darkgreen}{\small{ m}}}{\small{ (-a, -ar)}}\foreignlanguage{czech}{{\footnotesize{ zool.}}}\foreignlanguage{czech}{{ albatros}}{\textit{ Diomedeidae}}\par\begin{center}\setlength\fboxsep{0pt}\setlength\fboxrule{0.5pt}\fbox{\includegraphics[width=6cm]{20754.jpg}}\end{center}\par\begin{center}\footnotesize {Autor: U.S. Fish and Wildlife Service Licence: PD}\end{center}}
\entry{{al··búinn}}{{\textipa{[{a}{\textsubring{l}}{\textsubring{b}}{u}{\textsci}{n}]}}{\color{darkgreen}{\small{ adj}}}{ 1.}\foreignlanguage{czech}{{ připravený, přichystaný, oblečený}}{ 2.}{\textit{ (reiðubúinn)}}\foreignlanguage{czech}{{ ochotný (k čemu)}},}
\entry{{albúm}}{{\textipa{[{a}{l}{\textsubring{b}}{u}{m}]}}{\color{darkgreen}{\small{ n}}}{\small{ (-s, -)}}\foreignlanguage{czech}{{ album}}}
\entry{{alda}}{{\textipa{[{a}{l}{\textsubring{d}}{a}]}}{\color{darkgreen}{\small{ f}}}{\small{ (öldu, öldur)}}{ 1.}{\textit{ (bylgja)}}\foreignlanguage{czech}{{ vlna (na vodě ap.)}} $\triangleright$ {\textit{\textbf{ Á sjónum eru stórar öldur.}}}{\textit{\foreignlanguage{czech}{ Na moři jsou velké vlny.}}}{ 2.}\foreignlanguage{czech}{{ nerovnost na zemi, zvlnění (kopce ap.)}},}
\entry{{alda··mót}}{{\textipa{[{a}{l}{\textsubring{d}}{a}{m}{ou}{\textsubring{d}}]}}{\color{darkgreen}{\small{ n}}}{\small{ pl}}\foreignlanguage{czech}{{ přelom století}} $\triangleright$ {\textit{\textbf{ Þetta gerðist um aldamótin 1900.}}}{\textit{\foreignlanguage{czech}{ Stalo se to na přelomu 19. století.}}}}
\entry{{aldar··afmæli}}{{\textipa{[{a}{l}{\textsubring{d}}{a}{r}{a}{v}{m}{a}{i}{l}{\textsci}]}}{\color{darkgreen}{\small{ n}}}{\small{ (-s, -)}}\foreignlanguage{czech}{{ výročí stých narozenin}}}
\entry{{aldeilis}}{{\textipa{[{a}{l}{\textsubring{d}}{ei}{l}{\textsci}{s}]}}{\color{darkgreen}{\small{ adv}}}{\textit{ (alveg)}}\foreignlanguage{czech}{{ naprosto, úplně}}}
\entry{{aldin}}{{\textipa{[{a}{l}{\textsubring{d}}{\textsci}{n}]}}{\color{darkgreen}{\small{ n}}}{\small{ (-s, -)}}\foreignlanguage{czech}{{\footnotesize{ bot.}}}\foreignlanguage{czech}{{ plod}} $\triangleright$ {\textit{\textbf{ Tréð ber aldin.}}}{\textit{\foreignlanguage{czech}{ Strom nosí ovoce.}}}}
\entry{{aldin··bori}}{{\textipa{[{a}{l}{\textsubring{d}}{\textsci}{n}{\textsubring{b}}{\textopeno}{r}{\textsci}]}}{\color{darkgreen}{\small{ m}}}{\small{ (-a, -ar)}}\foreignlanguage{czech}{{\footnotesize{ zool.}}}\foreignlanguage{czech}{{ chroust}}{\textit{ Melolontha melolontha}}\par\begin{center}\setlength\fboxsep{0pt}\setlength\fboxrule{0.5pt}\fbox{\includegraphics[width=6cm]{10048.jpg}}\end{center}\par\begin{center}\footnotesize {Autor: Krejčík Stanislav Licence: COPYRIGHT/CC-BY}\end{center}}
\entry{{aldin··garð|ur}}{{\textipa{[{a}{l}{\textsubring{d}}{\textsci}{n}{\r{g}}{a}{r}{ð}{\textscy}{r}]}}{\color{darkgreen}{\small{ m}}}{\small{ (-s, -ar)}}\foreignlanguage{czech}{{ ovocný sad}}}
\entry{{aldin··tré}}{{\textipa{[{a}{l}{\textsubring{d}}{\textsci}{n}{t\textsuperscript{h}}{r}{j}{\textepsilon}]}}{\color{darkgreen}{\small{ n}}}{\small{ (-s, -)}}\foreignlanguage{czech}{{\footnotesize{ bot.}}}\foreignlanguage{czech}{{ ovocný strom}}}
\entry{{aldraður}}{{\textipa{[{a}{l}{\textsubring{d}}{r}{a}{ð}{\textscy}{r}]}}{\color{darkgreen}{\small{ adj}}}{\small{ (f öldruð)}}{\textit{ (gamall)}}\foreignlanguage{czech}{{ starý, v letech}}}
\entry{{aldrei}}{{\textipa{[{a}{l}{\textsubring{d}}{r}{ei}]}}{\color{darkgreen}{\small{ adv}}}\foreignlanguage{czech}{{ nikdy}} $\triangleright$ {\textit{\textbf{ Óli kemur aldrei of seint.}}}{\textit{\foreignlanguage{czech}{ Oli nechodí nikdy pozdě.}}}{\textsl{\textbf{ það er aldrei}}}\foreignlanguage{czech}{{ to je vynikající}},}
\entry{{ald|ur}}{{\textipa{[{a}{l}{\textsubring{d}}{\textscy}{r}]}}{\color{darkgreen}{\small{ m}}}{\small{ (-urs, -rar)}}{ 1.}\foreignlanguage{czech}{{ věk (počet let)}} $\triangleright$ {\textit{\textbf{ reyna að ákvarða aldur hans}}}{\textsl{\textbf{ vera á aldur við e-n}}}\foreignlanguage{czech}{{ být stejně starý jako (kdo)}},{\textsl{\textbf{ vera á besta aldri}}}\foreignlanguage{czech}{{ být v nejlepších letech}},{ 2.}{\textit{ (elli)}}\foreignlanguage{czech}{{ stáří}} $\triangleright$ {\textit{\textbf{ vera við aldur}}},{ 3.}{\textit{ (tími)}}\foreignlanguage{czech}{{ čas}},{\textsl{\textbf{ um (í) langan aldur}}}\foreignlanguage{czech}{{ dlouho, dlouhý čas}},}
\entry{{aldurs··forset|i}}{{\textipa{[{a}{l}{\textsubring{d}}{\textscy}{\textsubring{r}}{s}{f}{\textopeno}{\textsubring{r}}{s}{\textepsilon}{\textsubring{d}}{\textsci}]}}{\color{darkgreen}{\small{ m}}}{\small{ (-a, -ar)}}\foreignlanguage{czech}{{ nejstarší člen (v organizaci ap.)}} $\triangleright$ {\textit{\textbf{ aldursforseti alþingis}}}}
\entry{{aldurs··hóp|ur}}{{\textipa{[{a}{l}{\textsubring{d}}{\textscy}{\textsubring{r}}{s}{h}{ou}{\textsubring{b}}{\textscy}{r}]}}{\color{darkgreen}{\small{ m}}}{\small{ (-s, -ar)}}{\textit{ (aldursflokkur)}}\foreignlanguage{czech}{{ věková skupina}}}
\entry{{aldurs··mun|ur}}{{\textipa{[{a}{l}{\textsubring{d}}{\textscy}{\textsubring{r}}{s}{m}{\textscy}{n}{\textscy}{r}]}}{\color{darkgreen}{\small{ m}}}{\small{ (-ar, -ir)}}\foreignlanguage{czech}{{ věkový rozdíl}}}
\entry{{aldurs··skeið}}{{\textipa{[{a}{l}{\textsubring{d}}{\textscy}{\textsubring{r}}{s}{\r{\textObardotlessj}}{ei}{\texttheta}]}}{\color{darkgreen}{\small{ n}}}{\small{ ( -s, -)}}\foreignlanguage{czech}{{ časové období určitého věku}}}
\entry{{aldurs··tak·|mark}}{{\textipa{[{a}{l}{\textsubring{d}}{\textscy}{\textsubring{r}}{s}{t\textsuperscript{h}}{a}{\r{g}}{m}{a}{\textsubring{r}}{\r{g}}]}}{\color{darkgreen}{\small{ n}}}{\small{ (-marks, -mörk)}}\foreignlanguage{czech}{{ věková hranice}}}
\entry{{al··efli}}{{\textipa{[{a}{\textlengthmark}{\textsubring{l}}{\textepsilon}{\textsubring{b}}{l}{\textsci}]}}{\color{darkgreen}{\small{ n}}}{\small{ (-s)}}{\textit{ (kraftur)}}\foreignlanguage{czech}{{ všechna síla}}{\textsl{\textbf{ af alefli}}}\foreignlanguage{czech}{{ ze všech sil}},}
\entry{{al··eig|a}}{{\textipa{[{a}{\textlengthmark}{\textsubring{l}}{ei}{\textbabygamma}{a}]}}{\color{darkgreen}{\small{ f}}}{\small{ (-u)}}\foreignlanguage{czech}{{ veškerý majetek}} $\triangleright$ {\textit{\textbf{ Þetta er aleiga mín.}}}}
\entry{{aleinn}}{{\textipa{[{a}{\textlengthmark}{l}{ei}{\textsubring{d}}{\textsubring{n}}]}}{\color{darkgreen}{\small{ adj}}}\foreignlanguage{czech}{{ samotný, sám}}}
\entry{{al··elda}}{{\textipa{[{a}{\textlengthmark}{\textsubring{l}}{\textepsilon}{l}{\textsubring{d}}{a}]}}{\color{darkgreen}{\small{ adj}}}{\small{ indecl}}\foreignlanguage{czech}{{ hořící, celý v ohni}} $\triangleright$ {\textit{\textbf{ Húsið var alelda.}}}}
\entry{{al··farið}}{{\textipa{[{a}{\textsubring{l}}{f}{a}{r}{\textsci}{\texttheta}]}}{\color{darkgreen}{\small{ adv}}}{ 1.}\foreignlanguage{czech}{{ nadobro, napořád}} $\triangleright$ {\textit{\textbf{ Hann er kominn alfarið.}}}{\textit{\foreignlanguage{czech}{ Nadobro odešel.}}}{ 2.}{\textit{ (alveg)}}\foreignlanguage{czech}{{ úplně, naprosto}},}
\entry{{al··farin|n}}{{\textipa{[{a}{\textsubring{l}}{f}{a}{r}{\textsci}{n}]}}{\color{darkgreen}{\small{ adj}}}{ 1.}\foreignlanguage{czech}{{ na dobro odešlý (člověk ap.)}}{ 2.}\foreignlanguage{czech}{{ (o cestě) často chozený (po které se často chodí)}},}
\entry{{al·fræði··|bók}}{{\textipa{[{a}{\textsubring{l}}{f}{r}{a}{i}{ð}{\textsci}{\textsubring{b}}{ou}{\r{g}}]}}{\color{darkgreen}{\small{ f}}}{\small{ (-bókar, -bækur)}}\foreignlanguage{czech}{{ encyklopedie}}}
\entry{{alg.}}{{\color{darkgreen}{\small{ zkr}}}{\textsl{\textbf{ algengur}}}\foreignlanguage{czech}{{ běžný}}}
\entry{{al··gáður}}{{\textipa{[{a}{\textsubring{l}}{\r{g}}{au}{ð}{\textscy}{r}]}}{\color{darkgreen}{\small{ adj}}}{\textit{ (ódrukkinn)}}\foreignlanguage{czech}{{ střízlivý, při plném vědomí}}}
\entry{{al··gengur}}{{\textipa{[{a}{\textsubring{l}}{\r{\textObardotlessj}}{ei}{}{\r{g}}{\textscy}{r}]}}{\color{darkgreen}{\small{ adj}}}{\textit{ (tíður)}}\foreignlanguage{czech}{{ obvyklý, častý, běžný}} $\triangleright$ {\textit{\textbf{ Sóleyjar eru algengar.}}}{\textit{\foreignlanguage{czech}{ Pryskyřičníky se běžně vyskytují.}}}}
\entry{{alger}}{{\textipa{[{a}{l}{\r{\textObardotlessj}}{\textepsilon}{r}]}}{\color{darkgreen}{\small{ adj}}}{ 1.}{\textit{ (einber)}}\foreignlanguage{czech}{{ úplný, naprostý, absolutní, vyložený, vyslovený}} $\triangleright$ {\textit{\textbf{ Maðurinn er alger asni.}}}{ 2.}{\textit{ (algerður)}}\foreignlanguage{czech}{{ hotový, dokončený}},}
\entry{{alger··lega}}{{\textipa{[{a}{l}{\r{\textObardotlessj}}{\textepsilon}{r}{l}{\textepsilon}{\textbabygamma}{a}]}}{\color{darkgreen}{\small{ adv}}}\foreignlanguage{czech}{{ zcela, úplně, dočista, naprosto}}}
\entry{{alger··legur}}{{\textipa{[{a}{l}{\r{\textObardotlessj}}{\textepsilon}{r}{l}{\textepsilon}{\textbabygamma}{\textscy}{r}]}}{\color{darkgreen}{\small{ adj}}}{\textit{ (fullkominn)}}\foreignlanguage{czech}{{ naprostý, absolutní, úplný}}}
\entry{{al··gildur}}{{\textipa{[{a}{\textsubring{l}}{\r{\textObardotlessj}}{\textsci}{l}{\textsubring{d}}{\textscy}{r}]}}{\color{darkgreen}{\small{ adj}}}\foreignlanguage{czech}{{ obecný, univerzální}}}
\entry{{algjör}}{{\textipa{[{a}{l}{\r{\textObardotlessj}}{\ae}{r}]}}{\color{darkgreen}{\small{ adj}}} $\rightarrow$       alger}
\entry{{algjör··lega}}{{\textipa{[{a}{l}{\r{\textObardotlessj}}{\ae}{r}{l}{\textepsilon}{\textbabygamma}{a}]}}{\color{darkgreen}{\small{ adv}}} $\rightarrow$       algerlega}
\entry{{al·gyðis··trú}}{{\textipa{[{a}{\textsubring{l}}{\r{\textObardotlessj}}{\textsci}{ð}{\textsci}{s}{t\textsuperscript{h}}{r}{u}]}}{\color{darkgreen}{\small{ f}}}{\small{ (-ar)}}\foreignlanguage{czech}{{\footnotesize{ náb.}}}\foreignlanguage{czech}{{ panteismus}}}
\entry{{al··heill}}{{\textipa{[{a}{\textlengthmark}{\textsubring{l}}{}{h}{ei}{\textsubring{d}}{\textsubring{l}}]}}{\color{darkgreen}{\small{ adj}}}{ 1.}{\textit{ (óskemmdur)}}\foreignlanguage{czech}{{ nepoškozený, nerozbitý, celý}}{ 2.}\foreignlanguage{czech}{{ zcela zdravý}},}
\entry{{al··heim|ur}}{{\textipa{[{a}{\textlengthmark}{\textsubring{l}}{}{h}{ei}{m}{\textscy}{r}]}}{\color{darkgreen}{\small{ m}}}{\small{ (-s)}}{\textit{ (geimur)}}\foreignlanguage{czech}{{\footnotesize{ astro.}}}\foreignlanguage{czech}{{ vesmír, kosmos}}\par\begin{center}\setlength\fboxsep{0pt}\setlength\fboxrule{0.5pt}\fbox{\includegraphics[width=6cm]{ds_image_alheimur_0_1.jpg}}\end{center}\par\begin{center}\footnotesize {Autor: NASA, ESA, and R. Massey (California Institute of Technology) Licence: Public Domain}\end{center}}
\entry{{al··hliða}}{{\textipa{[{a}{\textsubring{l}}{\textsci}{ð}{a}]}}{\color{darkgreen}{\small{ adj}}}{\small{ indecl}}{\textit{ (fjölhæfur)}}\foreignlanguage{czech}{{ všestranný, mnohostranný (umělec ap.)}} $\triangleright$ {\textit{\textbf{ vera alhliða leikari}}}}
\entry{{al··hug|ur}}{{\textipa{[{a}{\textlengthmark}{\textsubring{l}}{}{h}{\textscy}{\textbabygamma}{\textscy}{r}]}}{\color{darkgreen}{\small{ m}}}{\small{ (-ar)}}\foreignlanguage{czech}{{ upřímná myšlenka}}{\textsl{\textbf{ óska e-s af alhug}}}\foreignlanguage{czech}{{ přát si (co) celým srdcem}},}
\entry{{al··hæf|a}}{{\textipa{[{a}{\textlengthmark}{\textsubring{l}}{}{h}{a}{i}{v}{a}]}}{\color{darkgreen}{\small{ v}}}{\small{ (-ði, -t)}}\foreignlanguage{czech}{{ zevšeobecnit, zobecnit, generalizovat}}}
\entry{{al··hæfing}}{{\textipa{[{a}{\textlengthmark}{\textsubring{l}}{}{h}{a}{i}{v}{i}{}{\r{g}}]}}{\color{darkgreen}{\small{ f}}}{\small{ (-ar, -ar)}}\foreignlanguage{czech}{{ zevšeobecnění, zobecnění}}}
\entry{{ali··fugl}}{{\textipa{[{a}{\textlengthmark}{l}{\textsci}{f}{\textscy}{\r{g}}{\textsubring{l}}]}}{\color{darkgreen}{\small{ m}}}{\small{ (-s, -ar)}}\foreignlanguage{czech}{{\footnotesize{ zem.}}}\foreignlanguage{czech}{{ drůbež (pěstované pro maso a vejce)}}}
\entry{{alin}}{{\textipa{[{a}{\textlengthmark}{l}{\textsci}{n}]}}{\color{darkgreen}{\small{ f}}}{\small{ (álnar, álnir)}}\foreignlanguage{czech}{{\footnotesize{ hist.}}}\foreignlanguage{czech}{{ loket (délková míra)}}{\textit{ (eignir)}}{\textsl{\textbf{ álnir}}}{\footnotesize{ pl}}\foreignlanguage{czech}{{ majetek, bohatství}},{\textit{ (verða efnaður)}}{\textsl{\textbf{ komast í álnir}}}{\footnotesize{ přen.}}\foreignlanguage{czech}{{ přijít k penězům}},}
\entry{{alk|i}}{{\textipa{[{a}{\textsubring{l}}{\r{\textObardotlessj}}{\textsci}]}}{\color{darkgreen}{\small{ m}}}{\small{ (-a, -ar)}}\foreignlanguage{czech}{{\footnotesize{ neform.}}}\foreignlanguage{czech}{{ alkoholik}}}
\entry{{al··kominn}}{{\textipa{[{a}{\textsubring{l}}{k\textsuperscript{h}}{\textopeno}{m}{\textsci}{n}]}}{\color{darkgreen}{\small{ adj}}}\foreignlanguage{czech}{{ přišlý na pořád (o člověku ap.)}}}
\entry{{alkóhólist|i}}{{\textipa{[{a}{\textsubring{l}}{\r{g}}{ou}{ou}{l}{\textsci}{s}{\textsubring{d}}{\textsci}]}}{\color{darkgreen}{\small{ m}}}{\small{ (-a, -ar)}}\foreignlanguage{czech}{{ alkoholik}}}
\entry{{al··kunnur}}{{\textipa{[{a}{\textsubring{l}}{k\textsuperscript{h}}{\textscy}{n}{\textscy}{r}]}}{\color{darkgreen}{\small{ adj}}}{\textit{ (alþekktur)}}\foreignlanguage{czech}{{ všeobecně známý}}}
\entry{{alla··vega}}{{\textipa{[{a}{\textsubring{d}}{l}{a}{v}{\textepsilon}{\textbabygamma}{a}]}}{\color{darkgreen}{\small{ adv}}}{ 1.}{\textit{ (á allan hátt)}}\foreignlanguage{czech}{{ jakýmkoliv způsobem}}{ 2.}{\textit{ (að minnsta kosti)}}\foreignlanguage{czech}{{\footnotesize{ neform.}}}\foreignlanguage{czech}{{ alespoň, aspoň}},{ 3.}\foreignlanguage{czech}{{ v každém případě}},}
\entry{{alla··vegana}}{{\textipa{[{a}{\textsubring{d}}{l}{a}{v}{\textepsilon}{\textbabygamma}{a}{n}{a}]}}{\color{darkgreen}{\small{ adv}}} $\rightarrow$       allavega}
\entry{{all··mikill}}{{\textipa{[{a}{\textsubring{d}}{\textsubring{l}}{m}{\textsci}{\r{\textObardotlessj}}{\textsci}{\textsubring{d}}{\textsubring{l}}]}}{\color{darkgreen}{\small{ adj}}}\foreignlanguage{czech}{{ značný, velký}}}
\entry{{allra··handa}}{{\textipa{[{a}{\textsubring{d}}{l}{r}{a}{h}{a}{n}{\textsubring{d}}{a}]}}{\color{darkgreen}{\small{ adj}}}{\small{ indecl}}{\textit{ (margvíslegur)}}\foreignlanguage{czech}{{ různorodý, rozmanitý}}}
\entry{{alls}}{{\textipa{[{a}{l}{s}]}}{\color{darkgreen}{\small{ adv}}}{ 1.}{\textit{ (samtals)}}\foreignlanguage{czech}{{ vcelku, celkem}} $\triangleright$ {\textit{\textbf{ Í pokanum voru alls 10 epli.}}}{\textit{\foreignlanguage{czech}{ V sáčku bylo celkem 10 jablek.}}}{ 2.}{\textit{ (algerlega)}}\foreignlanguage{czech}{{ vyloženě, vysloveně, zcela}} $\triangleright$ {\textit{\textbf{ alls ónógur}}}{\textit{\foreignlanguage{czech}{ zcela nedostačující}}},{\textit{ (engan veginn)}}{\textsl{\textbf{ alls ekki}}}\foreignlanguage{czech}{{ vůbec, ani trochu}},}
\entry{{alls··ber}}{{\textipa{[{a}{l}{s}{\textsubring{b}}{\textepsilon}{r}]}}{\color{darkgreen}{\small{ adj}}}{\textit{ (kviknakinn)}}\foreignlanguage{czech}{{ dočista nahý, úplně nahý}}}
\entry{{alls··konar}}{{\textipa{[{a}{l}{s}{k\textsuperscript{h}}{\textopeno}{n}{a}{r}]}}{\color{darkgreen}{\small{ adj}}}{\small{ indecl}}\foreignlanguage{czech}{{ různorodý, různý}}}
\entry{{alls··laus}}{{\textipa{[{a}{l}{s}{l}{\ae i}{s}]}}{\color{darkgreen}{\small{ adj}}}\foreignlanguage{czech}{{ strádající, (jsoucí) na mizině}}}
\entry{{alls··ráðandi}}{{\textipa{[{a}{l}{s}{r}{au}{ð}{a}{n}{\textsubring{d}}{\textsci}]}}{\color{darkgreen}{\small{ adj}}}{\small{ indecl}}\foreignlanguage{czech}{{ vládnoucí všemu}}}
\entry{{alls··staðar}}{{\textipa{[{a}{l}{s}{\textsubring{d}}{a}{ð}{a}{r}]}}{\color{darkgreen}{\small{ adv}}}{\textit{ (hvar sem er)}}\foreignlanguage{czech}{{ všude}}{\textit{ (hvergi)}}}
\entry{{all··stór}}{{\textipa{[{a}{\textsubring{d}}{\textsubring{l}}{s}{\textsubring{d}}{ou}{r}]}}{\color{darkgreen}{\small{ adj}}}\foreignlanguage{czech}{{ značný, pořádný, slušný}}}
\entry{{allt}}{{\textipa{[{a}{\textsubring{l}}{\textsubring{d}}]}}{\color{darkgreen}{\small{ adv}}}{\textsl{\textbf{ allt í einu}}}\foreignlanguage{czech}{{ najednou, náhle, znenadání  }} $\triangleright$ {\textit{\textbf{ Óveðrið brast á allt í einu. }}}{\textit{\foreignlanguage{czech}{ Bouře z ničeho nic propukla.}}}{\textit{ (samtals)}}{\textsl{\textbf{ í allt}}}\foreignlanguage{czech}{{ celkem}} $\triangleright$ {\textit{\textbf{ Þetta voru 35 epli í allt.}}}{\textit{\foreignlanguage{czech}{ Celkem to bylo 35 jablek.}}},{\textit{ (alla leið)}}\foreignlanguage{czech}{{ celou cestu}} $\triangleright$ {\textit{\textbf{ allt út á nes}}}{\textit{\foreignlanguage{czech}{ až na mys}}},{\textit{ (alveg)}}\foreignlanguage{czech}{{ celou dobu}} $\triangleright$ {\textit{\textbf{ allt frá því}}}{\textit{\foreignlanguage{czech}{ celou dobu}}},{\textit{ (næstum)}}{\textsl{\textbf{ allt að (því)}}}\foreignlanguage{czech}{{ téměř}} $\triangleright$ {\textit{\textbf{ Ferðin var allt að tveimur tímum.}}}{\textit{\foreignlanguage{czech}{ Cesta trvala téměř dvě hodiny.}}},{\textsl{\textbf{ allt annað}}}\foreignlanguage{czech}{{ úplně něco jiného}},{\textit{ (þrátt fyrir)}}{\textsl{\textbf{ allt svo}}}{\footnotesize{ conj}}\foreignlanguage{czech}{{ jakkoliv}},{\textsl{\textbf{ allt svo lengi}}}{\footnotesize{ conj}}\foreignlanguage{czech}{{ dokud}},}
\entry{{all··taf}}{{\textipa{[{a}{\textsubring{d}}{\textsubring{l}}{t\textsuperscript{h}}{a}{f}]}}{\color{darkgreen}{\small{ adv}}}{\textit{ (sífellt)}}\foreignlanguage{czech}{{ vždy, (neu)stále}} $\triangleright$ {\textit{\textbf{ Eldurinn er alltaf heitur.}}}{\textit{\foreignlanguage{czech}{ Oheň je stále horký.}}}}
\entry{{all··tént}}{{\textipa{[{a}{\textsubring{d}}{\textsubring{l}}{t\textsuperscript{h}}{j}{\textepsilon}{\textsubring{n}}{\textsubring{d}}]}}{\color{darkgreen}{\small{ adv}}} $\rightarrow$       alltjent}
\entry{{allt··jent}}{{\textipa{[{a}{\textsubring{l}}{\textsubring{d}}{j}{\textepsilon}{\textsubring{n}}{\textsubring{d}}]}}{\color{darkgreen}{\small{ adv}}}{ 1.}{\textit{ (hvað sem öðru líður)}}\foreignlanguage{czech}{{ přece, beztak, stejně}}{ 2.}{\textit{ (alltaf)}}\foreignlanguage{czech}{{ vždy}},}

\end{document}
